\documentclass[a4paper,12pt]{article}

\usepackage[T2A]{fontenc} % ru lang
% \usepackage[utf8]{inputenc}
\usepackage[english,russian]{babel} % ru lang

\usepackage[linesnumbered,ruled,vlined]{algorithm2e}

\usepackage{amsmath} % align*
\usepackage{float} % tables not float
\usepackage{indentfirst} % ru style
\usepackage{booktabs} % table horisontal lines
\usepackage{derivative} % partial \pdv

\usepackage[newfloat]{minted} % listings

\usepackage[top=3cm,bottom=3cm,left=2cm,right=2cm]{geometry} % large formulas not overful hbox
\setcounter{MaxMatrixCols}{25} % pmatrix


\begin{document}
	\begin{titlepage}	% начало титульной страницы

	\begin{center}		% выравнивание по центру

		\large Санкт-Петербургский политехнический университет Петра Великого\\
		\large Институт компьютерных наук и кибербезопасности\\
		\large Высшая школа программной инженерии \\[6cm]
		% название института, затем отступ 6см

    \huge Курсовая Работа\\[0.5cm] % название работы, затем отступ 0,5см
		\large Разработка программного обеспечения для воспроизведения на ЭВМ двумерных стационарных моделей \\[0.1cm]
		\large по дисциплине \\
		\large <<Разработка программного обеспечения для моделирования физических процессов>> \\[5cm]

	\end{center}

		\noindent\large Выполнил: \hfill \large Клюкин С. А.\\
		\noindent\large Группа: \hfill \large гр. 5130904/10102\\

		\noindent\large Проверил: \hfill \large Воскобойников С. П.

	\vfill % заполнить всё доступное ниже пространство

	\begin{center}
	\large Санкт-Петербург\\
	\large \the\year % вывести дату
	\end{center} % закончить выравнивание по центру

\end{titlepage} % конец титульной страницы

\vfill % заполнить всё доступное ниже пространство

	\newpage
	\tableofcontents
	\newpage

	\section{Вступление}
	\subsection{Постановка задачи}

	Вариант C3. Используя интегро-интерполяционный метод (метод баланса), разработать программу для моделирования нестационарного распределения температуры в цилиндре, описываемого математической моделью вида:
 
	\begin{align*}
		&\frac{\partial u}{\partial t} = \frac{1}{r} \frac{\partial}{\partial r}
		\left ( rk(r, t)\frac{\partial u}{\partial r} \right ) - q(r, t)u + f(r,t),
		\ r \in \left[ 0, R\right],\ t \in [0, T],
		\\
		&0 < c_1 \leq k(r, t) \leq c_2,\ 0 \leq q(r, t)
	\end{align*}

	Начальное условие:
	\[
		\left. u \right\vert_{t=0} = \varphi(r)
	\]

	Граничные условия: 
 
    \centerline{\(\left. u\right\vert_{r = 0}\) - ограничено}
    \[-k \left. \frac{\partial u}{\partial r}\right\vert_{r = R} = \chi\left.u\right\vert_{r = R} - \nu, \chi > 0\]
    
	\subsection{Используемое ПО}

	\begin{enumerate}
    \item Python 3.10.12
    \item numpy 1.26.2
    \end{enumerate}
	\newpage

	\section{Основная часть}
	\section{Разностная схема}
Введем основную сетку:
\begin{align*}
  &N_r - \text{число разбиений на } [R_0, R_1] & &N_z - \text{число разбиений на } [0, L] \\
  &r_0 < r_1 < \cdots < r_N & &z_0 < z_1 < \cdots < z_N \\
  &r_0 = R_0,\quad r_N = R_1 & &z_0 = 0,\quad z_N = L \\
  &h_r = \frac{R_1 - R_0}{N_r} & &h_z = \frac{L - 0}{N_z}
\end{align*}

Введем дополнительную сетку:
\begin{align*}
  &r_{i-\frac{1}{2}} = \frac{r_i + r_{i - 1}}{2}\quad i=1,\cdots, N_r & &z_{j-\frac{1}{2}} = \frac{z_j + z_{j - 1}}{2}\quad j=1,\cdots, N_z \\
  & \hbar_i = \begin{cases}
    \frac{h_r}{2},\ i = 0 \\ \\
    h_r,\ i = 1, 2, \dots, N_r-1 \\ \\
    \frac{h_r}{2},\ i = N_r
  \end{cases} &
  & \hbar_j = \begin{cases}
    \frac{h_z}{2},\ j = 0 \\ \\
    h_z,\ j = 1, 2, \dots, N_z-1 \\ \\
    \frac{h_z}{2},\ j = N_z
  \end{cases}
\end{align*}

Преобразуем наше начальное уравнение домножив на r

\[
  - \left [ \pdv{}{r} \left ( r k(r) \pdv{u}{r} \right ) 
  + \pdv[2]{ru}{z} \right ] = rf(r, z)
\]

\subsection{Внутренние точки}
% Проинтегрируем уравнение (\hyperref[eq1]{1}) в области
$ [r_{i -\frac{1}{2}}, r_{i +\frac{1}{2}}] \times  [z_{j -\frac{1}{2}}, z_{j +\frac{1}{2}}] $
для $ i = 1, 2, \dots, N_r - 1 $ и $ j = 1, 2, \dots, N_z - 1$.

\[
  - \mLim{+\frac{1}{2}}{-\frac{1}{2}}{+\frac{1}{2}}{-\frac{1}{2}} \left [ \pdv{}{r} \left ( r k(r) \pdv{u}{r} \right ) 
  + \pdv[2]{ru}{z} \right ] dr dz = \mLim{+\frac{1}{2}}{-\frac{1}{2}}{+\frac{1}{2}}{-\frac{1}{2}} rf(r, z) dr dz
\]

Получим:

\begin{align*}
  &- \left [
   \mLimZ{z}{+\frac{1}{2}}{-\frac{1}{2}}{j}  \left . r k(r) \pdv{u}{r} \right \vert_{r = r_{i + \frac{1}{2}}} dz
  - \mLimZ{z}{+\frac{1}{2}}{-\frac{1}{2}}{j} \left . r k(r) \pdv{u}{r} \right \vert_{r = r_{i - \frac{1}{2}}} dz
  \right . \\
  &\left . + \mLimS{r}{+\frac{1}{2}}{-\frac{1}{2}} \left . r \pdv{u}{z} \right \vert_{z = z_{j + \frac{1}{2}}} dr
  - \mLimS{r}{+\frac{1}{2}}{-\frac{1}{2}} \left . r \pdv{u}{z} \right \vert_{z = z_{j - \frac{1}{2}}} dr
  \right ] = \mLim{+\frac{1}{2}}{-\frac{1}{2}}{+\frac{1}{2}}{-\frac{1}{2}} rf(r, z) dr dz
\end{align*}

Воспользуемся формулами численного дифференцирования:
\[
  \left . k(r) \pdv{u}{r} \right \vert_{r = r_{i - \frac{1}{2}}}
  \approx k(r_{i - \frac{1}{2}}) \frac{v_{i, j} - v_{i - 1, j}}{h_r}
\]

\[
  \left . \pdv{u}{r} \right \vert_{z = z_{j - \frac{1}{2}}}
  \approx \frac{v_{i, j} - v_{i, j - 1}}{h_z}
\]

Также воспользуемся формулой средних прямоугольников:
\[
  \mLimS{r}{+\frac{1}{2}}{-\frac{1}{2}} r \varphi(r, z) dr
  = \hbar_i r_i \varphi_i
\]
\[
  \mLim{+\frac{1}{2}}{-\frac{1}{2}}{+\frac{1}{2}}{-\frac{1}{2}} r \varphi(r, z) drdz
  = \hbar_i\hbar_j r_i \varphi_{i, j}
\]

В итоге получим уравнение разностной схемы:
\begin{align*}
  &- \left [ 
  \hbar_j r_{i+\frac{1}{2}} k(r_{i+\frac{1}{2}}) \frac{v_{i+1, j} - v_{i, j}}{h_{r}}
  - \hbar_j r_{i-\frac{1}{2}} k(r_{i-\frac{1}{2}}) \frac{v_{i, j} - v_{i - 1, j}}{h_{r}}
  \right . \\
  &\left .
  + \hbar_i r_{i} \frac{v_{i, j + 1} - v_{i, j}}{h_{z}}
  - \hbar_i r_{i} \frac{v_{i, j} - v_{i, j - 1}}{h_z}
  \right ]  = \hbar_i \hbar_j r_i f_{i, j}
\end{align*}

Так как выбранная основная сетка является равномерной, то $ \hbar_i = h_r $ и $ \hbar_j = h_z$
, для $ i = 1, 2, \dots, N_r - 1 $ и $ j = 1, 2, \dots, N_z - 1$.

\begin{align*}
  &- \left [ 
  h_z r_{i+\frac{1}{2}} k(r_{i+\frac{1}{2}}) \frac{v_{i+1, j} - v_{i, j}}{h_{r}}
  - h_z r_{i-\frac{1}{2}} k(r_{i-\frac{1}{2}}) \frac{v_{i, j} - v_{i - 1, j}}{h_{r}}
  \right . \\
  &\left .
  + h_r r_{i} \frac{v_{i, j + 1} - v_{i, j}}{h_{z}}
  - h_r r_{i} \frac{v_{i, j} - v_{i, j - 1}}{h_z}
  \right ]  = h_r h_z r_i f_{i, j}
\end{align*}

Умножим на $\frac{h_z}{h_r r_i}$, чтобы получилась подходящая для применяемого метода решения СЛАУ форма.
\begin{align*}
  - \left [ 
  h_z^2 \frac{r_{i+\frac{1}{2}}}{r_i} k(r_{i+\frac{1}{2}}) \frac{v_{i+1, j} - v_{i, j}}{h_{r}^2}
  - h_z^2 \frac{r_{i-\frac{1}{2}}}{r_i} k(r_{i-\frac{1}{2}}) \frac{v_{i, j} - v_{i - 1, j}}{h_{r}^2}
  + v_{i, j + 1} - 2 v_{i, j} + v_{i, j - 1}
  \right ]  = h_z^2 f_{i, j}
\end{align*}

\subsection{На левой границе}

$ [r_{i}, r_{i +\frac{1}{2}}] \times  [z_{j -\frac{1}{2}}, z_{j +\frac{1}{2}}] $
для $ i = 0 $ и $ j = 1, 2, \dots, N_z - 1$.

Получаем:
\begin{align*}
  &- \left [
   \mLimZ{z}{+\frac{1}{2}}{-\frac{1}{2}}{j}  \left . r k(r) \pdv{u}{r} \right \vert_{r = r_{i + \frac{1}{2}}} dz
  - \mLimZ{z}{+\frac{1}{2}}{-\frac{1}{2}}{j} \left . r k(r) \pdv{u}{r} \right \vert_{r = r_{i}} dz
  \right . \\
  &\left . + \mLimS{r}{+\frac{1}{2}}{} \left . r\pdv{u}{z} \right \vert_{z = z_{j + \frac{1}{2}}} dr
  - \mLimS{r}{+\frac{1}{2}}{} \left . r\pdv{u}{z} \right \vert_{z = z_{j - \frac{1}{2}}} dr
  \right ] = \mLim{+\frac{1}{2}}{}{+\frac{1}{2}}{-\frac{1}{2}} rf(r, z) dr dz
\end{align*}

Имеем граничное условие:
\[ \left . k(r) \pdv{u}{r} \right \vert_{r=R_0} = \chi_1 \left . u \right \vert_{r=R_0} - \varphi_1(z) \]

Получаем уравнение разностной схемы:

\begin{align*}
  &- \left [ 
  h_z r_{i+\frac{1}{2}} k(r_{i+\frac{1}{2}}) \frac{v_{i+1, j} - v_{i, j}}{h_{r}}
  - h_z r_{i} (\chi_1(z_j) v_{i, j} - \varphi_1(z_j))
  \right . \\
  &\left .
  + \frac{h_r}{2} r_{i} \frac{v_{i, j + 1} - v_{i, j}}{h_{z}}
  - \frac{h_r}{2} r_{i} \frac{v_{i, j} - v_{i, j - 1}}{h_z}
  \right ]  = \frac{h_r}{2} h_z r_i f_{i, j}
\end{align*}

\begin{align*}
  - \left [ 
  2 h_z^2 \frac{r_{i+\frac{1}{2}}}{r_i} k(r_{i+\frac{1}{2}}) \frac{v_{i+1, j} - v_{i, j}}{h_{r}^2}
  - 2 h_z^2 \frac{\chi_1(z_j) v_{i, j} - \varphi_1(z_j)}{h_r}
  + v_{i, j + 1} - 2 v_{i, j} + v_{i, j - 1}
  \right ]  = h_z^2 f_{i, j}
\end{align*}

\subsection{На правой границе}
$ [r_{i -\frac{1}{2}}, r_{i}] \times  [z_{j -\frac{1}{2}}, z_{j +\frac{1}{2}}] $
для $ i = N_r $ и $ j = 1, 2, \dots, N_z - 1$.

Имеем граничное условие:
\[ \left . u \right \vert_{r=R_1} = \varphi_2(z) \]

Будем его сразу использовать:
\[ v_{N_r,j} = \varphi_2(z_j) \]

\subsection{На нижней границе}
$ [r_{i  -\frac{1}{2}}, r_{i +\frac{1}{2}}] \times  [z_{j}, z_{j +\frac{1}{2}}] $
для $ i = 1, 2, \dots, N_r - 1 $ и $ j = 0$.

\[ \left . u \right \vert_{z=0} = \varphi_3(r) \]
\[ v_{i,0} = \varphi_3(r_i) \]

\subsection{На верхней границе}
$ i = 1, 2, \dots, N_r - 1 $ и $ j = N_z $
\[ \left . u \right \vert_{z=L} = \varphi_4(r) \]
\[ v_{i,N_z} = \varphi_4(r_i) \]

\subsection{Левый-нижний угол}

$ [r_{i}, r_{i +\frac{1}{2}}] \times  [z_{j}, z_{j +\frac{1}{2}}] $
для $ i = 0 $ и $ j = 0$.
\[ v_{0,0} = \varphi_3(r_0) \]

\subsection{Левый-верхний угол}

$ i = 0 $ и $ j = N_z $

\[ v_{0,N_z} = \varphi_4(r_0) \]

\subsection{Правый-верхний угол}

$ i = N_r $ и $ j = N_z $, возьмём известное граничное условие:

\[ v_{N_r,N_z} = \varphi_4(r_{N_r}) \]

\subsection{Правый-нижний угол}

$ i = N_r $ и $ j = 0 $.

\[ v_{N_r,0} = \varphi_2(z_0) \]

\subsection{На нижней границе}
$ [r_{i  -\frac{1}{2}}, r_{i +\frac{1}{2}}] \times  [z_{j}, z_{j +\frac{1}{2}}] $
для $ i = 1, 2, \dots, N_r - 1 $ и $ j = 0$.

\[ \left . u \right \vert_{z=0} = \varphi_3(r) \]
\[ v_{i,0} = \varphi_3(r_i) \]

\subsection{На верхней границе}
$ i = 1, 2, \dots, N_r - 1 $ и $ j = N_z $
\[ \left . u \right \vert_{z=L} = \varphi_4(r) \]
\[ v_{i,N_z} = \varphi_4(r_i) \]

	\subsubsection{Явный метод Эйлера}

Мы имеем уравнение:
\[
  v(t_{n+1}) = v(t_n) + \int\limits_{t_{n}}^{t_n+1} (A(t)v(t) + g(t)) dt
\]

Интерполлируем интеграл по формуле левых треугольников:
\[
  \int\limits_{a}^{b} f(x) \mathrm{d} x \approx f(a) (b - a)
\]

Тем самым получаем:
\[
  v(t_{n+1}) = v(t_n) + (t_{n+1} - t_n) A(t_n) v(t_n) + (t_{n+1} - t_n) g(t_n)
\]

Введем обозначение:
\[
  H = t_{n+1} - t_n
\]
\[
  v(t_{n+1}) = v(t_n) + H A(t_n) v(t_n) + H g(t_n)
\]
\[
  v(t_{n+1}) = (E + HA(t_n))v(t_n) + Hg(t_n)
\]

Явный метод ломанных Эйлера:
\[
  \begin{cases}
    v(t_{n+1}) = (E + HA(t_n))v(t_n) + Hg(t_n) \\
    v(t_0) = \varphi(r)
  \end{cases}
\]

\subsubsection{Неявный метод Эйлера}

Теперь проинтерполируем интеграл формулой правых треугольников:
\[
  \int\limits_{a}^{b} f(x) \mathrm{d} x \approx f(b) (b - a)
\]

Получаем:
\[
  v(t_{n+1}) = v(t_n) + HA(t_{n+1}) v(t_{n+1}) + Hg(t_{n+1})
\]
\[
  (E - HA(t_{n+1})) v(t_{n+1}) = v(t_n) + Hg(t_{n+1})
\]

Неявный метод ломанных Эйлера:
\[
  \begin{cases}
    (E - HA(t_{n+1})) v(t_{n+1}) = v(t_n) + Hg(t_{n+1}) \\
    v(t_0) = \varphi(r)
  \end{cases}
\]

	\subsection{Тестирование}

Погрешность решения определим следующим образом:
\[
  \max_{t_0, \cdots, t_n} \left\lVert \tilde{v}(t) - v(t) \right\rVert ,
\]
где норма вектора берется как его максимальный по модулю элемент.

Для всех тестов возьмем одинаковый интервал для r и t.
\[
  r \in [0, 1]
\]
\[
  t \in [0, 1]
\]

Далее для каждого теста будем задавать k, q, u, $\chi$. $\phi(r)$ будем определять как $u(r, 0)$. Далее будем выражать f, $\nu$ и проводить запуск с вычислением погрешности для разных N и H с помощью обоих методов.

\subsubsection{Стационарное решение}
\[  u(r, t) = 1 \]
\[  \chi(r, t) = 1 \]
\[  k(r, t) = 3r+2 \]
\[  q(r, t) = 4r+3 \]
Выполним запуск явного метода для N из множества [8, 16, 32, 64] и H из множества [0.1, 0.01, 0.001, 0.0001]. Получаются следующие значения погрешности: [[1213806000000.0, inf, 0.0, 0.0], [5.230617e+18, inf, inf, 0.0], [4.672042e+24, inf, inf, 3.5762787e-07], [1.1984469e+31, inf, inf, inf]]. Покажем их графически:
\begin{center}
  \includegraphics[width=0.7\textwidth]{img/4exp4.png}
\end{center}
Черные квадраты означают большую погрешность, а белые - малую.

Покажем этот же результат в таблице:
\begin{table}[H]
  \centering
  \begin{tabular}{c | c | c | c | c}
    \toprule
  & 0.1 & 0.01 & 0.001 & 0.0001 \\ 
\midrule
8 & 1.21E+12 & INF & 0.00E+00 & 0.00E+00 \\ 
16 & 5.23E+18 & INF & INF & 0.00E+00 \\ 
32 & 4.67E+24 & INF & INF & 3.58E-07 \\ 
64 & 1.20E+31 & INF & INF & INF \\ 
    \bottomrule
  \end{tabular}
  \caption{Погрешность явного стационарного теста}
\end{table}
Словом INF обозначим ячейки, в которых из-за взрыва погрешности произошло переполнение. По горизонтали откладываем H, а по вертикали N.

Запустим неявный метод.
[8, 16, 32, 64, 128],
[0.1, 0.01, 0.001, 0.0001, 1e-05],
\begin{center}
  \includegraphics[width=0.7\textwidth]{img/4imp5.png}
\end{center}
\begin{table}[H]
  \centering
  \begin{tabular}{c | c | c | c | c | c}
    \toprule
  & 0.1 & 0.01 & 0.001 & 0.0001 & 1e-05 \\ 
\midrule
8 & 1.67E-06 & 7.75E-07 & 5.96E-08 & 1.43E-06 & 5.07E-06 \\ 
16 & 9.60E-06 & 9.12E-06 & 1.19E-07 & 5.96E-07 & 8.46E-06 \\ 
32 & 6.44E-06 & 1.55E-06 & 1.07E-06 & 4.77E-07 & 1.43E-06 \\ 
64 & 4.53E-06 & 1.43E-06 & 3.60E-05 & 9.54E-07 & 8.34E-07 \\ 
128 & 1.42E-05 & 7.78E-05 & 3.70E-06 & 5.96E-06 & 1.55E-06 \\ 
    \bottomrule
  \end{tabular}
  \caption{Погрешность неявного стационарного теста}
\end{table}


\subsubsection{Нестационарное решение}
\[  u(r, t) = r^2 + t^2 \]
\[  \chi(r, t) = 2t+1 \]
\[  k(r, t) = \frac{cos(r)}{2} + 3 \]
\[  q(r, t) = \frac{sin(r)}{2} + 2 \]
Явный метод.
[8, 16, 32, 64],
[0.1, 0.01, 0.001, 0.0001, 1e-05, 1e-06],
\begin{center}
  \includegraphics[width=0.7\textwidth]{img/3exp_4x6.png}
\end{center}
\begin{table}[H]
  \centering
  \begin{tabular}{c | c | c | c | c | c | c}
    \toprule
  & 0.1 & 0.01 & 0.001 & 0.0001 & 1e-05 & 1e-06 \\ 
\midrule
8 & 1.14E+15 & INF & 6.28E-03 & 6.85E-03 & 6.84E-03 & 6.43E-03 \\ 
16 & 1.49E+20 & INF & INF & 1.66E-03 & 1.67E-03 & 1.29E-03 \\ 
32 & 1.91E+25 & INF & INF & 3.75E-04 & 3.90E-04 & 1.56E-03 \\ 
64 & 2.46E+30 & INF & INF & INF & 8.87E-05 & 1.65E-03 \\ 
    \bottomrule
  \end{tabular}
  \caption{Погрешность явного нестационарного теста}
\end{table}

Неявный метод.
[8, 16, 32, 64],
[0.1, 0.01, 0.001, 0.0001],
\begin{center}
  \includegraphics[width=0.7\textwidth]{img/3imp4.png}
\end{center}
\begin{table}[H]
  \centering
  \begin{tabular}{c | c | c | c | c}
    \toprule
  & 0.1 & 0.01 & 0.001 & 0.0001 \\ 
\midrule
8 & 1.26E-02 & 5.37E-03 & 6.77E-03 & 6.91E-03 \\ 
16 & 1.70E-02 & 4.11E-04 & 1.58E-03 & 1.70E-03 \\ 
32 & 1.81E-02 & 1.53E-03 & 2.82E-04 & 4.00E-04 \\ 
64 & 1.84E-02 & 1.87E-03 & 9.42E-05 & 1.20E-04 \\ 
    \bottomrule
  \end{tabular}
  \caption{Погрешность неявного нестационарного теста}
\end{table}

Из результатов тестирования можно сделать несколько выводов:
\begin{enumerate}
  \item В явном методе во всех точках левее некоторой прямой, проведенной на диаграмме, происходит взрыв погрешности.
  \item В явном методе при $H = 0.1$ взрыв погрешности сравнительно небольшой особенно для малых N.
  \item Уменьшение H или увеличение N при фиксированном другом параметре не всегда приводит к уменьшению погрешности.
\end{enumerate}

Первое явление можно объяснить тем, какое ограничение на N и H явный метод накладывает для достижения устойчивости. Из теории знаем:

\[ \tau < \frac{2}{|\lambda|_{max}} \]
\[ |\lambda|_{max} < ||A|| \]
Далее из вида коэффициентов a, b, c получаем
\[ ||A|| \sim \frac{1}{h^2},\]
где h - шаг при равномерной сетке.

Тогда $\tau < h^2$ или в нашем случае $H < N^{-2}$. И действительно, на диаграммах границу можно представить как прямую $-3lg(H) = 2lb(N)$

Второе явление можно объяснить тем, что при $H = 0.1$ на отрезках времени шириной 0.1 рост ошибки линейный. В случае $H = 0.01$ в течение времени 0.1 наблюдается экспонента. Зависимость от N можно объяснить тем, как ведет себя наибольший модуль собственных значений матрицы А в зависимости от h.

Неубывание начиная с определенного момента погрешности при уменьшении H при постоянном N можно объяснить тем, что погрешность численного интегрирования стремится к нулю, а погрешность разностной схемы остается постоянной.

Неубывание начиная с определенного момента погрешности при увеличении N при постоянном H можно объяснить тем, что погрешность разностной схемы стремится к нулю, а погрешность интегрирования не убывает.

Все вычисления выполнены с одинарной точностью.
Для визуализации стационарного теста на диаграммах от погрешностей был взят логарифм.

	\newpage

	\section{Заключение}
	\subsection{Вывод}
	Задание выполнено полностью. Были написаны: вычисление коэффициентов разностной схемы, явный и неявный метод Эйлера.
	Оба метода были протестированы. Были выявлены особенности выбора метода и параметров дискретизации.
	\newpage
	\subsection{Код}
	\inputminted{python}{main.py}
\end{document}
