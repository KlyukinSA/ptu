\subsection{Тестирование}

Погрешность решения определим следующим образом:
\[
  \max_{t_0, \cdots, t_n} \left\lVert \tilde{v}(t) - v(t) \right\rVert ,
\]
где норма вектора берется как его максимальный по модулю элемент.

Для всех тестов возьмем одинаковый интервал для r и t.
\[
  r \in [0, 1]
\]
\[
  t \in [0, 1]
\]

Далее для каждого теста будем задавать k, q, u, $\chi$. $\phi(r)$ будем определять как $u(r, 0)$. Далее будем выражать f, $\nu$ и проводить запуск с вычислением погрешности для разных N и H с помощью обоих методов.

\subsubsection{Стационарное решение}
\[  u(r, t) = 1 \]
\[  \chi(r, t) = 1 \]
\[  k(r, t) = 3r+2 \]
\[  q(r, t) = 4r+3 \]
Выполним запуск явного метода для N из множества [8, 16, 32, 64] и H из множества [0.1, 0.01, 0.001, 0.0001]. Получаются следующие значения погрешности: [[1213806000000.0, inf, 0.0, 0.0], [5.230617e+18, inf, inf, 0.0], [4.672042e+24, inf, inf, 3.5762787e-07], [1.1984469e+31, inf, inf, inf]]. Покажем их графически:
\begin{center}
  \includegraphics[width=0.7\textwidth]{img/4exp4.png}
\end{center}
Черные квадраты означают большую погрешность, а белые - малую.

Покажем этот же результат в таблице:
\begin{table}[H]
  \centering
  \begin{tabular}{c | c | c | c | c}
    \toprule
  & 0.1 & 0.01 & 0.001 & 0.0001 \\ 
\midrule
8 & 1.21E+12 & INF & 0.00E+00 & 0.00E+00 \\ 
16 & 5.23E+18 & INF & INF & 0.00E+00 \\ 
32 & 4.67E+24 & INF & INF & 3.58E-07 \\ 
64 & 1.20E+31 & INF & INF & INF \\ 
    \bottomrule
  \end{tabular}
  \caption{Погрешность явного стационарного теста}
\end{table}
Словом INF обозначим ячейки, в которых из-за взрыва погрешности произошло переполнение. По горизонтали откладываем H, а по вертикали N.

Запустим неявный метод.
[8, 16, 32, 64, 128],
[0.1, 0.01, 0.001, 0.0001, 1e-05],
\begin{center}
  \includegraphics[width=0.7\textwidth]{img/4imp5.png}
\end{center}
\begin{table}[H]
  \centering
  \begin{tabular}{c | c | c | c | c | c}
    \toprule
  & 0.1 & 0.01 & 0.001 & 0.0001 & 1e-05 \\ 
\midrule
8 & 1.67E-06 & 7.75E-07 & 5.96E-08 & 1.43E-06 & 5.07E-06 \\ 
16 & 9.60E-06 & 9.12E-06 & 1.19E-07 & 5.96E-07 & 8.46E-06 \\ 
32 & 6.44E-06 & 1.55E-06 & 1.07E-06 & 4.77E-07 & 1.43E-06 \\ 
64 & 4.53E-06 & 1.43E-06 & 3.60E-05 & 9.54E-07 & 8.34E-07 \\ 
128 & 1.42E-05 & 7.78E-05 & 3.70E-06 & 5.96E-06 & 1.55E-06 \\ 
    \bottomrule
  \end{tabular}
  \caption{Погрешность неявного стационарного теста}
\end{table}


\subsubsection{Нестационарное решение}
\[  u(r, t) = r^2 + t^2 \]
\[  \chi(r, t) = 2t+1 \]
\[  k(r, t) = \frac{cos(r)}{2} + 3 \]
\[  q(r, t) = \frac{sin(r)}{2} + 2 \]
Явный метод.
[8, 16, 32, 64],
[0.1, 0.01, 0.001, 0.0001, 1e-05, 1e-06],
\begin{center}
  \includegraphics[width=0.7\textwidth]{img/3exp_4x6.png}
\end{center}
\begin{table}[H]
  \centering
  \begin{tabular}{c | c | c | c | c | c | c}
    \toprule
  & 0.1 & 0.01 & 0.001 & 0.0001 & 1e-05 & 1e-06 \\ 
\midrule
8 & 1.14E+15 & INF & 6.28E-03 & 6.85E-03 & 6.84E-03 & 6.43E-03 \\ 
16 & 1.49E+20 & INF & INF & 1.66E-03 & 1.67E-03 & 1.29E-03 \\ 
32 & 1.91E+25 & INF & INF & 3.75E-04 & 3.90E-04 & 1.56E-03 \\ 
64 & 2.46E+30 & INF & INF & INF & 8.87E-05 & 1.65E-03 \\ 
    \bottomrule
  \end{tabular}
  \caption{Погрешность явного нестационарного теста}
\end{table}

Неявный метод.
[8, 16, 32, 64],
[0.1, 0.01, 0.001, 0.0001],
\begin{center}
  \includegraphics[width=0.7\textwidth]{img/3imp4.png}
\end{center}
\begin{table}[H]
  \centering
  \begin{tabular}{c | c | c | c | c}
    \toprule
  & 0.1 & 0.01 & 0.001 & 0.0001 \\ 
\midrule
8 & 1.26E-02 & 5.37E-03 & 6.77E-03 & 6.91E-03 \\ 
16 & 1.70E-02 & 4.11E-04 & 1.58E-03 & 1.70E-03 \\ 
32 & 1.81E-02 & 1.53E-03 & 2.82E-04 & 4.00E-04 \\ 
64 & 1.84E-02 & 1.87E-03 & 9.42E-05 & 1.20E-04 \\ 
    \bottomrule
  \end{tabular}
  \caption{Погрешность неявного нестационарного теста}
\end{table}

Из результатов тестирования можно сделать несколько выводов:
\begin{enumerate}
  \item В явном методе во всех точках левее некоторой прямой, проведенной на диаграмме, происходит взрыв погрешности.
  \item В явном методе при $H = 0.1$ взрыв погрешности сравнительно небольшой особенно для малых N.
  \item Уменьшение H или увеличение N при фиксированном другом параметре не всегда приводит к уменьшению погрешности.
\end{enumerate}

Первое явление можно объяснить тем, какое ограничение на N и H явный метод накладывает для достижения устойчивости. Из теории знаем:

\[ \tau < \frac{2}{|\lambda|_{max}} \]
\[ |\lambda|_{max} < ||A|| \]
Далее из вида коэффициентов a, b, c получаем
\[ ||A|| \sim \frac{1}{h^2},\]
где h - шаг при равномерной сетке.

Тогда $\tau < h^2$ или в нашем случае $H < N^{-2}$. И действительно, на диаграммах границу можно представить как прямую $-3lg(H) = 2lb(N)$

Второе явление можно объяснить тем, что при $H = 0.1$ на отрезках времени шириной 0.1 рост ошибки линейный. В случае $H = 0.01$ в течение времени 0.1 наблюдается экспонента. Зависимость от N можно объяснить тем, как ведет себя наибольший модуль собственных значений матрицы А в зависимости от h.

Неубывание начиная с определенного момента погрешности при уменьшении H при постоянном N можно объяснить тем, что погрешность численного интегрирования стремится к нулю, а погрешность разностной схемы остается постоянной.

Неубывание начиная с определенного момента погрешности при увеличении N при постоянном H можно объяснить тем, что погрешность разностной схемы стремится к нулю, а погрешность интегрирования не убывает.

Все вычисления выполнены с одинарной точностью.
Для визуализации стационарного теста на диаграммах от погрешностей был взят логарифм.
