\documentclass[a4paper,12pt]{article}

\usepackage[T2A]{fontenc} % ru lang
% \usepackage[utf8]{inputenc}
\usepackage[english,russian]{babel} % ru lang

\usepackage[linesnumbered,ruled,vlined]{algorithm2e}

\usepackage{amsmath} % align*
\usepackage{float} % tables not float
\usepackage{indentfirst} % ru style
\usepackage{booktabs} % table horisontal lines
\usepackage{derivative} % partial \pdv

\usepackage[newfloat]{minted} % listings

\usepackage[top=3cm,bottom=3cm,left=2cm,right=2cm]{geometry} % large formulas not overful hbox
\setcounter{MaxMatrixCols}{25} % pmatrix



\begin{document}
\newcommand\mLim[4]{
  \int\limits^{r_{i #1}}_{r_{i #2}}
  \int\limits^{z_{j #3}}_{z_{j #4}}
}

\newcommand\Int[2]{
  \int\limits^{#1}_{#2}
}

\newcommand\mLimS[3]
{
  \int\limits^{#1_{i#2}}_{#1_{i#3}}
}

\newcommand\mLimZ[4]
{
  \int\limits^{#1_{#4 #2}}_{#1_{#4 #3}}
}

\begin{titlepage}	% начало титульной страницы

	\begin{center}		% выравнивание по центру

		\large Санкт-Петербургский политехнический университет Петра Великого\\
		\large Институт компьютерных наук и кибербезопасности\\
		\large Высшая школа программной инженерии \\[6cm]
		% название института, затем отступ 6см

    \huge Курсовая Работа\\[0.5cm] % название работы, затем отступ 0,5см
		\large Разработка программного обеспечения для воспроизведения на ЭВМ двумерных стационарных моделей \\[0.1cm]
		\large по дисциплине \\
		\large <<Разработка программного обеспечения для моделирования физических процессов>> \\[5cm]

	\end{center}

		\noindent\large Выполнил: \hfill \large Клюкин С. А.\\
		\noindent\large Группа: \hfill \large гр. 5130904/10102\\

		\noindent\large Проверил: \hfill \large Воскобойников С. П.

	\vfill % заполнить всё доступное ниже пространство

	\begin{center}
	\large Санкт-Петербург\\
	\large \the\year % вывести дату
	\end{center} % закончить выравнивание по центру

\end{titlepage} % конец титульной страницы

\vfill % заполнить всё доступное ниже пространство

  \tableofcontents
  \newpage
  \section{Вступление}
  \subsection{Постановка задачи}

  \textbf{Вариант E5.} Используя интегро-интерполяционный метод, разработать подпрограмму для моделирования распределения температуры в полом цилиндре, описываемого математической моделью

  \begin{equation}\label{eq1}
    - \left [ \frac{1}{r} \pdv{}{r} \left ( r k(r) \pdv{u}{r} \right ) 
    + \pdv[2]{u}{z} \right ] = f(r, z), \quad 0 < c_1 \le k(r) \le c_2
  \end{equation}

  С граничными условиями:
  \begin{align*}
    & 0 < R_0 \le r \leq R_1 &
    & 0 \leq z \leq L \\
    & \left . k(r) \pdv{u}{r} \right \vert_{r=R_0} = \chi_1 \left . u \right \vert_{r=R_0} - \varphi_1(z) &
    & \left . u \right \vert_{r=R_1} = \varphi_2(z) \\
    & \left . u \right \vert_{z=0} = \varphi_3(r) &
    & \left . u \right \vert_{z=L} = \varphi_4(r) \\
    & \chi_1 > 0 &
  \end{align*}
  
  Для решения алгебраической системы использовать метод нечётно-чётного исключения (полной редукции)

  \newpage
  \section{Разностная схема}
Введем основную сетку:
\begin{align*}
  &N_r - \text{число разбиений на } [R_0, R_1] & &N_z - \text{число разбиений на } [0, L] \\
  &r_0 < r_1 < \cdots < r_N & &z_0 < z_1 < \cdots < z_N \\
  &r_0 = R_0,\quad r_N = R_1 & &z_0 = 0,\quad z_N = L \\
  &h_r = \frac{R_1 - R_0}{N_r} & &h_z = \frac{L - 0}{N_z}
\end{align*}

Введем дополнительную сетку:
\begin{align*}
  &r_{i-\frac{1}{2}} = \frac{r_i + r_{i - 1}}{2}\quad i=1,\cdots, N_r & &z_{j-\frac{1}{2}} = \frac{z_j + z_{j - 1}}{2}\quad j=1,\cdots, N_z \\
  & \hbar_i = \begin{cases}
    \frac{h_r}{2},\ i = 0 \\ \\
    h_r,\ i = 1, 2, \dots, N_r-1 \\ \\
    \frac{h_r}{2},\ i = N_r
  \end{cases} &
  & \hbar_j = \begin{cases}
    \frac{h_z}{2},\ j = 0 \\ \\
    h_z,\ j = 1, 2, \dots, N_z-1 \\ \\
    \frac{h_z}{2},\ j = N_z
  \end{cases}
\end{align*}

Преобразуем наше начальное уравнение домножив на r

\[
  - \left [ \pdv{}{r} \left ( r k(r) \pdv{u}{r} \right ) 
  + \pdv[2]{ru}{z} \right ] = rf(r, z)
\]

\subsection{Внутренние точки}
% Проинтегрируем уравнение (\hyperref[eq1]{1}) в области
$ [r_{i -\frac{1}{2}}, r_{i +\frac{1}{2}}] \times  [z_{j -\frac{1}{2}}, z_{j +\frac{1}{2}}] $
для $ i = 1, 2, \dots, N_r - 1 $ и $ j = 1, 2, \dots, N_z - 1$.

\[
  - \mLim{+\frac{1}{2}}{-\frac{1}{2}}{+\frac{1}{2}}{-\frac{1}{2}} \left [ \pdv{}{r} \left ( r k(r) \pdv{u}{r} \right ) 
  + \pdv[2]{ru}{z} \right ] dr dz = \mLim{+\frac{1}{2}}{-\frac{1}{2}}{+\frac{1}{2}}{-\frac{1}{2}} rf(r, z) dr dz
\]

Получим:

\begin{align*}
  &- \left [
   \mLimZ{z}{+\frac{1}{2}}{-\frac{1}{2}}{j}  \left . r k(r) \pdv{u}{r} \right \vert_{r = r_{i + \frac{1}{2}}} dz
  - \mLimZ{z}{+\frac{1}{2}}{-\frac{1}{2}}{j} \left . r k(r) \pdv{u}{r} \right \vert_{r = r_{i - \frac{1}{2}}} dz
  \right . \\
  &\left . + \mLimS{r}{+\frac{1}{2}}{-\frac{1}{2}} \left . r \pdv{u}{z} \right \vert_{z = z_{j + \frac{1}{2}}} dr
  - \mLimS{r}{+\frac{1}{2}}{-\frac{1}{2}} \left . r \pdv{u}{z} \right \vert_{z = z_{j - \frac{1}{2}}} dr
  \right ] = \mLim{+\frac{1}{2}}{-\frac{1}{2}}{+\frac{1}{2}}{-\frac{1}{2}} rf(r, z) dr dz
\end{align*}

Воспользуемся формулами численного дифференцирования:
\[
  \left . k(r) \pdv{u}{r} \right \vert_{r = r_{i - \frac{1}{2}}}
  \approx k(r_{i - \frac{1}{2}}) \frac{v_{i, j} - v_{i - 1, j}}{h_r}
\]

\[
  \left . \pdv{u}{r} \right \vert_{z = z_{j - \frac{1}{2}}}
  \approx \frac{v_{i, j} - v_{i, j - 1}}{h_z}
\]

Также воспользуемся формулой средних прямоугольников:
\[
  \mLimS{r}{+\frac{1}{2}}{-\frac{1}{2}} r \varphi(r, z) dr
  = \hbar_i r_i \varphi_i
\]
\[
  \mLim{+\frac{1}{2}}{-\frac{1}{2}}{+\frac{1}{2}}{-\frac{1}{2}} r \varphi(r, z) drdz
  = \hbar_i\hbar_j r_i \varphi_{i, j}
\]

В итоге получим уравнение разностной схемы:
\begin{align*}
  &- \left [ 
  \hbar_j r_{i+\frac{1}{2}} k(r_{i+\frac{1}{2}}) \frac{v_{i+1, j} - v_{i, j}}{h_{r}}
  - \hbar_j r_{i-\frac{1}{2}} k(r_{i-\frac{1}{2}}) \frac{v_{i, j} - v_{i - 1, j}}{h_{r}}
  \right . \\
  &\left .
  + \hbar_i r_{i} \frac{v_{i, j + 1} - v_{i, j}}{h_{z}}
  - \hbar_i r_{i} \frac{v_{i, j} - v_{i, j - 1}}{h_z}
  \right ]  = \hbar_i \hbar_j r_i f_{i, j}
\end{align*}

Так как выбранная основная сетка является равномерной, то $ \hbar_i = h_r $ и $ \hbar_j = h_z$
, для $ i = 1, 2, \dots, N_r - 1 $ и $ j = 1, 2, \dots, N_z - 1$.

\begin{align*}
  &- \left [ 
  h_z r_{i+\frac{1}{2}} k(r_{i+\frac{1}{2}}) \frac{v_{i+1, j} - v_{i, j}}{h_{r}}
  - h_z r_{i-\frac{1}{2}} k(r_{i-\frac{1}{2}}) \frac{v_{i, j} - v_{i - 1, j}}{h_{r}}
  \right . \\
  &\left .
  + h_r r_{i} \frac{v_{i, j + 1} - v_{i, j}}{h_{z}}
  - h_r r_{i} \frac{v_{i, j} - v_{i, j - 1}}{h_z}
  \right ]  = h_r h_z r_i f_{i, j}
\end{align*}

Умножим на $\frac{h_z}{h_r r_i}$, чтобы получилась подходящая для применяемого метода решения СЛАУ форма.
\begin{align*}
  - \left [ 
  h_z^2 \frac{r_{i+\frac{1}{2}}}{r_i} k(r_{i+\frac{1}{2}}) \frac{v_{i+1, j} - v_{i, j}}{h_{r}^2}
  - h_z^2 \frac{r_{i-\frac{1}{2}}}{r_i} k(r_{i-\frac{1}{2}}) \frac{v_{i, j} - v_{i - 1, j}}{h_{r}^2}
  + v_{i, j + 1} - 2 v_{i, j} + v_{i, j - 1}
  \right ]  = h_z^2 f_{i, j}
\end{align*}

\subsection{На левой границе}

$ [r_{i}, r_{i +\frac{1}{2}}] \times  [z_{j -\frac{1}{2}}, z_{j +\frac{1}{2}}] $
для $ i = 0 $ и $ j = 1, 2, \dots, N_z - 1$.

Получаем:
\begin{align*}
  &- \left [
   \mLimZ{z}{+\frac{1}{2}}{-\frac{1}{2}}{j}  \left . r k(r) \pdv{u}{r} \right \vert_{r = r_{i + \frac{1}{2}}} dz
  - \mLimZ{z}{+\frac{1}{2}}{-\frac{1}{2}}{j} \left . r k(r) \pdv{u}{r} \right \vert_{r = r_{i}} dz
  \right . \\
  &\left . + \mLimS{r}{+\frac{1}{2}}{} \left . r\pdv{u}{z} \right \vert_{z = z_{j + \frac{1}{2}}} dr
  - \mLimS{r}{+\frac{1}{2}}{} \left . r\pdv{u}{z} \right \vert_{z = z_{j - \frac{1}{2}}} dr
  \right ] = \mLim{+\frac{1}{2}}{}{+\frac{1}{2}}{-\frac{1}{2}} rf(r, z) dr dz
\end{align*}

Имеем граничное условие:
\[ \left . k(r) \pdv{u}{r} \right \vert_{r=R_0} = \chi_1 \left . u \right \vert_{r=R_0} - \varphi_1(z) \]

Получаем уравнение разностной схемы:

\begin{align*}
  &- \left [ 
  h_z r_{i+\frac{1}{2}} k(r_{i+\frac{1}{2}}) \frac{v_{i+1, j} - v_{i, j}}{h_{r}}
  - h_z r_{i} (\chi_1(z_j) v_{i, j} - \varphi_1(z_j))
  \right . \\
  &\left .
  + \frac{h_r}{2} r_{i} \frac{v_{i, j + 1} - v_{i, j}}{h_{z}}
  - \frac{h_r}{2} r_{i} \frac{v_{i, j} - v_{i, j - 1}}{h_z}
  \right ]  = \frac{h_r}{2} h_z r_i f_{i, j}
\end{align*}

\begin{align*}
  - \left [ 
  2 h_z^2 \frac{r_{i+\frac{1}{2}}}{r_i} k(r_{i+\frac{1}{2}}) \frac{v_{i+1, j} - v_{i, j}}{h_{r}^2}
  - 2 h_z^2 \frac{\chi_1(z_j) v_{i, j} - \varphi_1(z_j)}{h_r}
  + v_{i, j + 1} - 2 v_{i, j} + v_{i, j - 1}
  \right ]  = h_z^2 f_{i, j}
\end{align*}

\subsection{На правой границе}
$ [r_{i -\frac{1}{2}}, r_{i}] \times  [z_{j -\frac{1}{2}}, z_{j +\frac{1}{2}}] $
для $ i = N_r $ и $ j = 1, 2, \dots, N_z - 1$.

Имеем граничное условие:
\[ \left . u \right \vert_{r=R_1} = \varphi_2(z) \]

Будем его сразу использовать:
\[ v_{N_r,j} = \varphi_2(z_j) \]

\subsection{На нижней границе}
$ [r_{i  -\frac{1}{2}}, r_{i +\frac{1}{2}}] \times  [z_{j}, z_{j +\frac{1}{2}}] $
для $ i = 1, 2, \dots, N_r - 1 $ и $ j = 0$.

\[ \left . u \right \vert_{z=0} = \varphi_3(r) \]
\[ v_{i,0} = \varphi_3(r_i) \]

\subsection{На верхней границе}
$ i = 1, 2, \dots, N_r - 1 $ и $ j = N_z $
\[ \left . u \right \vert_{z=L} = \varphi_4(r) \]
\[ v_{i,N_z} = \varphi_4(r_i) \]

\subsection{Левый-нижний угол}

$ [r_{i}, r_{i +\frac{1}{2}}] \times  [z_{j}, z_{j +\frac{1}{2}}] $
для $ i = 0 $ и $ j = 0$.
\[ v_{0,0} = \varphi_3(r_0) \]

\subsection{Левый-верхний угол}

$ i = 0 $ и $ j = N_z $

\[ v_{0,N_z} = \varphi_4(r_0) \]

\subsection{Правый-верхний угол}

$ i = N_r $ и $ j = N_z $, возьмём известное граничное условие:

\[ v_{N_r,N_z} = \varphi_4(r_{N_r}) \]

\subsection{Правый-нижний угол}

$ i = N_r $ и $ j = 0 $.

\[ v_{N_r,0} = \varphi_2(z_0) \]

\subsection{На нижней границе}
$ [r_{i  -\frac{1}{2}}, r_{i +\frac{1}{2}}] \times  [z_{j}, z_{j +\frac{1}{2}}] $
для $ i = 1, 2, \dots, N_r - 1 $ и $ j = 0$.

\[ \left . u \right \vert_{z=0} = \varphi_3(r) \]
\[ v_{i,0} = \varphi_3(r_i) \]

\subsection{На верхней границе}
$ i = 1, 2, \dots, N_r - 1 $ и $ j = N_z $
\[ \left . u \right \vert_{z=L} = \varphi_4(r) \]
\[ v_{i,N_z} = \varphi_4(r_i) \]


  \newpage
  \section{Невязка разностной схемы}

\subsection{Невязка во внутренних точках}

Будем искать невязку в следующем виде:
\[
  \tilde{\xi}_{i, j} = \frac{\xi_{i, j}}{h_r h_z}
\]

\begin{align*}
  &\tilde{\xi}_{i, j} = 
  h^2_r\left[ \frac{1}{12} r k \frac{\partial^4 u}{\partial r^4} + \frac{1}{6}\frac{\partial r k}{\partial r}\frac{\partial^3 u}{\partial r^3}
  + \frac{1}{8} \frac{\partial^2 r k}{\partial r^2}\frac{\partial^2 u}{\partial r^2}
  + \frac{1}{24} \frac{\partial^3 r k}{\partial r^3}\frac{\partial u}{\partial r} \right]_{i,j} + \mathcal{O}(h_r^3) + \\
  &+ h^2_z \left[
    \frac{1}{12} r \frac{\partial^4 u}{\partial z^4}
     \right]_{i, j} + \mathcal{O}(h_z^3)
\end{align*}
Порядок аппроксимации $p_r = 2 - 0 = 2 $, $ p_z = 2 - 0 = 2 $.

\subsection{Невязка на левой границе}

Будем искать невязку в следующем виде:
\[ \Tilde{\xi}_{0, j} = \frac{\xi_{0, j}}{h_z} \]
Получаем:
\begin{align*}
  &\Tilde{\xi}_{i, j} = 
  h^2_r \left[
  \frac{1}{6} rk \frac{\partial^3 u}{ \partial r^3} +
  \frac{1}{4} \frac{\partial rk}{ \partial r} \frac{\partial^2 u}{ \partial r^2} +
  \frac{1}{8} \frac{\partial^2 rk}{ \partial r^2} \frac{\partial u}{ \partial r}
  \right]_{i ,j} + \mathcal{O}(h^3_r) + \\
 & + h_r\left[ r h^2_z\left( 
  \frac{1}{24} \frac{\partial^4 u}{ \partial z^4}
  \right)_{i, j}
  + \mathcal{O}(h^3_z) \right]
\end{align*}

Порядок аппроксимации $ p_r = 2 - 0 = 2, p_z = 2 - 0 = 2 $.

По определению невязки на остальных границах она равна нулю.

Можно сделать вывод, что разностная схема в целом имеет второй порядок аппроксимации.


  \newpage
  \section{Запись СЛАУ}

Перейдём к одноиндексной записи
\[ m = j(N_r + 1) + i + 1 \]

Индексы изменяются в следующих границах:
\[ 0 \leq i \leq N_r \]
\[ 0 \leq j \leq N_z \]

Тогда
\[ 1 \le m \le (N_r + 1)(N_z + 1) \]
, и все уравнения можно записать в следующем виде
\[ a_m w_{m - L} + b_m w_{m - 1} + c_m w_m + d_m w_{m + 1} + e_m w_{m + L} = g_m \]

\subsection{Запись для внутренних точек}

\[ a_m = e_m = -1 \]

\[ b_m = -\frac{h_z^2}{h_r^2} \frac{r_{i-\frac{1}{2}}}{r_i} k(r_{i-\frac{1}{2}}) \]
\[ c_m = 2 + \frac{h_z^2}{h_r^2 r_i} \left(r_{i-\frac{1}{2}} k(r_{i-\frac{1}{2}}) + r_{i+\frac{1}{2}} k(r_{i+\frac{1}{2}}) \right) \]
\[ d_m = -\frac{h_z^2}{h_r^2} \frac{r_{i+\frac{1}{2}}}{r_i} k(r_{i+\frac{1}{2}}) \]
\[ g_m = h_z^2 f_{i, j} \]

\subsection{Запись для левой границы}

для $ i = 0 $ и $ j = 1, 2, \dots, N_z - 1$.
\[ m = j(N_r + 1) + 1 \]

\[ a_m = e_m = -1 \]
\[ b_m = 0 \]

\[ c_m = 2 + 2 \frac{h_z^2}{h_r} \left(\chi_1(z_j) + \frac{r_{i+\frac{1}{2}}}{r_i h_r} k(r_{i+\frac{1}{2}}) \right) \]
\[ d_m = -2 \frac{h_z^2}{h_r^2} \frac{r_{i+\frac{1}{2}}}{r_i} k(r_{i+\frac{1}{2}}) \]
\[ g_m = h_z^2 \left(f_{i, j} + 2 \frac{\varphi_1(z_j)}{h_r} \right) \]

Для остальных точек

\[ a,b,d,e = 0 \]

\subsection{Структура матрицы для Nx=Ny=4}

Обозначим нулевые элементы пустыми клетками. Обозначим элементы, нарушающие симметричность крестиками.

\begin{center}
\begin{tabular}{ c c c c | c c c c | c c c c | c c c c }
01 & 02 & 03 & 04 & 05 & 06 & 07 & 08 & 09 & 10 & 11 & 12 & 13 & 14 & 15 & 16 \\
\hline 
1 & & & & & & & & & & & & & & & \\
 & 1 & & & & & & & & & & & & & & \\
 & & 1 & & & & & & & & & & & & & \\
 & & & 1 & & & & & & & & & & & & \\
\hline
 x & & & & * & * & & & * & & & & & & & \\
 & x & & & * & * & * & & & * & & & & & & \\
 & & x & & & * & * & x & & & * & & & & & \\
 & & & & & & & 1 & & & & & & & & \\
\hline 
 & & & & * & & & & * & * & & & x & & & \\
 & & & & & * & & & * & * & * & & & x & & \\
 & & & & & & * & & & * & * & x & & & x & \\
 & & & & & & & & & & & 1 & & & & \\
\hline
 & & & & & & & & & & & & 1 & & & \\
 & & & & & & & & & & & & & 1 & & \\
 & & & & & & & & & & & & & & 1 & \\
 & & & & & & & & & & & & & & & 1
\end{tabular}
\end{center}

В итоге мы получаем СЛАУ с нессиметричной матрицей.
% Чтобы убрать крестик из матрицы, можно воспользоваться умножением на скаляр и сложением строк.
Для работы используемого метода решения СЛАУ надо достроить боковые блоки до -E. Для этого из каждой строки в которой только одна единица и не в крайних блоках вычтем строки где в этой колонке стоит единица. Изменится вектор g и не изменится блок C.


  \newpage
  \section{Метод нечётно-чётного исключения}

Коэффициенты $b_m, c_m, d_m$ не зависят от индекса j. Значит для каждой линии трёхдиагональный блок $C$ будет один и тот же:
\[ \begin{cases}
V_j = F_j, \quad j = 0 \\
-V_{j-1} + C V_j -V_{j+1} = F_j, \quad j = 1,2,\dots,N_z-1 \\
V_j = F_j, \quad j = N_z
\end{cases} \]

Обозначения
\[ N = N_z \]
\[ n = log_2(N) \in Z \]
\[ C^{(0)} = C, \quad F_j^{(0)} = F_j, \quad j=1,2,\dots,N-1 \]

Прямой ход
\[ C^{(k)} = (C^{(k-1)})^2 - 2E, \]
\[ F^{(k)}_j = F^{(k-1)}_{j-2^{k-1}} + C^{(k-1)} F^{(k-1)}_j + F^{(k-1)}_{j+2^{k-1}}, \]
\[ j = 2^k,2\cdot2^k,3\cdot2^k,\dots,N-2^k,\quad k = 1,2,\dots \]

Обратный
\[ C^{(k-1)} V_j = F^{(k-1)}_j + V_{j-2^{k-1}} + V_{j+2^{k-1}}, \]
\[ V_0 = F_0, \quad V_N = F_N, \]
\[ j = 2^{k-1},3\cdot2^{k-1},5\cdot2^{k-1},\dots,N-2^k,\quad k = n,n-1,\dots,1 \]

Введем векторы $p^{(k)}_j$ такие что
\[ F^{(k)}_j = \prod^{k-1}_{l=0} C^{(l)} p^{(k)}_j 2^k \]
\[ \prod^{-1}_{l=0} C^{(l)} = E, \quad p^{(0)}_j = F^{(0)}_j \]

Из прямого хода получаем
\[ 2 C^{(k-1)} p^{(k)}_j = p^{(k-1)}_{j-2^{k-1}} + C^{(k-1)} p^{(k-1)}_j + p^{(k-1)}_{j+2^{k-1}}, \]
Обозначим
\[ S^{(k-1)}_j = 2 p^{(k)}_j - p^{(k-1)}_j \]

Формулы для p
\[ C^{(k-1)} S^{(k-1)}_j = p^{(k-1)}_{j-2^{k-1}} + p^{(k-1)}_{j+2^{k-1}},\quad p^{(k)}_j = 0.5(p^{(k-1)}_j + S^{(k-1)}_j), \quad p^{(0)}_j = F^{(0)}_j, \]
\[ j = 2^k,2\cdot2^k,3\cdot2^k,\dots,N-2^k,\quad k = 1,2,\dots \]

Из обратного хода
\[ C^{(k-1)} V_j = 2^{k-1} \prod^{k-2}_{l=0} C^{(l)} p^{(k-1)}_j  + V_{j-2^{k-1}} + V_{j+2^{k-1}} \]

Обозначения

% $T_n(x)$ - полином Чебышева n-й степени первого рода

% $U_n(x)$ - полином Чебышева n-й степени второго рода

\[ C_{l, k-1} = C - 2 cos \frac{(2l-1) \pi }{2^k} E \]
\[ \alpha_{l, k-1} = \frac{(-1)^{l+1}}{2^{k-1}} sin \frac{(2l-1) \pi }{2^k} \]

Из теорем о полиномах Чебышева первого и второго рода
\[ (C^{(k-1)})^{-1} = \sum^{2^{k-1}}_{l=1} \alpha_{l, k-1} C_{l, k-1}^{-1} \]
\[ (C^{(k-1)})^{-1} \prod^{k-2 (!)}_{l=0} C^{(l)} = \frac{1}{2^{k-1}} \sum^{2^{k-1}}_{l=1} C_{l, k-1}^{-1} \]

Тогда формулы можно записать так
\[ S^{(k-1)}_j = \sum^{2^{k-1}}_{l=1} \alpha_{l, k-1} C_{l, k-1}^{-1} (p^{(k-1)}_{j-2^{k-1}} + p^{(k-1)}_{j+2^{k-1}}) \]
\[ V_j = \sum^{2^{k-1}}_{l=1} C_{l, k-1}^{-1} (p^{(k-1)}_j + \alpha_{l, k-1} (V_{j-2^{k-1}} + V_{j+2^{k-1}})) \]

\RestyleAlgo{tworuled}
\SetAlFnt{\normalsize}
\SetAlgoNoLine
\SetKw{KwBy}{by}

\begin{algorithm*}[H]
  \DontPrintSemicolon
  $ p^{(0)}_j = F_j, \quad j=1,2,\dots,N-1 $ \;
  \For{$k = 1$ \KwTo $n-1$}{
    \For{$j = 2^k$ \KwTo $N-2^k$ \KwBy $2^k$}{
      $\varphi = p^{(k-1)}_{j-2^{k-1}} + p^{(k-1)}_{j+2^{k-1}}$ \;
      $C_{l, k-1} v_l = \alpha_{l, k-1} \varphi, \quad l = 1,2,\dots,2^{k-1}$ \;
      $p^{(k)}_j = 0.5(p^{(k-1)}_j + v_1 + v_2 + \dots + v_{2^{k-1}})$ \;
    }
  }
  $V_0 = F_0, \quad V_N = F_N$ \;
  \For{$k = n$ \KwTo $1$}{
    \For{$j = 2^{k-1}$ \KwTo $N-2^{k-1}$ \KwBy $2\cdot 2^{k-1}$}{
      $\varphi = V_{j-2^{k-1}} + V_{j+2^{k-1}}, \quad \psi = p^{(k-1)}_j$ \;
      $C_{l, k-1} v_l = \psi + \alpha_{l, k-1} \varphi, \quad l = 1,2,\dots,2^{k-1}$ \;
      $V_j = v_1 + v_2 + \dots + v_{2^{k-1}}$ \;
    }
  }
\end{algorithm*}

Исходя из полученного метода, для хранения матрицы $A$ достаточно хранить один трёхдиагональный блок $C$, и хранить его надо в такой форме, которой достаточно, чтобы решать СЛАУ с ним. Будем хранить три вектора диагоналей.


  \newpage
  \section{Тестирование}

Судя по первой невязке погрешность появлется когда суммарная степень k и u относительно r становится 3. По второй - когда 2. Об общей погрешности будем судить по второй невязке $\varepsilon_2$.

Протестируем постренную модель на следующих тестовых наборах:
\begin{center}
  \begin{tabular}{c|c c|c c c c c|c|c c c|c}
  \toprule
  № теста & $k(r)$ & $u(r, z)$ & $\frac{\partial^4 u}{\partial r^4}$ & $\frac{\partial r k}{\partial r}\frac{\partial^3 u}{\partial r^3}$ & $\frac{\partial^2 r k}{\partial r^2}\frac{\partial^2 u}{\partial r^2}$ & $\frac{\partial^3 r k}{\partial r^3}\frac{\partial u}{\partial r}$ & $\frac{\partial^4 u}{\partial z^4}$ & $\varepsilon_1$ & $\frac{\partial^3 u}{ \partial r^3}$ & $\frac{\partial rk}{ \partial r} \frac{\partial^2 u}{ \partial r^2}$ & $\frac{\partial^2 rk}{ \partial r^2} \frac{\partial u}{ \partial r}$ & $\varepsilon_2$ \\
  \midrule
  1 & $ 1 $ & $r + z^3$ & 0 & 0 & 0 & 0 & 0 & 0 & 0 & 0 & 0 & 0 \\
  2 & $ r + 2z + 1 $ & $r + z^3$ & 0 & 0 & 0 & 0 & 0 & 0 & 0 & 0 & 1 & 1 \\
  3 & $ (r + 2z + 1)^2 $ & $r + z^3$ & 0 & 0 & 0 & 1 & 0 & 1 & 0 & 0 & 1 & 1 \\
  4 & $ sin(rz) + 2 $ & $log(r) + \sqrt{z}$ & 1 & 1 & 1 & 1 & 1 & 1 & 1 & 1 & 1 & 1 \\
  \bottomrule
  \end{tabular}
\end{center}

Вид функции $f$ можно вывести из вида функций $k,u$.

Вид функций $\varphi_i, i\neq1$ определяется видом функции $u$. Функция $\chi_1$ для работы метода решения СЛАУ должна не зависеть от $z$.
\[\chi_1(z) = 2\]
Можно вывести вид функции $\varphi_1$.

\begin{center}
  \begin{tabular}{c|c c|c c}
  \toprule
  № теста & $k(r)$ & $u(r, z)$ & $f(r, z)$ & $\varphi_1(z)$ \\
  \midrule
  1 & $ 1 $ & $r + z^3$ & $- \frac{1}{r} - 6 z$ & $- 1 + \left(R_{0} + z^{3}\right) \left(2\right)$ \\
  2 & $ r + 2z + 1 $ & $r + z^3$ & $- 6 z - 2 - \frac{2 z}{r} - \frac{1}{r}$ & $- R_{0} - 2 z + \left(R_{0} + z^{3}\right) \left(2\right) - 1$ \\
  3 & $ (r + 2z + 1)^2 $ & $r + z^3$ & $- 2 r - 10 z - 2 - \frac{\left(r + 2\right)^{2}}{r}$ & $\left(R_{0} + z^{3}\right) \left(2\right) - \left(R_{0} + 2 z + 1\right)^{2}$ \\
  4 & $ sin(rz) + 2 $ & $log(r) + \sqrt{z}$ & $\frac{0.25}{z^{1.5}} - \frac{z^{1.0} \cos{\left(r z \right)}}{r}$ & $\frac{R_{0} \left(z^{0.5} + \log{\left(R_{0} \right)}\right) \left(2\right) - \sin{\left(R_{0} z \right)} - 2}{R_{0}}$ \\
  \bottomrule
  \end{tabular}
\end{center}

Погрешность решения задачи вычисляется следующим образом:
\[ \delta_1 = \frac{\left. \|v - \Tilde{v}\| \right._1}{\left. \|v\| \right._1} \]
\[ \delta_2 = \frac{\left. \|v - \Tilde{v}\| \right._2}{\left. \|v\| \right._2} \]
\[ \delta_3 = \frac{\left. \|v - \Tilde{v}\| \right._\infty}{\left. \|v\| \right._\infty} \]

Параметры для всех тестов
\[ r \in [1, 3], \quad z \in [0, 2] \]

$N_z$ требуется брать как степень двойки.

Система алгебраических уравнений с известным решением для тестирования разработанной подпрограммы
\[ A = \begin{pmatrix}
  1 &   &   &   &   &   &   &   &   &   &   &   &   &   &   &  \\
    & 1 &   &   &   &   &   &   &   &   &   &   &   &   &   &  \\
    &   & 1 &   &   &   &   &   &   &   &   &   &   &   &   &  \\
    &   &   & 1 &   &   &   &   &   &   &   &   &   &   &   &  \\
  -1 &   &   &   & 5 & 2 &   &   & -1 &   &   &   &   &   &   &  \\
    & -1 &   &   & 2 & 5 & 2 &   &   & -1 &   &   &   &   &   &  \\
    &   & -1 &   &   & 2 & 5 &   &   &   & -1 &   &   &   &   &  \\
    &   &   &   &   &   &   & 1 &   &   &   &   &   &   &   &  \\
    &   &   &   & -1 &   &   &   & 5 & 2 &   &   & -1 &   &   &  \\
    &   &   &   &   & -1 &   &   & 2 & 5 & 2 &   &   & -1 &   &  \\
    &   &   &   &   &   & -1 &   &   & 2 & 5 &   &   &   & -1 &  \\
    &   &   &   &   &   &   &   &   &   &   & 1 &   &   &   &  \\
    &   &   &   &   &   &   &   &   &   &   &   & 1 &   &   &  \\
    &   &   &   &   &   &   &   &   &   &   &   &   & 1 &   &  \\
    &   &   &   &   &   &   &   &   &   &   &   &   &   & 1 &  \\
    &   &   &   &   &   &   &   &   &   &   &   &   &   &   & 1\\
\end{pmatrix} \]
\[ b = \begin{pmatrix}
   .55 &  .72 &  .6 &  .54 & 1.89 & 3.87 & 2.11 &  .89 & 4.57 & 3.82 & 4.2 &  .53 &  .57 &  .93 &
   . 7 &  . 9
\end{pmatrix}^{T} \]
\[ x = \begin{pmatrix}
   .55 &  .72 &  .6 &  .54 &  .42 &  .65 &  .44 &  .89 &  .96 &  .38 &  .79 &  .53 &  .57 &  .93 &
   . 7 &  . 9
\end{pmatrix}^{T} \]


  % \newpage
  % \section{Заключение}

  % В рамках данной курсовой работы мы использовали свои практические знания для
  % написания программного обеспечения, которое моделирует стационарные процессы
  % теплопроводности. Мы построили дискретную модель и протестировали её.

  % Была разработана программа моделирования стационарных процессов теплопроводности.
  % Привидены результаты экспериментального исследования точности программного обеспечения.

  % \newpage
  % \section*{Код программы}
\addcontentsline{toc}{section}{Код программы}

\inputminted[label=grid.py]{python}{lst/grid.py}
\inputminted[label=linalg.py]{python}{lst/linalg.py}
\inputminted[label=main.py]{python}{lst/main2.py}
\inputminted[label=model.py]{python}{lst/model.py}
\inputminted[label=sparse\_matrix.py]{python}{lst/sparse_matrix.py}
\inputminted[label=utils.py]{python}{lst/utils.py}


\end{document}
