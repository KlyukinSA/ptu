\documentclass[a4paper,12pt]{article}

\usepackage[T2A]{fontenc} % ru lang
% \usepackage[utf8]{inputenc}
\usepackage[english,russian]{babel} % ru lang

% \usepackage{fontspec}
% \setmainfont[Scale=1.167]{Times New Roman} % ??????

% \usepackage{unicode-math}

% \usepackage{microtype}
\usepackage[linesnumbered,ruled,vlined]{algorithm2e}

% \usepackage{amsmath, amsfonts, amsthm}
\usepackage{amsmath} % align*
% \usepackage{listings}
% \usepackage{enumerate}
\usepackage{float} % tables not float
% \usepackage{graphicx}
% \usepackage{nameref}
% \usepackage{hyperref}
% \usepackage{tabularx}
% \usepackage{makecell}
\usepackage{indentfirst} % ru style
\usepackage{booktabs} % table horisontal lines
\usepackage{derivative} % partial \pdv
% \usepackage{physics}

% \usepackage{tikz}
% \usepackage{tikzscale}
% \usetikzlibrary{decorations.pathreplacing,calligraphy}
% \usetikzlibrary{arrows.meta}
% \usetikzlibrary{backgrounds,fit,positioning}
% \usetikzlibrary{external}

\usepackage[newfloat]{minted} % listings

% \setminted{
%   frame=lines,
%   framesep=2mm,
%   baselinestretch=1.2,
%   fontsize=\footnotesize,
%   linenos,
%   breaklines
% }

\usepackage[top=3cm,bottom=3cm,left=1cm,right=1cm]{geometry} % large formulas not overful hbox
% \hypersetup{
% 	colorlinks=true,
% 	linkcolor=blue,
% 	filecolor=magenta,
% 	urlcolor=cyan
% }
% \urlstyle{same}
\setcounter{MaxMatrixCols}{16} % pmatrix



\begin{document}
\newcommand\mLim[4]{
  \int\limits^{r_{i #1}}_{r_{i #2}}
  \int\limits^{z_{j #3}}_{z_{j #4}}
}

\newcommand\Int[2]{
  \int\limits^{#1}_{#2}
}

\newcommand\mLimS[3]
{
  \int\limits^{#1_{i#2}}_{#1_{i#3}}
}

\newcommand\mLimZ[4]
{
  \int\limits^{#1_{#4 #2}}_{#1_{#4 #3}}
}

\begin{titlepage}	% начало титульной страницы

	\begin{center}		% выравнивание по центру

		\large Санкт-Петербургский политехнический университет Петра Великого\\
		\large Институт компьютерных наук и кибербезопасности\\
		\large Высшая школа программной инженерии \\[6cm]
		% название института, затем отступ 6см

    \huge Курсовая Работа\\[0.5cm] % название работы, затем отступ 0,5см
		\large Разработка программного обеспечения для воспроизведения на ЭВМ двумерных стационарных моделей \\[0.1cm]
		\large по дисциплине \\
		\large <<Разработка программного обеспечения для моделирования физических процессов>> \\[5cm]

	\end{center}

		\noindent\large Выполнил: \hfill \large Клюкин С. А.\\
		\noindent\large Группа: \hfill \large гр. 5130904/10102\\

		\noindent\large Проверил: \hfill \large Воскобойников С. П.

	\vfill % заполнить всё доступное ниже пространство

	\begin{center}
	\large Санкт-Петербург\\
	\large \the\year % вывести дату
	\end{center} % закончить выравнивание по центру

\end{titlepage} % конец титульной страницы

\vfill % заполнить всё доступное ниже пространство

  \tableofcontents
  \newpage
  \section{Вступление}
  \subsection{Постановка задачи}

  \textbf{Вариант E5.} Используя интегро-интерполяционный метод, разработать подпрограмму для моделирования распределения температуры в полом цилиндре, описываемого математической моделью

  \begin{equation}\label{eq1}
    - \left [ \frac{1}{r} \pdv{}{r} \left ( r k(r) \pdv{u}{r} \right ) 
    + \pdv[2]{u}{z} \right ] = f(r, z), \quad 0 < c_1 \le k(r) \le c_2
  \end{equation}

  С граничными условиями:
  \begin{align*}
    & 0 < R_0 \le r \leq R_1 &
    & 0 \leq z \leq L \\
    & \left . k(r) \pdv{u}{r} \right \vert_{r=R_0} = \chi_1 \left . u \right \vert_{r=R_0} - \varphi_1(z) &
    & \left . u \right \vert_{r=R_1} = \varphi_2(z) \\
    & \left . u \right \vert_{z=0} = \varphi_3(r) &
    & \left . u \right \vert_{z=L} = \varphi_4(r) \\
    & \chi_1 > 0 &
  \end{align*}
  
  Для решения алгебраической системы использовать метод нечётно-чётного исключения (полной редукции)

  \newpage
  \subsection{Интегро-интеполяционный метод (метод баланса)}

Введем основную сетку, где N - число разбиений.
\[r_0 < r_1 < \dots < r_N,\ r_i \in [0, R],\ r_0 = 0,\ r_N = R\]
\[
  h_i =r_i - r_{i-1},\ i=1,2, \dots, N
\]
\[
  r_{r-0.5} = \frac{r_i - r_{i-1}}{2},\ i=1,2, \dots, N
\]
Введем дополнительную сетку:
\[
  \hbar_i = \begin{cases}
    \frac{h_{i+1}}{2},\ i = 0 \\ \\
    \frac{h_i + h_{i+1}}{2},\ i = 1, 2, \dots, N-1 \\ \\
    \frac{h_i}{2},\ i = N
  \end{cases}
\]
Проведем аппроксимацию начального уравнения.

Получаем разностную схему для i = 0:
\[
  -\left[ r_{i+0.5} \cdot k_{i+0.5}\frac{v_{i+1}-v_i}{h_{i+1}} - \frac{r_{i+0.5} \hbar_i q_i v_i}{2} \right]\ =\ \frac{\hbar_i r_{i+0.5} f_i}{2}
\]

Получаем разностную схему для i\ =\ 1, 2, \dots, N-1:
\[
  -\left[ r_{i+0.5} \cdot k_{i+0.5}\frac{v_{i+1}-v_i}{h_{i+1}} - r_{i-0.5}k_{i-0.5}\frac{v_{i}\ -\ v_{i-1}}{h_{i}} - \hbar r_i q_i v_i\right]\ =\ \hbar_ir_if_i
\]

Получаем разностную схему для i=N:
\[
  -\left[ r_i (-\chi v_i + \nu) - r_{i-0.5}k_{i-0.5} \cdot \frac{v_i-v_{i-1}}{h_i}- \hbar_ir_iq_iv_i \right]\ =\ \hbar_ir_if_i
\]

После аппроксимации уравнения можно представить в виде системы из трёхдиагональной матрицы где a, c, b - диагонали матрица A и вектора g. Элементы матрицы:\newline
Для i\ =\ 0
\begin{align*}
  c_i = \frac{\hbar_i r_{i+0.5} q_i}{2} + \frac{r_{i+0.5} k_{i+0.5}}{h_{i+1}} \quad
  b_i = -r_{i+0.5} \cdot \frac{k_{i+0.5}}{h_{i+1}} \quad
  g_i = \frac{\hbar_i r_{i+0.5} f_i}{2}
\end{align*}
Для i\ =\ 1, 2, \dots, N-1
\begin{align*}
  a_i &= -r_{i-0.5}\frac{k_{i-0.5}}{h_i} \quad
  c_i = r_{i-0.5}\frac{k_{i-0.5}}{h_i} + r_{i+0.5}\frac{k_{i+0.5}}{h_{i+1}} + \hbar_i r_iq_i \quad
  b_i = -r_{i+0.5}\frac{k_{i+0.5}}{h_{i+1}} \\
  g_i &= \hbar_i r_i f_i
\end{align*}
Для i\ =\ N:
\begin{align*}
  a_i = -r_{i-0.5}\frac{k_{i-0.5}}{h_i} \quad
  c_i = \chi r_i + r_{i-0.5}\frac{k_{i-0.5}}{h_i} + \hbar_i r_iq_i \quad
  g_i = \hbar_i r_i f_i + \nu r_i
\end{align*}


  \newpage
  \section{Невязка разностной схемы}

\subsection{Невязка во внутренних точках}

Будем искать невязку в следующем виде:
\[
  \tilde{\xi}_{i, j} = \frac{\xi_{i, j}}{h_r h_z}
\]

\begin{align*}
  &\tilde{\xi}_{i, j} = 
  h^2_r\left[ \frac{1}{12} r k \frac{\partial^4 u}{\partial r^4} + \frac{1}{6}\frac{\partial r k}{\partial r}\frac{\partial^3 u}{\partial r^3}
  + \frac{1}{8} \frac{\partial^2 r k}{\partial r^2}\frac{\partial^2 u}{\partial r^2}
  + \frac{1}{24} \frac{\partial^3 r k}{\partial r^3}\frac{\partial u}{\partial r} \right]_{i,j} + \mathcal{O}(h_r^3) + \\
  &+ h^2_z \left[
    \frac{1}{12} r \frac{\partial^4 u}{\partial z^4}
     \right]_{i, j} + \mathcal{O}(h_z^3)
\end{align*}
Порядок аппроксимации $p_r = 2 - 0 = 2 $, $ p_z = 2 - 0 = 2 $.

\subsection{Невязка на левой границе}

Будем искать невязку в следующем виде:
\[ \Tilde{\xi}_{0, j} = \frac{\xi_{0, j}}{h_z} \]
Получаем:
\begin{align*}
  &\Tilde{\xi}_{i, j} = 
  h^2_r \left[
  \frac{1}{6} rk \frac{\partial^3 u}{ \partial r^3} +
  \frac{1}{4} \frac{\partial rk}{ \partial r} \frac{\partial^2 u}{ \partial r^2} +
  \frac{1}{8} \frac{\partial^2 rk}{ \partial r^2} \frac{\partial u}{ \partial r}
  \right]_{i ,j} + \mathcal{O}(h^3_r) + \\
 & + h_r\left[ r h^2_z\left( 
  \frac{1}{24} \frac{\partial^4 u}{ \partial z^4}
  \right)_{i, j}
  + \mathcal{O}(h^3_z) \right]
\end{align*}

Порядок аппроксимации $ p_r = 2 - 0 = 2, p_z = 2 - 0 = 2 $.

По определению невязки на остальных границах она равна нулю.

Можно сделать вывод, что разностная схема в целом имеет второй порядок аппроксимации.


  \newpage
  \section{Запись СЛАУ}

Перейдём к одноиндексной записи
\[ m = j(N_r + 1) + i + 1 \]

Индексы изменяются в следующих границах:
\[ 0 \leq i \leq N_r \]
\[ 0 \leq j \leq N_z \]

Тогда
\[ 1 \le m \le (N_r + 1)(N_z + 1) \]
, и все уравнения можно записать в следующем виде
\[ a_m w_{m - L} + b_m w_{m - 1} + c_m w_m + d_m w_{m + 1} + e_m w_{m + L} = g_m \]

\subsection{Запись для внутренних точек}

\[ a_m = e_m = -1 \]

\[ b_m = -\frac{h_z^2}{h_r^2} \frac{r_{i-\frac{1}{2}}}{r_i} k(r_{i-\frac{1}{2}}) \]
\[ c_m = 2 + \frac{h_z^2}{h_r^2 r_i} \left(r_{i-\frac{1}{2}} k(r_{i-\frac{1}{2}}) + r_{i+\frac{1}{2}} k(r_{i+\frac{1}{2}}) \right) \]
\[ d_m = -\frac{h_z^2}{h_r^2} \frac{r_{i+\frac{1}{2}}}{r_i} k(r_{i+\frac{1}{2}}) \]
\[ g_m = h_z^2 f_{i, j} \]

\subsection{Запись для левой границы}

для $ i = 0 $ и $ j = 1, 2, \dots, N_z - 1$.
\[ m = j(N_r + 1) + 1 \]

\[ a_m = e_m = -1 \]
\[ b_m = 0 \]

\[ c_m = 2 + 2 \frac{h_z^2}{h_r} \left(\chi_1(z_j) + \frac{r_{i+\frac{1}{2}}}{r_i h_r} k(r_{i+\frac{1}{2}}) \right) \]
\[ d_m = -2 \frac{h_z^2}{h_r^2} \frac{r_{i+\frac{1}{2}}}{r_i} k(r_{i+\frac{1}{2}}) \]
\[ g_m = h_z^2 \left(f_{i, j} + 2 \frac{\varphi_1(z_j)}{h_r} \right) \]

Для остальных точек

\[ a,b,d,e = 0 \]

\subsection{Структура матрицы для Nx=Ny=4}

Обозначим нулевые элементы пустыми клетками. Обозначим элементы, нарушающие симметричность крестиками.

\begin{center}
\begin{tabular}{ c c c c | c c c c | c c c c | c c c c }
01 & 02 & 03 & 04 & 05 & 06 & 07 & 08 & 09 & 10 & 11 & 12 & 13 & 14 & 15 & 16 \\
\hline 
1 & & & & & & & & & & & & & & & \\
 & 1 & & & & & & & & & & & & & & \\
 & & 1 & & & & & & & & & & & & & \\
 & & & 1 & & & & & & & & & & & & \\
\hline
 x & & & & * & * & & & * & & & & & & & \\
 & x & & & * & * & * & & & * & & & & & & \\
 & & x & & & * & * & x & & & * & & & & & \\
 & & & & & & & 1 & & & & & & & & \\
\hline 
 & & & & * & & & & * & * & & & x & & & \\
 & & & & & * & & & * & * & * & & & x & & \\
 & & & & & & * & & & * & * & x & & & x & \\
 & & & & & & & & & & & 1 & & & & \\
\hline
 & & & & & & & & & & & & 1 & & & \\
 & & & & & & & & & & & & & 1 & & \\
 & & & & & & & & & & & & & & 1 & \\
 & & & & & & & & & & & & & & & 1
\end{tabular}
\end{center}

В итоге мы получаем СЛАУ с нессиметричной матрицей.
% Чтобы убрать крестик из матрицы, можно воспользоваться умножением на скаляр и сложением строк.
Для работы используемого метода решения СЛАУ надо достроить боковые блоки до -E. Для этого из каждой строки в которой только одна единица и не в крайних блоках вычтем строки где в этой колонке стоит единица. Изменится вектор g и не изменится блок C.


  \newpage
  \section{Метод нечётно-чётного исключения}

Коэффициенты $b_m, c_m, d_m$ не зависят от индекса j. Значит для каждой линии трёхдиагональный блок $C$ будет один и тот же:
\[ \begin{cases}
V_j = F_j, \quad j = 0 \\
-V_{j-1} + C V_j -V_{j+1} = F_j, \quad j = 1,2,\dots,N_z-1 \\
V_j = F_j, \quad j = N_z
\end{cases} \]

Обозначения
\[ N = N_z \]
\[ n = log_2(N) \in Z \]
\[ C^{(0)} = C, \quad F_j^{(0)} = F_j, \quad j=1,2,\dots,N-1 \]

Прямой ход
\[ C^{(k)} = (C^{(k-1)})^2 - 2E, \]
\[ F^{(k)}_j = F^{(k-1)}_{j-2^{k-1}} + C^{(k-1)} F^{(k-1)}_j + F^{(k-1)}_{j+2^{k-1}}, \]
\[ j = 2^k,2\cdot2^k,3\cdot2^k,\dots,N-2^k,\quad k = 1,2,\dots \]

Обратный
\[ C^{(k-1)} V_j = F^{(k-1)}_j + V_{j-2^{k-1}} + V_{j+2^{k-1}}, \]
\[ V_0 = F_0, \quad V_N = F_N, \]
\[ j = 2^{k-1},3\cdot2^{k-1},5\cdot2^{k-1},\dots,N-2^k,\quad k = n,n-1,\dots,1 \]

Введем векторы $p^{(k)}_j$ такие что
\[ F^{(k)}_j = \prod^{k-1}_{l=0} C^{(l)} p^{(k)}_j 2^k \]
\[ \prod^{-1}_{l=0} C^{(l)} = E, \quad p^{(0)}_j = F^{(0)}_j \]

Из прямого хода получаем
\[ 2 C^{(k-1)} p^{(k)}_j = p^{(k-1)}_{j-2^{k-1}} + C^{(k-1)} p^{(k-1)}_j + p^{(k-1)}_{j+2^{k-1}}, \]
Обозначим
\[ S^{(k-1)}_j = 2 p^{(k)}_j - p^{(k-1)}_j \]

Формулы для p
\[ C^{(k-1)} S^{(k-1)}_j = p^{(k-1)}_{j-2^{k-1}} + p^{(k-1)}_{j+2^{k-1}},\quad p^{(k)}_j = 0.5(p^{(k-1)}_j + S^{(k-1)}_j), \quad p^{(0)}_j = F^{(0)}_j, \]
\[ j = 2^k,2\cdot2^k,3\cdot2^k,\dots,N-2^k,\quad k = 1,2,\dots \]

Из обратного хода
\[ C^{(k-1)} V_j = 2^{k-1} \prod^{k-2}_{l=0} C^{(l)} p^{(k-1)}_j  + V_{j-2^{k-1}} + V_{j+2^{k-1}} \]

Обозначения

% $T_n(x)$ - полином Чебышева n-й степени первого рода

% $U_n(x)$ - полином Чебышева n-й степени второго рода

\[ C_{l, k-1} = C - 2 cos \frac{(2l-1) \pi }{2^k} E \]
\[ \alpha_{l, k-1} = \frac{(-1)^{l+1}}{2^{k-1}} sin \frac{(2l-1) \pi }{2^k} \]

Из теорем о полиномах Чебышева первого и второго рода
\[ (C^{(k-1)})^{-1} = \sum^{2^{k-1}}_{l=1} \alpha_{l, k-1} C_{l, k-1}^{-1} \]
\[ (C^{(k-1)})^{-1} \prod^{k-2 (!)}_{l=0} C^{(l)} = \frac{1}{2^{k-1}} \sum^{2^{k-1}}_{l=1} C_{l, k-1}^{-1} \]

Тогда формулы можно записать так
\[ S^{(k-1)}_j = \sum^{2^{k-1}}_{l=1} \alpha_{l, k-1} C_{l, k-1}^{-1} (p^{(k-1)}_{j-2^{k-1}} + p^{(k-1)}_{j+2^{k-1}}) \]
\[ V_j = \sum^{2^{k-1}}_{l=1} C_{l, k-1}^{-1} (p^{(k-1)}_j + \alpha_{l, k-1} (V_{j-2^{k-1}} + V_{j+2^{k-1}})) \]

\RestyleAlgo{tworuled}
\SetAlFnt{\normalsize}
\SetAlgoNoLine
\SetKw{KwBy}{by}

\begin{algorithm*}[H]
  \DontPrintSemicolon
  $ p^{(0)}_j = F_j, \quad j=1,2,\dots,N-1 $ \;
  \For{$k = 1$ \KwTo $n-1$}{
    \For{$j = 2^k$ \KwTo $N-2^k$ \KwBy $2^k$}{
      $\varphi = p^{(k-1)}_{j-2^{k-1}} + p^{(k-1)}_{j+2^{k-1}}$ \;
      $C_{l, k-1} v_l = \alpha_{l, k-1} \varphi, \quad l = 1,2,\dots,2^{k-1}$ \;
      $p^{(k)}_j = 0.5(p^{(k-1)}_j + v_1 + v_2 + \dots + v_{2^{k-1}})$ \;
    }
  }
  $V_0 = F_0, \quad V_N = F_N$ \;
  \For{$k = n$ \KwTo $1$}{
    \For{$j = 2^{k-1}$ \KwTo $N-2^{k-1}$ \KwBy $2\cdot 2^{k-1}$}{
      $\varphi = V_{j-2^{k-1}} + V_{j+2^{k-1}}, \quad \psi = p^{(k-1)}_j$ \;
      $C_{l, k-1} v_l = \psi + \alpha_{l, k-1} \varphi, \quad l = 1,2,\dots,2^{k-1}$ \;
      $V_j = v_1 + v_2 + \dots + v_{2^{k-1}}$ \;
    }
  }
\end{algorithm*}

Исходя из полученного метода, для хранения матрицы $A$ достаточно хранить один трёхдиагональный блок $C$, и хранить его надо в такой форме, которой достаточно, чтобы решать СЛАУ с ним. Будем хранить три вектора диагоналей.


  \newpage
  \section{Тестирование}

Судя по первой невязке погрешность появлется когда суммарная степень k и u относительно r становится 3. По второй - когда 2. Об общей погрешности будем судить по второй невязке $\xi_2$.

Протестируем постренную модель на следующих тестовых наборах:
% \begin{tiny}
\begin{center}
  \begin{tabular}{c|c c|c c |c| c c|c}
  \toprule
  № теста & $k(r)$ & $u(r, z)$ & $\Phi_{1,r}$ & $\Phi_{1,z}$ & $\xi_1$ & $\Phi_{2,r}$ & $\Phi_{2,z}$ & $\xi_2$ \\
  \midrule
  1 & $ 1 $ & $r + z^3$                     & 0 & 0 & 0 & 0 & 0 & 0 \\
  2 & $ r + 1 $ & $r + z^3$                 & 0 & 0 & 0 & 1 & 0 & 1 \\
  3 & $ (r + 1)^2 $ & $r + z^3$             & 1 & 0 & 1 & 1 & 0 & 1 \\
  \bottomrule
  \end{tabular}
\end{center}
% \end{tiny}
Вид функции $f$ можно вывести из вида функций $k,u$.

Вид функций $\varphi_i, i\neq1$ определяется видом функции $u$. Функция $\chi_1$ для работы метода решения СЛАУ должна не зависеть от $z$.
\[\chi_1(z) = 2\]
Можно вывести вид функции $\varphi_1$.

\begin{center}
  \begin{tabular}{c|c c|c c}
  \toprule
  № теста & $k(r)$ & $u(r, z)$ & $f(r, z)$ & $\varphi_1(z)$ \\
  \midrule
1&$1.0$&$r + z^{3}$&$- 6 z - \frac{1.0}{r}$&$2 R_{0} + 2 z^{3} - 1.0$\\
2&$r + 1$&$r + z^{3}$&$- 6 z - 2 - \frac{1}{r}$&$R_{0} + 2 z^{3} - 1$\\
3&$\left(r + 1\right)^{2}$&$r + z^{3}$&$- 3 r - 6 z - 4 - \frac{1}{r}$&$- R_{0}^{2} + 2 z^{3} - 1$\\
  \bottomrule
  \end{tabular}
\end{center}

Погрешность решения задачи вычисляется следующим образом:
\[ \delta_1 = \frac{\left. \|v - \Tilde{v}\| \right._1}{\left. \|v\| \right._1} \]
\[ \delta_2 = \frac{\left. \|v - \Tilde{v}\| \right._2}{\left. \|v\| \right._2} \]
\[ \delta_3 = \frac{\left. \|v - \Tilde{v}\| \right._\infty}{\left. \|v\| \right._\infty} \]
В таблицах погрешности приведем еще одну колонку - отношение предыдущей ср. ариф. погрешности к текущей.

Параметры для всех тестов
\[ r \in [1, 3], \quad z \in [0, 2] \]

$N_z$ требуется брать как степень двойки.

Система алгебраических уравнений Ax=b с известным решением x для тестирования разработанной подпрограммы решения СЛАУ, Nr = 3, n = 2:
\begin{tiny}
\[ A = \begin{pmatrix}
  1 &   &   &   &   &   &   &   &   &   &   &   &   &   &   &   &   &   &   &  \\
    & 1 &   &   &   &   &   &   &   &   &   &   &   &   &   &   &   &   &   &  \\
    &   & 1 &   &   &   &   &   &   &   &   &   &   &   &   &   &   &   &   &  \\
    &   &   & 1 &   &   &   &   &   &   &   &   &   &   &   &   &   &   &   &  \\
  -1 &   &   &   & 5 & 2 &   &   & -1 &   &   &   &   &   &   &   &   &   &   &  \\
    & -1 &   &   & 2 & 5 & 2 &   &   & -1 &   &   &   &   &   &   &   &   &   &  \\
    &   & -1 &   &   & 2 & 5 &   &   &   & -1 &   &   &   &   &   &   &   &   &  \\
    &   &   & -1 &   &   &   & 1 &   &   &   & -1 &   &   &   &   &   &   &   &  \\
    &   &   &   & -1 &   &   &   & 5 & 2 &   &   & -1 &   &   &   &   &   &   &  \\
    &   &   &   &   & -1 &   &   & 2 & 5 & 2 &   &   & -1 &   &   &   &   &   &  \\
    &   &   &   &   &   & -1 &   &   & 2 & 5 &   &   &   & -1 &   &   &   &   &  \\
    &   &   &   &   &   &   & -1 &   &   &   & 1 &   &   &   & -1 &   &   &   &  \\
    &   &   &   &   &   &   &   & -1 &   &   &   & 5 & 2 &   &   & -1 &   &   &  \\
    &   &   &   &   &   &   &   &   & -1 &   &   & 2 & 5 & 2 &   &   & -1 &   &  \\
    &   &   &   &   &   &   &   &   &   & -1 &   &   & 2 & 5 &   &   &   & -1 &  \\
    &   &   &   &   &   &   &   &   &   &   & -1 &   &   &   & 1 &   &   &   & -1\\
    &   &   &   &   &   &   &   &   &   &   &   &   &   &   &   & 1 &   &   &  \\
    &   &   &   &   &   &   &   &   &   &   &   &   &   &   &   &   & 1 &   &  \\
    &   &   &   &   &   &   &   &   &   &   &   &   &   &   &   &   &   & 1 &  \\
    &   &   &   &   &   &   &   &   &   &   &   &   &   &   &   &   &   &   & 1\\
\end{pmatrix} \]
\[ b = \begin{pmatrix}
   .55 &  .72 &  .6 &  .54 & 1.89 & 3.87 & 2.11 & - .18 & 4.57 & 3.82 & 4.2 & - .45 & 3.73 & 4.72 &  .64 & -1.31 &  .02 &  .83 &  .78 &  .87\\
\end{pmatrix}^{T} \]
\[ x = \begin{pmatrix}
   .55 &  .72 &  .6 &  .54 &  .42 &  .65 &  .44 &  .89 &  .96 &  .38 &  .79 &  .53 &  .57 &  .93 &  .07 &  .09 &  .02 &  .83 &  .78 &  .87\\
\end{pmatrix}^{T} \]
\end{tiny}

\begin{table}[H]
  \begin{center}
    \begin{tabular}{*{5}c}
      \toprule
      $ N_r $ & $ N_z $ & $ \delta_1 $ & $ \delta_2 $ & $ \delta_3 $ \\
      \midrule
3&4&1.49e-16&2.12e-16&3.47e-16\\
      \bottomrule
    \end{tabular}
    \caption{Тест СЛАУ}
  \end{center}
\end{table}
\begin{table}[H]
  \begin{center}
    \begin{tabular}{*{6}c}
      \toprule
      $ N_r $ & $ N_z $ & $ \delta_1 $ & $ \delta_2 $ & $ \delta_3 $ & $\frac{\delta}{\delta}$ \\
      \midrule
2&2&4.93e-17&4.80e-17&2.96e-17&inf\\
4&4&9.69e-17&1.21e-16&1.20e-16&0.38\\
8&8&8.46e-16&4.15e-16&2.17e-16&0.23\\
16&16&8.56e-16&4.70e-16&3.24e-16&0.90\\
32&32&2.35e-15&2.63e-15&2.78e-15&0.21\\
      \bottomrule
    \end{tabular}
    \caption{Тест 1 (1, r + z**3)}
  \end{center}
\end{table}
\begin{table}[H]
  \begin{center}
    \begin{tabular}{*{6}c}
      \toprule
      $ N_r $ & $ N_z $ & $ \delta_1 $ & $ \delta_2 $ & $ \delta_3 $ & $\frac{\delta}{\delta}$ \\
      \midrule
2&2&2.35e-03&2.35e-03&1.73e-03&inf\\
4&4&1.18e-03&7.72e-04&4.22e-04&2.71\\
8&8&3.93e-04&2.07e-04&1.00e-04&3.39\\
16&16&1.11e-04&5.25e-05&2.43e-05&3.72\\
32&32&2.94e-05&1.32e-05&5.97e-06&3.87\\
      \bottomrule
    \end{tabular}
    \caption{Тест 2 (r + 1, r + z**3)}
  \end{center}
\end{table}
\begin{table}[H]
  \begin{center}
    \begin{tabular}{*{6}c}
      \toprule
      $ N_r $ & $ N_z $ & $ \delta_1 $ & $ \delta_2 $ & $ \delta_3 $ & $\frac{\delta}{\delta}$ \\
      \midrule
2&2&7.65e-03&7.67e-03&5.75e-03&inf\\
4&4&3.90e-03&2.60e-03&1.43e-03&2.66\\
8&8&1.30e-03&7.10e-04&3.44e-04&3.36\\
16&16&3.70e-04&1.82e-04&8.37e-05&3.71\\
32&32&9.79e-05&4.60e-05&2.06e-05&3.87\\
      \bottomrule
    \end{tabular}
    \caption{Тест 3 ((r + 1)**2, r + z**3)}
  \end{center}
\end{table}

Видно, что в тестах, где погрешности аппроксимации нет (тест 1), погрешность растет из-за ошибок округления. В тестах 2 и 3 увеличение числа разбиений в 2 раза приводит к уменьшению погрешности примерно в 4 раза.
% падение погрешности при увеличении числа разбиений в 2 раза с течением увеличения устанавливается в 4. 
Это соответствует анализу погрешности.

Все вычисления выполнены с двойной точностью.


  % \newpage
  % \section{Заключение}

  % В рамках данной курсовой работы мы использовали свои практические знания для
  % написания программного обеспечения, которое моделирует стационарные процессы
  % теплопроводности. Мы построили дискретную модель и протестировали её.

  % Была разработана программа моделирования стационарных процессов теплопроводности.
  % Привидены результаты экспериментального исследования точности программного обеспечения.

  % \newpage
  % \section*{Код программы}
\addcontentsline{toc}{section}{Код программы}

\inputminted[label=main.py,fontsize=\tiny]{python}{lst/main.py}

\end{document}
