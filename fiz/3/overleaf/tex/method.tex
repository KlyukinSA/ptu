\section{Метод нечётно-чётного исключения}

Коэффициенты $b_m, c_m, d_m$ не зависят от индекса j. Значит для каждой линии трёхдиагональный блок $C$ будет один и тот же:
\[ \begin{cases}
V_j = F_j, \quad j = 0 \\
-V_{j-1} + C V_j -V_{j+1} = F_j, \quad j = 1,2,\dots,N_z-1 \\
V_j = F_j, \quad j = N_z
\end{cases} \]

Обозначения
\[ N = N_z \]
\[ n = log_2(N) \in Z \]
\[ C^{(0)} = C, \quad F_j^{(0)} = F_j, \quad j=1,2,\dots,N-1 \]

Прямой ход
\[ C^{(k)} = (C^{(k-1)})^2 - 2E, \]
\[ F^{(k)}_j = F^{(k-1)}_{j-2^{k-1}} + C^{(k-1)} F^{(k-1)}_j + F^{(k-1)}_{j+2^{k-1}}, \]
\[ j = 2^k,2\cdot2^k,3\cdot2^k,\dots,N-2^k,\quad k = 1,2,\dots \]

Обратный
\[ C^{(k-1)} V_j = F^{(k-1)}_j + V_{j-2^{k-1}} + V_{j+2^{k-1}}, \]
\[ V_0 = F_0, \quad V_N = F_N, \]
\[ j = 2^{k-1},3\cdot2^{k-1},5\cdot2^{k-1},\dots,N-2^k,\quad k = n,n-1,\dots,1 \]

Введем векторы $p^{(k)}_j$ такие что
\[ F^{(k)}_j = \prod^{k-1}_{l=0} C^{(l)} p^{(k)}_j 2^k \]
\[ \prod^{-1}_{l=0} C^{(l)} = E, \quad p^{(0)}_j = F^{(0)}_j \]

Из прямого хода получаем
\[ 2 C^{(k-1)} p^{(k)}_j = p^{(k-1)}_{j-2^{k-1}} + C^{(k-1)} p^{(k-1)}_j + p^{(k-1)}_{j+2^{k-1}}, \]
Обозначим
\[ S^{(k-1)}_j = 2 p^{(k)}_j - p^{(k-1)}_j \]

Формулы для p
\[ C^{(k-1)} S^{(k-1)}_j = p^{(k-1)}_{j-2^{k-1}} + p^{(k-1)}_{j+2^{k-1}},\quad p^{(k)}_j = 0.5(p^{(k-1)}_j + S^{(k-1)}_j), \quad p^{(0)}_j = F^{(0)}_j, \]
\[ j = 2^k,2\cdot2^k,3\cdot2^k,\dots,N-2^k,\quad k = 1,2,\dots \]

Из обратного хода
\[ C^{(k-1)} V_j = 2^{k-1} \prod^{k-2}_{l=0} C^{(l)} p^{(k-1)}_j  + V_{j-2^{k-1}} + V_{j+2^{k-1}} \]

Обозначения

% $T_n(x)$ - полином Чебышева n-й степени первого рода

% $U_n(x)$ - полином Чебышева n-й степени второго рода

\[ C_{l, k-1} = C - 2 cos \frac{(2l-1) \pi }{2^k} E \]
\[ \alpha_{l, k-1} = \frac{(-1)^{l+1}}{2^{k-1}} sin \frac{(2l-1) \pi }{2^k} \]

Из теорем о полиномах Чебышева первого и второго рода
\[ (C^{(k-1)})^{-1} = \sum^{2^{k-1}}_{l=1} \alpha_{l, k-1} C_{l, k-1}^{-1} \]
\[ (C^{(k-1)})^{-1} \prod^{k-2 (!)}_{l=0} C^{(l)} = \frac{1}{2^{k-1}} \sum^{2^{k-1}}_{l=1} C_{l, k-1}^{-1} \]

Тогда формулы можно записать так
\[ S^{(k-1)}_j = \sum^{2^{k-1}}_{l=1} \alpha_{l, k-1} C_{l, k-1}^{-1} (p^{(k-1)}_{j-2^{k-1}} + p^{(k-1)}_{j+2^{k-1}}) \]
\[ V_j = \sum^{2^{k-1}}_{l=1} C_{l, k-1}^{-1} (p^{(k-1)}_j + \alpha_{l, k-1} (V_{j-2^{k-1}} + V_{j+2^{k-1}})) \]

\RestyleAlgo{tworuled}
\SetAlFnt{\normalsize}
\SetAlgoNoLine
\SetKw{KwBy}{by}

\begin{algorithm*}[H]
  \DontPrintSemicolon
  $ p^{(0)}_j = F_j, \quad j=1,2,\dots,N-1 $ \;
  \For{$k = 1$ \KwTo $n-1$}{
    \For{$j = 2^k$ \KwTo $N-2^k$ \KwBy $2^k$}{
      $\varphi = p^{(k-1)}_{j-2^{k-1}} + p^{(k-1)}_{j+2^{k-1}}$ \;
      $C_{l, k-1} v_l = \alpha_{l, k-1} \varphi, \quad l = 1,2,\dots,2^{k-1}$ \;
      $p^{(k)}_j = 0.5(p^{(k-1)}_j + v_1 + v_2 + \dots + v_{2^{k-1}})$ \;
    }
  }
  $V_0 = F_0, \quad V_N = F_N$ \;
  \For{$k = n$ \KwTo $1$}{
    \For{$j = 2^{k-1}$ \KwTo $N-2^{k-1}$ \KwBy $2\cdot 2^{k-1}$}{
      $\varphi = V_{j-2^{k-1}} + V_{j+2^{k-1}}, \quad \psi = p^{(k-1)}_j$ \;
      $C_{l, k-1} v_l = \psi + \alpha_{l, k-1} \varphi, \quad l = 1,2,\dots,2^{k-1}$ \;
      $V_j = v_1 + v_2 + \dots + v_{2^{k-1}}$ \;
    }
  }
\end{algorithm*}

Исходя из полученного метода, для хранения матрицы $A$ достаточно хранить один трёхдиагональный блок $C$, и хранить его надо в такой форме, которой достаточно, чтобы решать СЛАУ с ним. Будем хранить три вектора диагоналей.
