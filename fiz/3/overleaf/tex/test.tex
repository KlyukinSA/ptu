\section{Тестирование}

Судя по первой невязке погрешность появлется когда суммарная степень k и u относительно r становится 3. По второй - когда 2. Об общей погрешности будем судить по второй невязке $\varepsilon_2$.

Протестируем постренную модель на следующих тестовых наборах:
\begin{center}
  \begin{tabular}{c|c c|c c c c c|c|c c c|c}
  \toprule
  № теста & $k(r)$ & $u(r, z)$ & $\frac{\partial^4 u}{\partial r^4}$ & $\frac{\partial r k}{\partial r}\frac{\partial^3 u}{\partial r^3}$ & $\frac{\partial^2 r k}{\partial r^2}\frac{\partial^2 u}{\partial r^2}$ & $\frac{\partial^3 r k}{\partial r^3}\frac{\partial u}{\partial r}$ & $\frac{\partial^4 u}{\partial z^4}$ & $\varepsilon_1$ & $\frac{\partial^3 u}{ \partial r^3}$ & $\frac{\partial rk}{ \partial r} \frac{\partial^2 u}{ \partial r^2}$ & $\frac{\partial^2 rk}{ \partial r^2} \frac{\partial u}{ \partial r}$ & $\varepsilon_2$ \\
  \midrule
  1 & $ 1 $ & $r + z^3$ & 0 & 0 & 0 & 0 & 0 & 0 & 0 & 0 & 0 & 0 \\
  2 & $ r + 2z + 1 $ & $r + z^3$ & 0 & 0 & 0 & 0 & 0 & 0 & 0 & 0 & 1 & 1 \\
  3 & $ (r + 2z + 1)^2 $ & $r + z^3$ & 0 & 0 & 0 & 1 & 0 & 1 & 0 & 0 & 1 & 1 \\
  4 & $ sin(rz) + 2 $ & $log(r) + \sqrt{z}$ & 1 & 1 & 1 & 1 & 1 & 1 & 1 & 1 & 1 & 1 \\
  \bottomrule
  \end{tabular}
\end{center}

Вид функции $f$ можно вывести из вида функций $k,u$.

Вид функций $\varphi_i, i\neq1$ определяется видом функции $u$. Функция $\chi_1$ для работы метода решения СЛАУ должна не зависеть от $z$.
\[\chi_1(z) = 2\]
Можно вывести вид функции $\varphi_1$.

\begin{center}
  \begin{tabular}{c|c c|c c}
  \toprule
  № теста & $k(r)$ & $u(r, z)$ & $f(r, z)$ & $\varphi_1(z)$ \\
  \midrule
  1 & $ 1 $ & $r + z^3$ & $- \frac{1}{r} - 6 z$ & $- 1 + \left(R_{0} + z^{3}\right) \left(2\right)$ \\
  2 & $ r + 2z + 1 $ & $r + z^3$ & $- 6 z - 2 - \frac{2 z}{r} - \frac{1}{r}$ & $- R_{0} - 2 z + \left(R_{0} + z^{3}\right) \left(2\right) - 1$ \\
  3 & $ (r + 2z + 1)^2 $ & $r + z^3$ & $- 2 r - 10 z - 2 - \frac{\left(r + 2\right)^{2}}{r}$ & $\left(R_{0} + z^{3}\right) \left(2\right) - \left(R_{0} + 2 z + 1\right)^{2}$ \\
  4 & $ sin(rz) + 2 $ & $log(r) + \sqrt{z}$ & $\frac{0.25}{z^{1.5}} - \frac{z^{1.0} \cos{\left(r z \right)}}{r}$ & $\frac{R_{0} \left(z^{0.5} + \log{\left(R_{0} \right)}\right) \left(2\right) - \sin{\left(R_{0} z \right)} - 2}{R_{0}}$ \\
  \bottomrule
  \end{tabular}
\end{center}

Погрешность решения задачи вычисляется следующим образом:
\[ \delta_1 = \frac{\left. \|v - \Tilde{v}\| \right._1}{\left. \|v\| \right._1} \]
\[ \delta_2 = \frac{\left. \|v - \Tilde{v}\| \right._2}{\left. \|v\| \right._2} \]
\[ \delta_3 = \frac{\left. \|v - \Tilde{v}\| \right._\infty}{\left. \|v\| \right._\infty} \]

Параметры для всех тестов
\[ r \in [1, 3], \quad z \in [0, 2] \]

$N_z$ требуется брать как степень двойки.

Система алгебраических уравнений с известным решением для тестирования разработанной подпрограммы
\[ A = \begin{pmatrix}
  1 &   &   &   &   &   &   &   &   &   &   &   &   &   &   &  \\
    & 1 &   &   &   &   &   &   &   &   &   &   &   &   &   &  \\
    &   & 1 &   &   &   &   &   &   &   &   &   &   &   &   &  \\
    &   &   & 1 &   &   &   &   &   &   &   &   &   &   &   &  \\
  -1 &   &   &   & 5 & 2 &   &   & -1 &   &   &   &   &   &   &  \\
    & -1 &   &   & 2 & 5 & 2 &   &   & -1 &   &   &   &   &   &  \\
    &   & -1 &   &   & 2 & 5 &   &   &   & -1 &   &   &   &   &  \\
    &   &   &   &   &   &   & 1 &   &   &   &   &   &   &   &  \\
    &   &   &   & -1 &   &   &   & 5 & 2 &   &   & -1 &   &   &  \\
    &   &   &   &   & -1 &   &   & 2 & 5 & 2 &   &   & -1 &   &  \\
    &   &   &   &   &   & -1 &   &   & 2 & 5 &   &   &   & -1 &  \\
    &   &   &   &   &   &   &   &   &   &   & 1 &   &   &   &  \\
    &   &   &   &   &   &   &   &   &   &   &   & 1 &   &   &  \\
    &   &   &   &   &   &   &   &   &   &   &   &   & 1 &   &  \\
    &   &   &   &   &   &   &   &   &   &   &   &   &   & 1 &  \\
    &   &   &   &   &   &   &   &   &   &   &   &   &   &   & 1\\
\end{pmatrix} \]
\[ b = \begin{pmatrix}
   .55 &  .72 &  .6 &  .54 & 1.89 & 3.87 & 2.11 &  .89 & 4.57 & 3.82 & 4.2 &  .53 &  .57 &  .93 &
   . 7 &  . 9
\end{pmatrix}^{T} \]
\[ x = \begin{pmatrix}
   .55 &  .72 &  .6 &  .54 &  .42 &  .65 &  .44 &  .89 &  .96 &  .38 &  .79 &  .53 &  .57 &  .93 &
   . 7 &  . 9
\end{pmatrix}^{T} \]
