\section{Тестирование}

Судя по первой невязке погрешность появлется когда суммарная степень k и u относительно r становится 3. По второй - когда 2. Об общей погрешности будем судить по второй невязке $\xi_2$.

Протестируем постренную модель на следующих тестовых наборах:
% \begin{tiny}
\begin{center}
  \begin{tabular}{c|c c|c c |c| c c|c}
  \toprule
  № теста & $k(r)$ & $u(r, z)$ & $\Phi_{1,r}$ & $\Phi_{1,z}$ & $\xi_1$ & $\Phi_{2,r}$ & $\Phi_{2,z}$ & $\xi_2$ \\
  \midrule
  1 & $ 1 $ & $r + z^3$                     & 0 & 0 & 0 & 0 & 0 & 0 \\
  2 & $ r + 1 $ & $r + z^3$                 & 0 & 0 & 0 & 1 & 0 & 1 \\
  3 & $ (r + 1)^2 $ & $r + z^3$             & 1 & 0 & 1 & 1 & 0 & 1 \\
  \bottomrule
  \end{tabular}
\end{center}
% \end{tiny}
Вид функции $f$ можно вывести из вида функций $k,u$.

Вид функций $\varphi_i, i\neq1$ определяется видом функции $u$. Функция $\chi_1$ для работы метода решения СЛАУ должна не зависеть от $z$.
\[\chi_1(z) = 2\]
Можно вывести вид функции $\varphi_1$.

\begin{center}
  \begin{tabular}{c|c c|c c}
  \toprule
  № теста & $k(r)$ & $u(r, z)$ & $f(r, z)$ & $\varphi_1(z)$ \\
  \midrule
1&$1.0$&$r + z^{3}$&$- 6 z - \frac{1.0}{r}$&$2 R_{0} + 2 z^{3} - 1.0$\\
2&$r + 1$&$r + z^{3}$&$- 6 z - 2 - \frac{1}{r}$&$R_{0} + 2 z^{3} - 1$\\
3&$\left(r + 1\right)^{2}$&$r + z^{3}$&$- 3 r - 6 z - 4 - \frac{1}{r}$&$- R_{0}^{2} + 2 z^{3} - 1$\\
  \bottomrule
  \end{tabular}
\end{center}

Погрешность решения задачи вычисляется следующим образом:
\[ \delta_1 = \frac{\left. \|v - \Tilde{v}\| \right._1}{\left. \|v\| \right._1} \]
\[ \delta_2 = \frac{\left. \|v - \Tilde{v}\| \right._2}{\left. \|v\| \right._2} \]
\[ \delta_3 = \frac{\left. \|v - \Tilde{v}\| \right._\infty}{\left. \|v\| \right._\infty} \]
В таблицах погрешности приведем еще одну колонку - отношение предыдущей ср. ариф. погрешности к текущей.

Параметры для всех тестов
\[ r \in [1, 3], \quad z \in [0, 2] \]

$N_z$ требуется брать как степень двойки.

Система алгебраических уравнений Ax=b с известным решением x для тестирования разработанной подпрограммы решения СЛАУ, Nr = 3, n = 2:
\begin{tiny}
\[ A = \begin{pmatrix}
  1 &   &   &   &   &   &   &   &   &   &   &   &   &   &   &   &   &   &   &  \\
    & 1 &   &   &   &   &   &   &   &   &   &   &   &   &   &   &   &   &   &  \\
    &   & 1 &   &   &   &   &   &   &   &   &   &   &   &   &   &   &   &   &  \\
    &   &   & 1 &   &   &   &   &   &   &   &   &   &   &   &   &   &   &   &  \\
  -1 &   &   &   & 5 & 2 &   &   & -1 &   &   &   &   &   &   &   &   &   &   &  \\
    & -1 &   &   & 2 & 5 & 2 &   &   & -1 &   &   &   &   &   &   &   &   &   &  \\
    &   & -1 &   &   & 2 & 5 &   &   &   & -1 &   &   &   &   &   &   &   &   &  \\
    &   &   & -1 &   &   &   & 1 &   &   &   & -1 &   &   &   &   &   &   &   &  \\
    &   &   &   & -1 &   &   &   & 5 & 2 &   &   & -1 &   &   &   &   &   &   &  \\
    &   &   &   &   & -1 &   &   & 2 & 5 & 2 &   &   & -1 &   &   &   &   &   &  \\
    &   &   &   &   &   & -1 &   &   & 2 & 5 &   &   &   & -1 &   &   &   &   &  \\
    &   &   &   &   &   &   & -1 &   &   &   & 1 &   &   &   & -1 &   &   &   &  \\
    &   &   &   &   &   &   &   & -1 &   &   &   & 5 & 2 &   &   & -1 &   &   &  \\
    &   &   &   &   &   &   &   &   & -1 &   &   & 2 & 5 & 2 &   &   & -1 &   &  \\
    &   &   &   &   &   &   &   &   &   & -1 &   &   & 2 & 5 &   &   &   & -1 &  \\
    &   &   &   &   &   &   &   &   &   &   & -1 &   &   &   & 1 &   &   &   & -1\\
    &   &   &   &   &   &   &   &   &   &   &   &   &   &   &   & 1 &   &   &  \\
    &   &   &   &   &   &   &   &   &   &   &   &   &   &   &   &   & 1 &   &  \\
    &   &   &   &   &   &   &   &   &   &   &   &   &   &   &   &   &   & 1 &  \\
    &   &   &   &   &   &   &   &   &   &   &   &   &   &   &   &   &   &   & 1\\
\end{pmatrix} \]
\[ b = \begin{pmatrix}
   .55 &  .72 &  .6 &  .54 & 1.89 & 3.87 & 2.11 & - .18 & 4.57 & 3.82 & 4.2 & - .45 & 3.73 & 4.72 &  .64 & -1.31 &  .02 &  .83 &  .78 &  .87\\
\end{pmatrix}^{T} \]
\[ x = \begin{pmatrix}
   .55 &  .72 &  .6 &  .54 &  .42 &  .65 &  .44 &  .89 &  .96 &  .38 &  .79 &  .53 &  .57 &  .93 &  .07 &  .09 &  .02 &  .83 &  .78 &  .87\\
\end{pmatrix}^{T} \]
\end{tiny}

\begin{table}[H]
  \begin{center}
    \begin{tabular}{*{5}c}
      \toprule
      $ N_r $ & $ N_z $ & $ \delta_1 $ & $ \delta_2 $ & $ \delta_3 $ \\
      \midrule
3&4&1.49e-16&2.12e-16&3.47e-16\\
      \bottomrule
    \end{tabular}
    \caption{Тест СЛАУ}
  \end{center}
\end{table}
\begin{table}[H]
  \begin{center}
    \begin{tabular}{*{6}c}
      \toprule
      $ N_r $ & $ N_z $ & $ \delta_1 $ & $ \delta_2 $ & $ \delta_3 $ & $\frac{\delta}{\delta}$ \\
      \midrule
2&2&4.93e-17&4.80e-17&2.96e-17&inf\\
4&4&9.69e-17&1.21e-16&1.20e-16&0.38\\
8&8&8.46e-16&4.15e-16&2.17e-16&0.23\\
16&16&8.56e-16&4.70e-16&3.24e-16&0.90\\
32&32&2.35e-15&2.63e-15&2.78e-15&0.21\\
      \bottomrule
    \end{tabular}
    \caption{Тест 1 (1, r + z**3)}
  \end{center}
\end{table}
\begin{table}[H]
  \begin{center}
    \begin{tabular}{*{6}c}
      \toprule
      $ N_r $ & $ N_z $ & $ \delta_1 $ & $ \delta_2 $ & $ \delta_3 $ & $\frac{\delta}{\delta}$ \\
      \midrule
2&2&2.35e-03&2.35e-03&1.73e-03&inf\\
4&4&1.18e-03&7.72e-04&4.22e-04&2.71\\
8&8&3.93e-04&2.07e-04&1.00e-04&3.39\\
16&16&1.11e-04&5.25e-05&2.43e-05&3.72\\
32&32&2.94e-05&1.32e-05&5.97e-06&3.87\\
      \bottomrule
    \end{tabular}
    \caption{Тест 2 (r + 1, r + z**3)}
  \end{center}
\end{table}
\begin{table}[H]
  \begin{center}
    \begin{tabular}{*{6}c}
      \toprule
      $ N_r $ & $ N_z $ & $ \delta_1 $ & $ \delta_2 $ & $ \delta_3 $ & $\frac{\delta}{\delta}$ \\
      \midrule
2&2&7.65e-03&7.67e-03&5.75e-03&inf\\
4&4&3.90e-03&2.60e-03&1.43e-03&2.66\\
8&8&1.30e-03&7.10e-04&3.44e-04&3.36\\
16&16&3.70e-04&1.82e-04&8.37e-05&3.71\\
32&32&9.79e-05&4.60e-05&2.06e-05&3.87\\
      \bottomrule
    \end{tabular}
    \caption{Тест 3 ((r + 1)**2, r + z**3)}
  \end{center}
\end{table}

Видно, что в тестах, где погрешности аппроксимации нет (тест 1), погрешность растет из-за ошибок округления. В тестах 2 и 3 увеличение числа разбиений в 2 раза приводит к уменьшению погрешности примерно в 4 раза.
% падение погрешности при увеличении числа разбиений в 2 раза с течением увеличения устанавливается в 4. 
Это соответствует анализу погрешности.

Все вычисления выполнены с двойной точностью.
