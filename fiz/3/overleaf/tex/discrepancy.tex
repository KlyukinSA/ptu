\section{Невязка разностной схемы}

\subsection{Невязка во внутренних точках}

Будем искать невязку в следующем виде:
\[ \tilde{\xi}_{i, j} = \frac{\xi_{i, j}}{h_r h_z} \]

Обозначим
\[ \Phi_{1,r} = \frac{1}{12} r k \frac{\partial^4 u}{\partial r^4} + \frac{1}{6}\frac{\partial r k}{\partial r}\frac{\partial^3 u}{\partial r^3} + \frac{1}{8} \frac{\partial^2 r k}{\partial r^2}\frac{\partial^2 u}{\partial r^2} + \frac{1}{24} \frac{\partial^3 r k}{\partial r^3}\frac{\partial u}{\partial r} \]
\[ \Phi_{1,z} = \frac{1}{12} r \frac{\partial^4 u}{\partial z^4} \]

Тогда невязка
\begin{align*}
  &\tilde{\xi}_{i, j} = h_r^0 h_z^0 \left[ rf(r, z) + \pdv{}{r} \left ( r k(r) \pdv{u}{r} \right ) + \pdv[2]{ru}{z} \right]_{i, j} + \\
  & + h^2_r\left[ \Phi_{1,r} \right]_{i,j} + \mathcal{O}(h_r^3) + h^2_z \left[ \Phi_{1,z} \right]_{i, j} + \mathcal{O}(h_z^3)
\end{align*}

% Можно заметить, что первая скобка тождественно равна нулю по основному уравнению. Значит 
Порядок аппроксимации $ p_r = 2 - 0 = 2 $, $ p_z = 2 - 0 = 2 $.

\subsection{Невязка на левой границе}

Будем искать невязку в следующем виде:
\[ \tilde{\xi}_{0, j} = \frac{\xi_{0, j}}{h_z} \]

Обозначим
\[ \Phi_{2,r} = \frac{1}{6} rk \frac{\partial^3 u}{ \partial r^3} +   \frac{1}{4} \frac{\partial rk}{ \partial r} \frac{\partial^2 u}{ \partial r^2} + \frac{1}{8} \frac{\partial^2 rk}{ \partial r^2} \frac{\partial u}{ \partial r} \]
\[ \Phi_{2,z} = \frac{1}{24} \frac{\partial^4 u}{ \partial z^4} \]

Тогда невязка
\begin{align*}
  &\tilde{\xi}_{i, j} = h_r^0 \left[ r k(r) \pdv{u}{r} - r \chi_1 u + r \varphi_1(z) \right]_{i, j} + h_z^0 h_r \frac{1}{2} \left[ rf(r, z) + \pdv{}{r} \left ( r k(r) \pdv{u}{r} \right ) + \pdv[2]{ru}{z} \right]_{i, j} + \\
  & + h^2_r \left[ \Phi_{2,r} \right]_{i ,j} + \mathcal{O}(h^3_r) + h_r\left( r h^2_z\left[ \Phi_{2,z} \right]_{i, j}
  + \mathcal{O}(h^3_z) \right)
\end{align*}

% Можно заметить что первая скобка равна нулю по граничному условию. Значит это главный член погрешности относительно r и . Можно заметить что вторая скобка равна нулю и это главный член погрешности относительно z. Значит 
$ p_r = 2 - 0 = 2 $ и $ p_z = 2 - 0 = 2 $.

По определению невязки на остальных границах она равна нулю.

Можно сделать вывод, что разностная схема в целом имеет второй порядок аппроксимации.
