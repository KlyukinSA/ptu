\section{Невязка разностной схемы}

\subsection{Невязка во внутренних точках}

Будем искать невязку в следующем виде:
\[
  \tilde{\xi}_{i, j} = \frac{\xi_{i, j}}{h_r h_z}
\]

\begin{align*}
  &\tilde{\xi}_{i, j} = 
  h^2_r\left[ \frac{1}{12} r k \frac{\partial^4 u}{\partial r^4} + \frac{1}{6}\frac{\partial r k}{\partial r}\frac{\partial^3 u}{\partial r^3}
  + \frac{1}{8} \frac{\partial^2 r k}{\partial r^2}\frac{\partial^2 u}{\partial r^2}
  + \frac{1}{24} \frac{\partial^3 r k}{\partial r^3}\frac{\partial u}{\partial r} \right]_{i,j} + \mathcal{O}(h_r^3) + \\
  &+ h^2_z \left[
    \frac{1}{12} r \frac{\partial^4 u}{\partial z^4}
     \right]_{i, j} + \mathcal{O}(h_z^3)
\end{align*}
Порядок аппроксимации $p_r = 2 - 0 = 2 $, $ p_z = 2 - 0 = 2 $.

\subsection{Невязка на левой границе}

Будем искать невязку в следующем виде:
\[ \Tilde{\xi}_{0, j} = \frac{\xi_{0, j}}{h_z} \]
Получаем:
\begin{align*}
  &\Tilde{\xi}_{i, j} = 
  h^2_r \left[
  \frac{1}{6} rk \frac{\partial^3 u}{ \partial r^3} +
  \frac{1}{4} \frac{\partial rk}{ \partial r} \frac{\partial^2 u}{ \partial r^2} +
  \frac{1}{8} \frac{\partial^2 rk}{ \partial r^2} \frac{\partial u}{ \partial r}
  \right]_{i ,j} + \mathcal{O}(h^3_r) + \\
 & + h_r\left[ r h^2_z\left( 
  \frac{1}{24} \frac{\partial^4 u}{ \partial z^4}
  \right)_{i, j}
  + \mathcal{O}(h^3_z) \right]
\end{align*}

Порядок аппроксимации $ p_r = 2 - 0 = 2, p_z = 2 - 0 = 2 $.

По определению невязки на остальных границах она равна нулю.

Можно сделать вывод, что разностная схема в целом имеет второй порядок аппроксимации.
