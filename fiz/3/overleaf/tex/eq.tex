\section{Запись СЛАУ}

Перейдём к одноиндексной записи
\[ m = j(N_r + 1) + i + 1 \]

Индексы изменяются в следующих границах:
\[ 0 \leq i \leq N_r \]
\[ 0 \leq j \leq N_z \]

Тогда
\[ 1 \le m \le (N_r + 1)(N_z + 1) \]
, и все уравнения можно записать в следующем виде
\[ a_m w_{m - L} + b_m w_{m - 1} + c_m w_m + d_m w_{m + 1} + e_m w_{m + L} = g_m \]

\subsection{Запись для внутренних точек}

\[ a_m = e_m = -1 \]

\[ b_m = -\frac{h_z^2}{h_r^2} \frac{r_{i-\frac{1}{2}}}{r_i} k(r_{i-\frac{1}{2}}) \]
\[ c_m = 2 + \frac{h_z^2}{h_r^2 r_i} \left(r_{i-\frac{1}{2}} k(r_{i-\frac{1}{2}}) + r_{i+\frac{1}{2}} k(r_{i+\frac{1}{2}}) \right) \]
\[ d_m = -\frac{h_z^2}{h_r^2} \frac{r_{i+\frac{1}{2}}}{r_i} k(r_{i+\frac{1}{2}}) \]
\[ g_m = h_z^2 f_{i, j} \]

\subsection{Запись для левой границы}

для $ i = 0 $ и $ j = 1, 2, \dots, N_z - 1$.
\[ m = j(N_r + 1) + 1 \]

\[ a_m = e_m = -1 \]
\[ b_m = 0 \]

\[ c_m = 2 + 2 \frac{h_z^2}{h_r} \left(\chi_1(z_j) + \frac{r_{i+\frac{1}{2}}}{r_i h_r} k(r_{i+\frac{1}{2}}) \right) \]
\[ d_m = -2 \frac{h_z^2}{h_r^2} \frac{r_{i+\frac{1}{2}}}{r_i} k(r_{i+\frac{1}{2}}) \]
\[ g_m = h_z^2 \left(f_{i, j} + 2 \frac{\varphi_1(z_j)}{h_r} \right) \]

Для остальных точек

\[ a,b,d,e = 0 \]

\subsection{Структура матрицы для Nx=Ny=4}

Обозначим нулевые элементы пустыми клетками. Обозначим элементы, нарушающие симметричность крестиками.

\begin{center}
\begin{tabular}{ c c c c | c c c c | c c c c | c c c c }
01 & 02 & 03 & 04 & 05 & 06 & 07 & 08 & 09 & 10 & 11 & 12 & 13 & 14 & 15 & 16 \\
\hline 
1 & & & & & & & & & & & & & & & \\
 & 1 & & & & & & & & & & & & & & \\
 & & 1 & & & & & & & & & & & & & \\
 & & & 1 & & & & & & & & & & & & \\
\hline
 x & & & & * & * & & & * & & & & & & & \\
 & x & & & * & * & * & & & * & & & & & & \\
 & & x & & & * & * & x & & & * & & & & & \\
 & & & & & & & 1 & & & & & & & & \\
\hline 
 & & & & * & & & & * & * & & & x & & & \\
 & & & & & * & & & * & * & * & & & x & & \\
 & & & & & & * & & & * & * & x & & & x & \\
 & & & & & & & & & & & 1 & & & & \\
\hline
 & & & & & & & & & & & & 1 & & & \\
 & & & & & & & & & & & & & 1 & & \\
 & & & & & & & & & & & & & & 1 & \\
 & & & & & & & & & & & & & & & 1
\end{tabular}
\end{center}

В итоге мы получаем СЛАУ с нессиметричной матрицей.
% Чтобы убрать крестик из матрицы, можно воспользоваться умножением на скаляр и сложением строк.
Для работы используемого метода решения СЛАУ надо достроить боковые блоки до -E. Для этого из каждой строки в которой только одна единица и не в крайних блоках вычтем строки где в этой колонке стоит единица. Изменится вектор g и не изменится блок C.
