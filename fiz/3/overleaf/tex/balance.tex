\section{Разностная схема}
Введем основную сетку:
\begin{align*}
  &N_r - \text{число разбиений на } [R_0, R_1] & &N_z - \text{число разбиений на } [0, L] \\
  &r_0 < r_1 < \cdots < r_N & &z_0 < z_1 < \cdots < z_N \\
  &r_0 = R_0,\quad r_N = R_1 & &z_0 = 0,\quad z_N = L \\
  &h_r = \frac{R_1 - R_0}{N_r} & &h_z = \frac{L - 0}{N_z}
\end{align*}

Введем дополнительную сетку:
\begin{align*}
  &r_{i-\frac{1}{2}} = \frac{r_i + r_{i - 1}}{2}\quad i=1,\cdots, N_r & &z_{j-\frac{1}{2}} = \frac{z_j + z_{j - 1}}{2}\quad j=1,\cdots, N_z \\
  & \hbar_i = \begin{cases}
    \frac{h_r}{2},\ i = 0 \\ \\
    h_r,\ i = 1, 2, \dots, N_r-1 \\ \\
    \frac{h_r}{2},\ i = N_r
  \end{cases} &
  & \hbar_j = \begin{cases}
    \frac{h_z}{2},\ j = 0 \\ \\
    h_z,\ j = 1, 2, \dots, N_z-1 \\ \\
    \frac{h_z}{2},\ j = N_z
  \end{cases}
\end{align*}

Преобразуем наше начальное уравнение домножив на r

\[
  - \left [ \pdv{}{r} \left ( r k(r) \pdv{u}{r} \right ) 
  + \pdv[2]{ru}{z} \right ] = rf(r, z)
\]

\subsection{Внутренние точки}
% Проинтегрируем уравнение (\hyperref[eq1]{1}) в области
$ [r_{i -\frac{1}{2}}, r_{i +\frac{1}{2}}] \times  [z_{j -\frac{1}{2}}, z_{j +\frac{1}{2}}] $
для $ i = 1, 2, \dots, N_r - 1 $ и $ j = 1, 2, \dots, N_z - 1$.

\[
  - \mLim{+\frac{1}{2}}{-\frac{1}{2}}{+\frac{1}{2}}{-\frac{1}{2}} \left [ \pdv{}{r} \left ( r k(r) \pdv{u}{r} \right ) 
  + \pdv[2]{ru}{z} \right ] dr dz = \mLim{+\frac{1}{2}}{-\frac{1}{2}}{+\frac{1}{2}}{-\frac{1}{2}} rf(r, z) dr dz
\]

Получим:

\begin{align*}
  &- \left [
   \mLimZ{z}{+\frac{1}{2}}{-\frac{1}{2}}{j}  \left . r k(r) \pdv{u}{r} \right \vert_{r = r_{i + \frac{1}{2}}} dz
  - \mLimZ{z}{+\frac{1}{2}}{-\frac{1}{2}}{j} \left . r k(r) \pdv{u}{r} \right \vert_{r = r_{i - \frac{1}{2}}} dz
  \right . \\
  &\left . + \mLimS{r}{+\frac{1}{2}}{-\frac{1}{2}} \left . r \pdv{u}{z} \right \vert_{z = z_{j + \frac{1}{2}}} dr
  - \mLimS{r}{+\frac{1}{2}}{-\frac{1}{2}} \left . r \pdv{u}{z} \right \vert_{z = z_{j - \frac{1}{2}}} dr
  \right ] = \mLim{+\frac{1}{2}}{-\frac{1}{2}}{+\frac{1}{2}}{-\frac{1}{2}} rf(r, z) dr dz
\end{align*}

Воспользуемся формулами численного дифференцирования:
\[
  \left . k(r) \pdv{u}{r} \right \vert_{r = r_{i - \frac{1}{2}}}
  \approx k(r_{i - \frac{1}{2}}) \frac{v_{i, j} - v_{i - 1, j}}{h_r}
\]

\[
  \left . \pdv{u}{r} \right \vert_{z = z_{j - \frac{1}{2}}}
  \approx \frac{v_{i, j} - v_{i, j - 1}}{h_z}
\]

Также воспользуемся формулой средних прямоугольников:
\[
  \mLimS{r}{+\frac{1}{2}}{-\frac{1}{2}} r \varphi(r, z) dr
  = \hbar_i r_i \varphi_i
\]
\[
  \mLim{+\frac{1}{2}}{-\frac{1}{2}}{+\frac{1}{2}}{-\frac{1}{2}} r \varphi(r, z) drdz
  = \hbar_i\hbar_j r_i \varphi_{i, j}
\]

В итоге получим уравнение разностной схемы:
\begin{align*}
  &- \left [ 
  \hbar_j r_{i+\frac{1}{2}} k(r_{i+\frac{1}{2}}) \frac{v_{i+1, j} - v_{i, j}}{h_{r}}
  - \hbar_j r_{i-\frac{1}{2}} k(r_{i-\frac{1}{2}}) \frac{v_{i, j} - v_{i - 1, j}}{h_{r}}
  \right . \\
  &\left .
  + \hbar_i r_{i} \frac{v_{i, j + 1} - v_{i, j}}{h_{z}}
  - \hbar_i r_{i} \frac{v_{i, j} - v_{i, j - 1}}{h_z}
  \right ]  = \hbar_i \hbar_j r_i f_{i, j}
\end{align*}

Так как выбранная основная сетка является равномерной, то $ \hbar_i = h_r $ и $ \hbar_j = h_z$
, для $ i = 1, 2, \dots, N_r - 1 $ и $ j = 1, 2, \dots, N_z - 1$.

\begin{align*}
  &- \left [ 
  h_z r_{i+\frac{1}{2}} k(r_{i+\frac{1}{2}}) \frac{v_{i+1, j} - v_{i, j}}{h_{r}}
  - h_z r_{i-\frac{1}{2}} k(r_{i-\frac{1}{2}}) \frac{v_{i, j} - v_{i - 1, j}}{h_{r}}
  \right . \\
  &\left .
  + h_r r_{i} \frac{v_{i, j + 1} - v_{i, j}}{h_{z}}
  - h_r r_{i} \frac{v_{i, j} - v_{i, j - 1}}{h_z}
  \right ]  = h_r h_z r_i f_{i, j}
\end{align*}

Умножим на $\frac{h_z}{h_r r_i}$, чтобы получилась подходящая для применяемого метода решения СЛАУ форма.
\begin{align*}
  - \left [ 
  h_z^2 \frac{r_{i+\frac{1}{2}}}{r_i} k(r_{i+\frac{1}{2}}) \frac{v_{i+1, j} - v_{i, j}}{h_{r}^2}
  - h_z^2 \frac{r_{i-\frac{1}{2}}}{r_i} k(r_{i-\frac{1}{2}}) \frac{v_{i, j} - v_{i - 1, j}}{h_{r}^2}
  + v_{i, j + 1} - 2 v_{i, j} + v_{i, j - 1}
  \right ]  = h_z^2 f_{i, j}
\end{align*}

\subsection{На левой границе}

$ [r_{i}, r_{i +\frac{1}{2}}] \times  [z_{j -\frac{1}{2}}, z_{j +\frac{1}{2}}] $
для $ i = 0 $ и $ j = 1, 2, \dots, N_z - 1$.

Получаем:
\begin{align*}
  &- \left [
   \mLimZ{z}{+\frac{1}{2}}{-\frac{1}{2}}{j}  \left . r k(r) \pdv{u}{r} \right \vert_{r = r_{i + \frac{1}{2}}} dz
  - \mLimZ{z}{+\frac{1}{2}}{-\frac{1}{2}}{j} \left . r k(r) \pdv{u}{r} \right \vert_{r = r_{i}} dz
  \right . \\
  &\left . + \mLimS{r}{+\frac{1}{2}}{} \left . r\pdv{u}{z} \right \vert_{z = z_{j + \frac{1}{2}}} dr
  - \mLimS{r}{+\frac{1}{2}}{} \left . r\pdv{u}{z} \right \vert_{z = z_{j - \frac{1}{2}}} dr
  \right ] = \mLim{+\frac{1}{2}}{}{+\frac{1}{2}}{-\frac{1}{2}} rf(r, z) dr dz
\end{align*}

Имеем граничное условие:
\[ \left . k(r) \pdv{u}{r} \right \vert_{r=R_0} = \chi_1 \left . u \right \vert_{r=R_0} - \varphi_1(z) \]

Получаем уравнение разностной схемы:

\begin{align*}
  &- \left [ 
  h_z r_{i+\frac{1}{2}} k(r_{i+\frac{1}{2}}) \frac{v_{i+1, j} - v_{i, j}}{h_{r}}
  - h_z r_{i} (\chi_1(z_j) v_{i, j} - \varphi_1(z_j))
  \right . \\
  &\left .
  + \frac{h_r}{2} r_{i} \frac{v_{i, j + 1} - v_{i, j}}{h_{z}}
  - \frac{h_r}{2} r_{i} \frac{v_{i, j} - v_{i, j - 1}}{h_z}
  \right ]  = \frac{h_r}{2} h_z r_i f_{i, j}
\end{align*}

\begin{align*}
  - \left [ 
  2 h_z^2 \frac{r_{i+\frac{1}{2}}}{r_i} k(r_{i+\frac{1}{2}}) \frac{v_{i+1, j} - v_{i, j}}{h_{r}^2}
  - 2 h_z^2 \frac{\chi_1(z_j) v_{i, j} - \varphi_1(z_j)}{h_r}
  + v_{i, j + 1} - 2 v_{i, j} + v_{i, j - 1}
  \right ]  = h_z^2 f_{i, j}
\end{align*}

\subsection{На правой границе}
$ [r_{i -\frac{1}{2}}, r_{i}] \times  [z_{j -\frac{1}{2}}, z_{j +\frac{1}{2}}] $
для $ i = N_r $ и $ j = 1, 2, \dots, N_z - 1$.

Имеем граничное условие:
\[ \left . u \right \vert_{r=R_1} = \varphi_2(z) \]

Будем его сразу использовать:
\[ v_{N_r,j} = \varphi_2(z_j) \]

\subsection{На нижней границе}
$ [r_{i  -\frac{1}{2}}, r_{i +\frac{1}{2}}] \times  [z_{j}, z_{j +\frac{1}{2}}] $
для $ i = 1, 2, \dots, N_r - 1 $ и $ j = 0$.

\[ \left . u \right \vert_{z=0} = \varphi_3(r) \]
\[ v_{i,0} = \varphi_3(r_i) \]

\subsection{На верхней границе}
$ i = 1, 2, \dots, N_r - 1 $ и $ j = N_z $
\[ \left . u \right \vert_{z=L} = \varphi_4(r) \]
\[ v_{i,N_z} = \varphi_4(r_i) \]

\subsection{Левый-нижний угол}

$ [r_{i}, r_{i +\frac{1}{2}}] \times  [z_{j}, z_{j +\frac{1}{2}}] $
для $ i = 0 $ и $ j = 0$.
\[ v_{0,0} = \varphi_3(r_0) \]

\subsection{Левый-верхний угол}

$ i = 0 $ и $ j = N_z $

\[ v_{0,N_z} = \varphi_4(r_0) \]

\subsection{Правый-верхний угол}

$ i = N_r $ и $ j = N_z $, возьмём известное граничное условие:

\[ v_{N_r,N_z} = \varphi_4(r_{N_r}) \]

\subsection{Правый-нижний угол}

$ i = N_r $ и $ j = 0 $.

\[ v_{N_r,0} = \varphi_2(z_0) \]

\subsection{На нижней границе}
$ [r_{i  -\frac{1}{2}}, r_{i +\frac{1}{2}}] \times  [z_{j}, z_{j +\frac{1}{2}}] $
для $ i = 1, 2, \dots, N_r - 1 $ и $ j = 0$.

\[ \left . u \right \vert_{z=0} = \varphi_3(r) \]
\[ v_{i,0} = \varphi_3(r_i) \]

\subsection{На верхней границе}
$ i = 1, 2, \dots, N_r - 1 $ и $ j = N_z $
\[ \left . u \right \vert_{z=L} = \varphi_4(r) \]
\[ v_{i,N_z} = \varphi_4(r_i) \]
