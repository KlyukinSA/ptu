\documentclass[a4paper,12pt]{article}

\usepackage[T2A]{fontenc} % ru lang
% \usepackage[utf8]{inputenc}
\usepackage[english,russian]{babel} % ru lang

% \usepackage{fontspec}
% \setmainfont[Scale=1.167]{Times New Roman} % ??????

% \usepackage{unicode-math}

% \usepackage{microtype}
\usepackage[linesnumbered,ruled,vlined]{algorithm2e}

% \usepackage{amsmath, amsfonts, amsthm}
\usepackage{amsmath} % align*
% \usepackage{listings}
% \usepackage{enumerate}
\usepackage{float} % tables not float
% \usepackage{graphicx}
% \usepackage{nameref}
% \usepackage{hyperref}
% \usepackage{tabularx}
% \usepackage{makecell}
\usepackage{indentfirst} % ru style
\usepackage{booktabs} % table horisontal lines
\usepackage{derivative} % partial \pdv
% \usepackage{physics}

% \usepackage{tikz}
% \usepackage{tikzscale}
% \usetikzlibrary{decorations.pathreplacing,calligraphy}
% \usetikzlibrary{arrows.meta}
% \usetikzlibrary{backgrounds,fit,positioning}
% \usetikzlibrary{external}

\usepackage[newfloat]{minted} % listings

% \setminted{
%   frame=lines,
%   framesep=2mm,
%   baselinestretch=1.2,
%   fontsize=\footnotesize,
%   linenos,
%   breaklines
% }

\usepackage[top=3cm,bottom=3cm,left=1cm,right=1cm]{geometry} % large formulas not overful hbox
% \hypersetup{
% 	colorlinks=true,
% 	linkcolor=blue,
% 	filecolor=magenta,
% 	urlcolor=cyan
% }
% \urlstyle{same}
\setcounter{MaxMatrixCols}{16} % pmatrix


\begin{document}
	\begin{titlepage}	% начало титульной страницы

	\begin{center}		% выравнивание по центру

		\large Санкт-Петербургский политехнический университет Петра Великого\\
		\large Институт компьютерных наук и кибербезопасности\\
		\large Высшая школа программной инженерии \\[6cm]
		% название института, затем отступ 6см

    \huge Курсовая Работа\\[0.5cm] % название работы, затем отступ 0,5см
		\large Разработка программного обеспечения для воспроизведения на ЭВМ двумерных стационарных моделей \\[0.1cm]
		\large по дисциплине \\
		\large <<Разработка программного обеспечения для моделирования физических процессов>> \\[5cm]

	\end{center}

		\noindent\large Выполнил: \hfill \large Клюкин С. А.\\
		\noindent\large Группа: \hfill \large гр. 5130904/10102\\

		\noindent\large Проверил: \hfill \large Воскобойников С. П.

	\vfill % заполнить всё доступное ниже пространство

	\begin{center}
	\large Санкт-Петербург\\
	\large \the\year % вывести дату
	\end{center} % закончить выравнивание по центру

\end{titlepage} % конец титульной страницы

\vfill % заполнить всё доступное ниже пространство

	\newpage
	\tableofcontents
	\newpage

	\section{Вступление}
	\subsection{Постановка задачи}

	Вариант C3. Используя интегро-интерполяционный метод (метод баланса), разработать программу для моделирования стационарного распределения температуры в цилиндре, описываемого математической моделью вида
	\begin{align*}
		-\left[ \frac{1}{r} \frac{d}{dr} \left(r k(r)\frac{du}{dr} \right)\ -\ q(r)u \right]
		&= f(r),\ r \in \left[ 0,\ R\right],\ R > 0,
		\\
		 0 < c_1 \leq k(r) \leq c_2,\ 0 \leq q(r)&
	\end{align*}
	Граничные условия: 
 
	\centerline{\(\left. u\right\vert_{r = 0}\) - ограничено}
	\[-k \left. \frac{du}{dr}\right\vert_{r = R} = \chi\left.u\right\vert_{r = R} - \nu, \chi > 0\]
    
	\subsection{Используемое ПО}

	\begin{enumerate}
		\item Python 3.10.12
		\item numpy 1.26.2
	\end{enumerate}
	\newpage

	\section{Основная часть}
	\subsection{Интегро-интеполяционный метод (метод баланса)}

Введем основную сетку, где N - число разбиений.
\[r_0 < r_1 < \dots < r_N,\ r_i \in [0, R],\ r_0 = 0,\ r_N = R\]
\[
  h_i =r_i - r_{i-1},\ i=1,2, \dots, N
\]
\[
  r_{r-0.5} = \frac{r_i - r_{i-1}}{2},\ i=1,2, \dots, N
\]
Введем дополнительную сетку:
\[
  \hbar_i = \begin{cases}
    \frac{h_{i+1}}{2},\ i = 0 \\ \\
    \frac{h_i + h_{i+1}}{2},\ i = 1, 2, \dots, N-1 \\ \\
    \frac{h_i}{2},\ i = N
  \end{cases}
\]
Проведем аппроксимацию начального уравнения.

Получаем разностную схему для i = 0:
\[
  -\left[ r_{i+0.5} \cdot k_{i+0.5}\frac{v_{i+1}-v_i}{h_{i+1}} - \frac{r_{i+0.5} \hbar_i q_i v_i}{2} \right]\ =\ \frac{\hbar_i r_{i+0.5} f_i}{2}
\]

Получаем разностную схему для i\ =\ 1, 2, \dots, N-1:
\[
  -\left[ r_{i+0.5} \cdot k_{i+0.5}\frac{v_{i+1}-v_i}{h_{i+1}} - r_{i-0.5}k_{i-0.5}\frac{v_{i}\ -\ v_{i-1}}{h_{i}} - \hbar r_i q_i v_i\right]\ =\ \hbar_ir_if_i
\]

Получаем разностную схему для i=N:
\[
  -\left[ r_i (-\chi v_i + \nu) - r_{i-0.5}k_{i-0.5} \cdot \frac{v_i-v_{i-1}}{h_i}- \hbar_ir_iq_iv_i \right]\ =\ \hbar_ir_if_i
\]

После аппроксимации уравнения можно представить в виде системы из трёхдиагональной матрицы где a, c, b - диагонали матрица A и вектора g. Элементы матрицы:\newline
Для i\ =\ 0
\begin{align*}
  c_i = \frac{\hbar_i r_{i+0.5} q_i}{2} + \frac{r_{i+0.5} k_{i+0.5}}{h_{i+1}} \quad
  b_i = -r_{i+0.5} \cdot \frac{k_{i+0.5}}{h_{i+1}} \quad
  g_i = \frac{\hbar_i r_{i+0.5} f_i}{2}
\end{align*}
Для i\ =\ 1, 2, \dots, N-1
\begin{align*}
  a_i &= -r_{i-0.5}\frac{k_{i-0.5}}{h_i} \quad
  c_i = r_{i-0.5}\frac{k_{i-0.5}}{h_i} + r_{i+0.5}\frac{k_{i+0.5}}{h_{i+1}} + \hbar_i r_iq_i \quad
  b_i = -r_{i+0.5}\frac{k_{i+0.5}}{h_{i+1}} \\
  g_i &= \hbar_i r_i f_i
\end{align*}
Для i\ =\ N:
\begin{align*}
  a_i = -r_{i-0.5}\frac{k_{i-0.5}}{h_i} \quad
  c_i = \chi r_i + r_{i-0.5}\frac{k_{i-0.5}}{h_i} + \hbar_i r_iq_i \quad
  g_i = \hbar_i r_i f_i + \nu r_i
\end{align*}


	\subsection{Метод прогонки}

Метод прогонки это простой способ решать трёхдиагональные системы.
\begin{align*}
  \begin{pmatrix}
    c_1 & b_1 & &  &  &  & 0 \\
    a_2 & c_2 & b_2 &  & & & \\
     &   \ddots & \ddots & \ddots & & & \\
     & &  \ddots & \ddots & \ddots & &  \\
     & & & \ddots & \ddots & \ddots &  \\
     &  &  & & a_{n-1} & c_{n-1} & b_{n-1} \\
     0 & & & &  &a_n & c_n
  \end{pmatrix}
  \begin{pmatrix}
    x_1 \\
    x_2 \\
    \vdots \\
    \vdots \\
    \vdots \\
    x_{n-1} \\
    x_n
  \end{pmatrix} =
  \begin{pmatrix}
    g_1 \\
    g_2 \\
    \vdots \\
    \vdots \\
    \vdots \\
    g_{n-1} \\
    g_n
  \end{pmatrix}
\end{align*}
\subsubsection*{Этап 1}
\textbf{Строка 1.} Разделим первую строку на $c_1$:
\begin{align*}
  &c_1 x_1+b_1x_2 = g_1 \\
  &x_1 + \textcolor{red}{\gamma_1} x_2 = \textcolor{red}{\rho_1},\ \gamma_1 = \frac{b_1}{c_1},\ \rho_1 = \frac{g_1}{c_1}
\end{align*}
\begin{align*}
  \begin{pmatrix}
    \textcolor{red}{1} & \textcolor{red}{\gamma_1} & &  &  &  & 0 \\
    a_2 & c_2 & b_2 &  & & & \\
     &   \ddots & \ddots & \ddots & & & \\
     & &  \ddots & \ddots & \ddots & &  \\
     & & & \ddots & \ddots & \ddots &  \\
     &  &  & & a_{n-1} & c_{n-1} & b_{n-1} \\
     0 & & & &  &a_n & c_n
  \end{pmatrix}
  \begin{pmatrix}
    x_1 \\
    x_2 \\
    \vdots \\
    \vdots \\
    \vdots \\
    x_{n-1} \\
    x_n
  \end{pmatrix} =
  \begin{pmatrix}
    \textcolor{red}{\rho_1} \\
    g_2 \\
    \vdots \\
    \vdots \\
    \vdots \\
    g_{n-1} \\
    g_n
  \end{pmatrix}
\end{align*}
\textbf{Строки от 2 до N-1}. Здесь представлена общая формула для всех строк в промежутке
\[	a_n x_{n-1} + c_n x_n + b_n x_{n + 1} = g_n,\ n = 2, 3, \dots, N-1 \]
Умножим n-1 строку на $a_n$ и вычтем из строки под номером n. Получим строку
\[ (c_n - a_n \cdot \textcolor{red}{\gamma_{n-1}})x_n + c_n x_{n + 1} = g_n - a_n \textcolor{red}{\rho_{n - 1}} \]
Разделим на $ (c_n - a_n \cdot \gamma_{n-1}) $
\[	x_n + \frac{b_n}{c_n - a_n \gamma_{n-1}} x_{n+1} = \frac{g_n - a_n \rho_{n-1}}{c_n - a_n\gamma_{n-1}} \]
\[	x_n + \textcolor{red}{\gamma_n} x_{n+1} = \textcolor{red}{\rho_n},\ \gamma_n = \frac{b_n}{c_n - a_n\gamma_{n-1}},\ \rho_n = \frac{g_n - a_n\rho_{n-1}}{c_n-a_n\gamma_{n-1}} \]

\begin{align*}
  \begin{pmatrix}
    \textcolor{red}{1} & \textcolor{red}{\gamma_1} & &  &  &  & 0 \\
    0 & \textcolor{red}{1} & \textcolor{red}{\gamma_2} &  & & & \\
     &   \ddots & \ddots & \ddots & & & \\
     & &  \ddots & \ddots & \ddots & &  \\
     & & & \ddots & \ddots & \ddots &  \\
     &  &  & & a_{n-1} & c_{n-1} & b_{n-1} \\
     0 & & & &  &a_n & c_n
  \end{pmatrix}
  \begin{pmatrix}
    x_1 \\
    x_2 \\
    \vdots \\
    \vdots \\
    \vdots \\
    x_{n-1} \\
    x_n
  \end{pmatrix} =
  \begin{pmatrix}
    \textcolor{red}{\rho_1} \\
    \textcolor{red}{\rho_2} \\
    \vdots \\
    \vdots \\
    \vdots \\
    g_{n-1} \\
    g_n
  \end{pmatrix}
\end{align*}
\newline
\textbf{Строка N.}
\begin{align*}
  &a_n x_{n-1} + c_n x_n = g_n \\
  &x_n= \textcolor{red}{\rho_n},\ \rho_n=\frac{r_n - a_n \rho_{n-1}}{c_n -a_n \gamma_{n-1}}
\end{align*}
\begin{align*}
  \begin{pmatrix}
    \textcolor{red}{1} & \textcolor{red}{\gamma_1} & &  &  &  & 0 \\
    0 & \textcolor{red}{1} & \textcolor{red}{\gamma_2} &  & & & \\
     &   \ddots & \ddots & \ddots & & & \\
     & &  \ddots & \ddots & \ddots & &  \\
     & & & \ddots & \ddots & \ddots &  \\
     &  &  & & \textcolor{red}{0} & \textcolor{red}{1} & \textcolor{red}{\gamma_{n-1}} \\
     0 & & & &  & \textcolor{red}{0} & \textcolor{red}{1}
  \end{pmatrix}
  \begin{pmatrix}
    x_1 \\
    x_2 \\
    \vdots \\
    \vdots \\
    \vdots \\
    x_{n-1} \\
    x_n
  \end{pmatrix} =
  \begin{pmatrix}
    \textcolor{red}{\rho_1} \\
    \textcolor{red}{\rho_2} \\
    \vdots \\
    \vdots \\
    \vdots \\
    \textcolor{red}{\rho_{n-1}} \\
    \textcolor{red}{\rho_n}
  \end{pmatrix}
\end{align*}
\subsubsection*{Этап 2}
Чтобы узнать значения вектора x нам нужно "подняться"\ по уже вычисленным значеням.
\begin{align*}
  &x_n= \textcolor{red}{\rho_n} \\
  &x_i = \textcolor{red}{\rho_i - \gamma_{i} x_{i+1}},\ i= n-1, n-2, \dots, 1
\end{align*}
	\newpage

	\subsection{Оценка погрешности}

\subsubsection{Невязка разностной схемы}

\[
  Av = g;\ A - (N + 1) \cdot (N + 1); v, g \in R^{(N + 1)}
\]

Пусть $ v $ - это точное решение разностной схемы, $ u $ -  точное решение дифференциального уравнения, 
$ \tilde{v} $ - полученное решение разностной схемы.

Ищем погрешность решения разностной схемы:
\[
  \varepsilon = \tilde{v} - u
\]

Введем обозначения:
\begin{itemize}
  \item Погрешность решения системы линейных алгебраических уравнений
  \[ z = \tilde{v} - v \]
  \item Погрешность от аппроксимации дифференциального уравнения разностной схемой
  \[ \zeta = v - u \]
  \item Невязка разностной схемы
  \[ \xi = g - Au \]
  \item Невязка алгебраической системы
  \[ r = g - A\tilde{v} \]
\end{itemize}

\subsubsection{Структура погрешности разностной схемы}
\[
  \left\lVert \varepsilon \right\rVert = \left\lVert \tilde{v} - u \right\rVert =
  \left\lVert \tilde{v} - v + v - u \right\rVert = \left\lVert z + \zeta  \right\rVert \leq \left\lVert z \right\rVert
  + \left\lVert \zeta \right\rVert 
\]

Для $\left\lVert \zeta\right\rVert$:
\[
  \xi = g - Au = A(A^{-1}g - u) = A(v - u)
\]
\[
  A\zeta  = \xi
\]

Тем самым погрешность от аппроксимации дифференциального уравнения разностной схемой, связана с невязкой разностной схемы:
\[
  \zeta = A^{-1} \xi 
\]

Для $\left\lVert z \right\rVert$:
\[
  r = g - A\tilde{v} = A(A^{-1}g - \tilde{v}) = A(v - \tilde{v})
\]
\[
  Az = r
\]
Тем самым погрешность решения системы линейных алгебраических уравнений, связана с невязкой алгебраической системы:
\[
  z = A^{-1}r
\]

Подставим в наше неравенство, тем самым получаем:
\[
  \left\lVert \varepsilon \right\rVert \leq \left\lVert A^{-1}r \right\rVert + \left\lVert A^{-1}\xi \right\rVert
  \leq \left\lVert A^{-1} \right\rVert ( \left\lVert r \right\rVert  + \left\lVert \xi \right\rVert)
  \quad \left\lVert A^{-1}\right\rVert  < C
\]

\subsubsection{Вклад от погрешности решения системы алгебраических уравнений}
\[
  \left\lVert z \right\rVert \leq \left\lVert A^{-1} \right\rVert \left\lVert r \right\rVert =
  \left\lVert A \right\rVert \left\lVert A^{-1} \right\rVert \frac{\left\lVert r \right\rVert }{\left\lVert A \right\rVert } 
\]

Знаем что:
\[
  \left\lVert A \right\rVert \geq \frac{\left\lVert g \right\rVert }{\left\lVert v\right\rVert }
\]

Из этого получаем:
\[
  \left\lVert z \right\rVert \leq \left\lVert A \right\rVert \left\lVert A^{-1} \right\rVert
  \frac{\left\lVert r \right\rVert }{\left\lVert A \right\rVert } \left\lVert v \right\rVert 
\]
\[
  cond(A) = \left\lVert A \right\rVert \left\lVert A^{-1} \right\rVert
\]
\[
  \frac{\left\lVert r \right\rVert }{\left\lVert A \right\rVert } \sim \varepsilon_{M}
\]


\[
  \left\lVert z \right\rVert \leq cond(A) \varepsilon_{M} \left\lVert v \right\rVert 
\]

Чтобы оценить поведение числа обусловленности, можно взять простую модельную квадратичную задачу, модифицированную для поиска собственных значений её матрицы А.
\[
  -\frac{d^2u}{dx} = \lambda u(x), x \in [0, L], u(0) = u(L) = 0
\]
% \begin{align*}
%     -\frac{d^2u}{dx} = \lambda u(x) & x \in [0, L] & u(0) = 0  u(L) = 0
% \end{align*}
Далее можно решить и получить собственные значения с помощью численного решения через разностную схему. Окажется, что собственные значения неотрицательны и отношение наибольшего к наименьшему примерно пропорционально $ N^2 $. При этом число обусловенности \( cond(A) = \frac{|\lambda|_{max}}{|\lambda|_{min}} \). Тогда погрешность $ \left\lVert z \right\rVert \sim \frac{1}{h^2} $. Используем это и в качестве грубой оценки погрешности для других задач.

\subsubsection{Разложение невязки}

Введём букву n = 1.

Граничное условие слева:
\begin{center}
  \includegraphics[width=0.7\textwidth]{img/ei0.png}
\end{center}
p = 2 - 0 = 2

Граничное условие справа:
\begin{center}
  \includegraphics[width=0.7\textwidth]{img/eiN.png}
\end{center}
Второе слагаемое из условия равно нулю.
p = 2 - 0 = 2

Основное уравнение:
\begin{center}
  \includegraphics[width=0.7\textwidth]{img/emid.png}
\end{center}
p = 3 - 1 = 2

\subsubsection{Зависимость погрешности от числа разбиений}

Выходит, что во всех случаях невязка $\left\lVert \xi \right\rVert \sim h^2$. А вот  погрешность решения системы алгебраических уравнений $ \left\lVert z \right\rVert \sim \frac{1}{h^2} $.
\begin{center}
  \includegraphics[width=0.7\textwidth]{img/err.pdf}
\end{center}

Значит существует такое число разбиений где погрешность минимальная.
	\newpage

	\subsection{Тестирование}

\subsubsection{Метод прогонки}

Сначала протестируем на простом случае
\begin{align*}
a = \begin{pmatrix} 0 & 0 & 0 \end{pmatrix}^\mathsf{T} \quad
b = \begin{pmatrix} 0 & 0 & 0 \end{pmatrix}^\mathsf{T} \quad
c = \begin{pmatrix} 1 & 1 & 1 \end{pmatrix}^\mathsf{T} \quad
g = \begin{pmatrix} 1 & 1 & 1 \end{pmatrix}^\mathsf{T}
\end{align*}
Очевидно что ответ \(u = \begin{pmatrix} 1 & 1 & 1 \end{pmatrix}^\mathsf{T}\)

Потом возьмем пример из задачника
\begin{align*}
a = \begin{pmatrix} 0 & 2 & 2 & 3 \end{pmatrix}^\mathsf{T} \quad
b = \begin{pmatrix} -1 & -1 & -0.8 & 0 \end{pmatrix}^\mathsf{T} \quad
c = \begin{pmatrix} 5 & 4,6 & 3,6 & 4,4 \end{pmatrix}^\mathsf{T} \\
g = \begin{pmatrix} 2 & 3,3 & 2,6 & 7,2 \end{pmatrix}^\mathsf{T} \quad
u = \begin{pmatrix} 0.5256 & 0.628 & 0.64 & 1.2 \end{pmatrix}^\mathsf{T}
\end{align*}

Для каждого из тестов посчитаем разницу полученного решения и истинного (u) и выберем наибольший по модулю элемент полученного вектора

1. 0

2. 1.1E-16

\subsubsection{Интегро-интерполяционный метод}

Выполним нашу программу на ряде входных параметров. Берем тестовые k, q, u, R, $\chi$, потом вычисляем f, $\nu$.

\begin{table}[H]
  \centering
  \begin{tabular}{ c | *{8}c }
    \toprule
    Номер теста & k & q & u & R & $\chi $ & f & $\nu $ \\
    \midrule
    1 & $ 3r+2 $ & $ 4r+3 $ & 1 & 1 & 1 & $ 4r+3 $ & 1 \\
    \midrule
    2 & $ r+1 $ & $ 2r+2 $ & $ r^2 $ & 1 & 1 & $ 2r^3+2r^2-6r-4 $ & 5 \\
    \bottomrule
  \end{tabular}
\end{table}

Первый тест характеризуется отсутствием погрешности разностной схемы, а второй - присутствием.

После запуска программы мы получаем следующие результаты. Выпишем наибольшие по модулю элементы вектора разности истинного решения с полученным для разного значения N – количество отсчетов.

  \begin{table}[H]
    \centering
    \begin{tabular}{c | c | c}
      \toprule
      N & Тест 1 & Тест 2 \\
      \midrule
4 & 0.0E+00 & 4.1E-02 \\
8 & 0.0E+00 & 1.0E-02 \\
16 & 0.0E+00 & 2.6E-03 \\
32 & 0.0E+00 & 6.5E-04 \\
64 & 5.7E-05 & 1.4E-04 \\
128 & 2.8E-05 & 2.4E-05 \\
256 & 7.1E-05 & 3.9E-05 \\
512 & 1.8E-04 & 6.4E-05 \\
1024 & 2.2E-02 & 9.0E-05 \\
2048 & 5.2E-02 & 4.5E-02 \\
4096 & 5.4E-02 & 5.4E-02 \\
8192 & 4.2E-01 & 2.8E+00 \\

      \bottomrule
    \end{tabular}
    \caption{Наибольшие по модулю элементы вектора разности истинного решения с полученным (погрешности)}
  \end{table}

Поведение погрешности в тесте 1 можно объяснить необходимостью решения системы алгебраических уравнений.

Поведение погрешности в тесте 2 можно объяснить тем, что сначала действует погрешность разностной схемы, а потом начинает проявляться погрешность решения системы алгебраических уравнений.

Покажем во сколько раз убывают погрешности во втором тесте с ростом N.

  \begin{table}[H]
    \centering
    \begin{tabular}{c | c}
      \toprule
      N & Тест 2 \\
      \midrule
4 & 3.9 \\
8 & 4.0 \\
16 & 4.0 \\
32 & 4.7 \\
64 & 5.9 \\
128 & 0.6 \\
256 & 0.6 \\
512 & 0.7 \\
1024 & 0.0 \\
2048 & 0.8 \\
4096 & 0.0 \\

      \bottomrule
    \end{tabular}
    \caption{Отношения погрешностей (текущее к следующему)}
  \end{table}

Можно проследить зависимость погрешности от числа разбиений, которую мы выявили в предыдущей главе - при увеличении шага в 2 раза погрешность должна уменьшаться примерно в 4 раза.

Все вычисления проведены с одинарной точностью.

	\newpage

	\section{Заключение}
	\subsection{Вывод}
	Задание выполнено в полном объеме.
	Был написан метод приближенного решения краевой задачи и метод прогонки.
	Программа была протестирована на разных функциях.
	Была оценена погрешность и выявлена зависимость погрешности решения от числа разбиений.
	\newpage
	\subsection{Код}
	\inputminted{python}{main.py}
\end{document}
