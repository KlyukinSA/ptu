\documentclass[a4paper,12pt]{article}

\usepackage[T2A]{fontenc} % ru lang
% \usepackage[utf8]{inputenc}
\usepackage[english,russian]{babel} % ru lang

\usepackage[linesnumbered,ruled,vlined]{algorithm2e}

\usepackage{amsmath} % align*
\usepackage{float} % tables not float
\usepackage{indentfirst} % ru style
\usepackage{booktabs} % table horisontal lines
\usepackage{derivative} % partial \pdv

\usepackage[newfloat]{minted} % listings

\usepackage[top=3cm,bottom=3cm,left=2cm,right=2cm]{geometry} % large formulas not overful hbox
\setcounter{MaxMatrixCols}{25} % pmatrix


\begin{document}
	\begin{titlepage}	% начало титульной страницы

	\begin{center}		% выравнивание по центру

		\large Санкт-Петербургский политехнический университет Петра Великого\\
		\large Институт компьютерных наук и кибербезопасности\\
		\large Высшая школа программной инженерии \\[6cm]
		% название института, затем отступ 6см

    \huge Курсовая Работа\\[0.5cm] % название работы, затем отступ 0,5см
		\large Разработка программного обеспечения для воспроизведения на ЭВМ двумерных стационарных моделей \\[0.1cm]
		\large по дисциплине \\
		\large <<Разработка программного обеспечения для моделирования физических процессов>> \\[5cm]

	\end{center}

		\noindent\large Выполнил: \hfill \large Клюкин С. А.\\
		\noindent\large Группа: \hfill \large гр. 5130904/10102\\

		\noindent\large Проверил: \hfill \large Воскобойников С. П.

	\vfill % заполнить всё доступное ниже пространство

	\begin{center}
	\large Санкт-Петербург\\
	\large \the\year % вывести дату
	\end{center} % закончить выравнивание по центру

\end{titlepage} % конец титульной страницы

\vfill % заполнить всё доступное ниже пространство

	\newpage
	\tableofcontents
	\newpage

	\section{Вступление}
	\subsection{Постановка задачи}

	Вариант C. Используя интегро-интерполяционный метод (метод баланса), разработать программу для моделирования стационарного распределения температуры в цилиндре, описываемого математической моделью вида
	\begin{align*}
		-\left[ \frac{1}{r} \frac{d}{dr} \left(r k(r)\frac{du}{dr} \right)\ -\ q(r)u \right]
		&= f(r),\ r \in \left[ 0,\ R\right],\ R > 0,
		\\
		 0 < c_1 \leq k(r) \leq c_2,\ 0 \leq q(r)&
	\end{align*}
	Граничные условия: 
 
    \(\left. u\right\vert_{r = 0}\) - ограничено 
    
    \(-k \left. \frac{du}{dr}\right\vert_{r = R} = \chi\left.u\right\vert_{r = R} -\nu, \chi > 0\)
    
	\subsection{Используемое ПО}

	\begin{enumerate}
		\item Python 3.10.12
		\item numpy 1.26.2
	\end{enumerate}
	\newpage

	\section{Основная часть}
	\section{Разностная схема}
Введем основную сетку:
\begin{align*}
  &N_r - \text{число разбиений на } [R_0, R_1] & &N_z - \text{число разбиений на } [0, L] \\
  &r_0 < r_1 < \cdots < r_N & &z_0 < z_1 < \cdots < z_N \\
  &r_0 = R_0,\quad r_N = R_1 & &z_0 = 0,\quad z_N = L \\
  &h_r = \frac{R_1 - R_0}{N_r} & &h_z = \frac{L - 0}{N_z}
\end{align*}

Введем дополнительную сетку:
\begin{align*}
  &r_{i-\frac{1}{2}} = \frac{r_i + r_{i - 1}}{2}\quad i=1,\cdots, N_r & &z_{j-\frac{1}{2}} = \frac{z_j + z_{j - 1}}{2}\quad j=1,\cdots, N_z \\
  & \hbar_i = \begin{cases}
    \frac{h_r}{2},\ i = 0 \\ \\
    h_r,\ i = 1, 2, \dots, N_r-1 \\ \\
    \frac{h_r}{2},\ i = N_r
  \end{cases} &
  & \hbar_j = \begin{cases}
    \frac{h_z}{2},\ j = 0 \\ \\
    h_z,\ j = 1, 2, \dots, N_z-1 \\ \\
    \frac{h_z}{2},\ j = N_z
  \end{cases}
\end{align*}

Преобразуем наше начальное уравнение домножив на r

\[
  - \left [ \pdv{}{r} \left ( r k(r) \pdv{u}{r} \right ) 
  + \pdv[2]{ru}{z} \right ] = rf(r, z)
\]

\subsection{Внутренние точки}
% Проинтегрируем уравнение (\hyperref[eq1]{1}) в области
$ [r_{i -\frac{1}{2}}, r_{i +\frac{1}{2}}] \times  [z_{j -\frac{1}{2}}, z_{j +\frac{1}{2}}] $
для $ i = 1, 2, \dots, N_r - 1 $ и $ j = 1, 2, \dots, N_z - 1$.

\[
  - \mLim{+\frac{1}{2}}{-\frac{1}{2}}{+\frac{1}{2}}{-\frac{1}{2}} \left [ \pdv{}{r} \left ( r k(r) \pdv{u}{r} \right ) 
  + \pdv[2]{ru}{z} \right ] dr dz = \mLim{+\frac{1}{2}}{-\frac{1}{2}}{+\frac{1}{2}}{-\frac{1}{2}} rf(r, z) dr dz
\]

Получим:

\begin{align*}
  &- \left [
   \mLimZ{z}{+\frac{1}{2}}{-\frac{1}{2}}{j}  \left . r k(r) \pdv{u}{r} \right \vert_{r = r_{i + \frac{1}{2}}} dz
  - \mLimZ{z}{+\frac{1}{2}}{-\frac{1}{2}}{j} \left . r k(r) \pdv{u}{r} \right \vert_{r = r_{i - \frac{1}{2}}} dz
  \right . \\
  &\left . + \mLimS{r}{+\frac{1}{2}}{-\frac{1}{2}} \left . r \pdv{u}{z} \right \vert_{z = z_{j + \frac{1}{2}}} dr
  - \mLimS{r}{+\frac{1}{2}}{-\frac{1}{2}} \left . r \pdv{u}{z} \right \vert_{z = z_{j - \frac{1}{2}}} dr
  \right ] = \mLim{+\frac{1}{2}}{-\frac{1}{2}}{+\frac{1}{2}}{-\frac{1}{2}} rf(r, z) dr dz
\end{align*}

Воспользуемся формулами численного дифференцирования:
\[
  \left . k(r) \pdv{u}{r} \right \vert_{r = r_{i - \frac{1}{2}}}
  \approx k(r_{i - \frac{1}{2}}) \frac{v_{i, j} - v_{i - 1, j}}{h_r}
\]

\[
  \left . \pdv{u}{r} \right \vert_{z = z_{j - \frac{1}{2}}}
  \approx \frac{v_{i, j} - v_{i, j - 1}}{h_z}
\]

Также воспользуемся формулой средних прямоугольников:
\[
  \mLimS{r}{+\frac{1}{2}}{-\frac{1}{2}} r \varphi(r, z) dr
  = \hbar_i r_i \varphi_i
\]
\[
  \mLim{+\frac{1}{2}}{-\frac{1}{2}}{+\frac{1}{2}}{-\frac{1}{2}} r \varphi(r, z) drdz
  = \hbar_i\hbar_j r_i \varphi_{i, j}
\]

В итоге получим уравнение разностной схемы:
\begin{align*}
  &- \left [ 
  \hbar_j r_{i+\frac{1}{2}} k(r_{i+\frac{1}{2}}) \frac{v_{i+1, j} - v_{i, j}}{h_{r}}
  - \hbar_j r_{i-\frac{1}{2}} k(r_{i-\frac{1}{2}}) \frac{v_{i, j} - v_{i - 1, j}}{h_{r}}
  \right . \\
  &\left .
  + \hbar_i r_{i} \frac{v_{i, j + 1} - v_{i, j}}{h_{z}}
  - \hbar_i r_{i} \frac{v_{i, j} - v_{i, j - 1}}{h_z}
  \right ]  = \hbar_i \hbar_j r_i f_{i, j}
\end{align*}

Так как выбранная основная сетка является равномерной, то $ \hbar_i = h_r $ и $ \hbar_j = h_z$
, для $ i = 1, 2, \dots, N_r - 1 $ и $ j = 1, 2, \dots, N_z - 1$.

\begin{align*}
  &- \left [ 
  h_z r_{i+\frac{1}{2}} k(r_{i+\frac{1}{2}}) \frac{v_{i+1, j} - v_{i, j}}{h_{r}}
  - h_z r_{i-\frac{1}{2}} k(r_{i-\frac{1}{2}}) \frac{v_{i, j} - v_{i - 1, j}}{h_{r}}
  \right . \\
  &\left .
  + h_r r_{i} \frac{v_{i, j + 1} - v_{i, j}}{h_{z}}
  - h_r r_{i} \frac{v_{i, j} - v_{i, j - 1}}{h_z}
  \right ]  = h_r h_z r_i f_{i, j}
\end{align*}

Умножим на $\frac{h_z}{h_r r_i}$, чтобы получилась подходящая для применяемого метода решения СЛАУ форма.
\begin{align*}
  - \left [ 
  h_z^2 \frac{r_{i+\frac{1}{2}}}{r_i} k(r_{i+\frac{1}{2}}) \frac{v_{i+1, j} - v_{i, j}}{h_{r}^2}
  - h_z^2 \frac{r_{i-\frac{1}{2}}}{r_i} k(r_{i-\frac{1}{2}}) \frac{v_{i, j} - v_{i - 1, j}}{h_{r}^2}
  + v_{i, j + 1} - 2 v_{i, j} + v_{i, j - 1}
  \right ]  = h_z^2 f_{i, j}
\end{align*}

\subsection{На левой границе}

$ [r_{i}, r_{i +\frac{1}{2}}] \times  [z_{j -\frac{1}{2}}, z_{j +\frac{1}{2}}] $
для $ i = 0 $ и $ j = 1, 2, \dots, N_z - 1$.

Получаем:
\begin{align*}
  &- \left [
   \mLimZ{z}{+\frac{1}{2}}{-\frac{1}{2}}{j}  \left . r k(r) \pdv{u}{r} \right \vert_{r = r_{i + \frac{1}{2}}} dz
  - \mLimZ{z}{+\frac{1}{2}}{-\frac{1}{2}}{j} \left . r k(r) \pdv{u}{r} \right \vert_{r = r_{i}} dz
  \right . \\
  &\left . + \mLimS{r}{+\frac{1}{2}}{} \left . r\pdv{u}{z} \right \vert_{z = z_{j + \frac{1}{2}}} dr
  - \mLimS{r}{+\frac{1}{2}}{} \left . r\pdv{u}{z} \right \vert_{z = z_{j - \frac{1}{2}}} dr
  \right ] = \mLim{+\frac{1}{2}}{}{+\frac{1}{2}}{-\frac{1}{2}} rf(r, z) dr dz
\end{align*}

Имеем граничное условие:
\[ \left . k(r) \pdv{u}{r} \right \vert_{r=R_0} = \chi_1 \left . u \right \vert_{r=R_0} - \varphi_1(z) \]

Получаем уравнение разностной схемы:

\begin{align*}
  &- \left [ 
  h_z r_{i+\frac{1}{2}} k(r_{i+\frac{1}{2}}) \frac{v_{i+1, j} - v_{i, j}}{h_{r}}
  - h_z r_{i} (\chi_1(z_j) v_{i, j} - \varphi_1(z_j))
  \right . \\
  &\left .
  + \frac{h_r}{2} r_{i} \frac{v_{i, j + 1} - v_{i, j}}{h_{z}}
  - \frac{h_r}{2} r_{i} \frac{v_{i, j} - v_{i, j - 1}}{h_z}
  \right ]  = \frac{h_r}{2} h_z r_i f_{i, j}
\end{align*}

\begin{align*}
  - \left [ 
  2 h_z^2 \frac{r_{i+\frac{1}{2}}}{r_i} k(r_{i+\frac{1}{2}}) \frac{v_{i+1, j} - v_{i, j}}{h_{r}^2}
  - 2 h_z^2 \frac{\chi_1(z_j) v_{i, j} - \varphi_1(z_j)}{h_r}
  + v_{i, j + 1} - 2 v_{i, j} + v_{i, j - 1}
  \right ]  = h_z^2 f_{i, j}
\end{align*}

\subsection{На правой границе}
$ [r_{i -\frac{1}{2}}, r_{i}] \times  [z_{j -\frac{1}{2}}, z_{j +\frac{1}{2}}] $
для $ i = N_r $ и $ j = 1, 2, \dots, N_z - 1$.

Имеем граничное условие:
\[ \left . u \right \vert_{r=R_1} = \varphi_2(z) \]

Будем его сразу использовать:
\[ v_{N_r,j} = \varphi_2(z_j) \]

\subsection{На нижней границе}
$ [r_{i  -\frac{1}{2}}, r_{i +\frac{1}{2}}] \times  [z_{j}, z_{j +\frac{1}{2}}] $
для $ i = 1, 2, \dots, N_r - 1 $ и $ j = 0$.

\[ \left . u \right \vert_{z=0} = \varphi_3(r) \]
\[ v_{i,0} = \varphi_3(r_i) \]

\subsection{На верхней границе}
$ i = 1, 2, \dots, N_r - 1 $ и $ j = N_z $
\[ \left . u \right \vert_{z=L} = \varphi_4(r) \]
\[ v_{i,N_z} = \varphi_4(r_i) \]

\subsection{Левый-нижний угол}

$ [r_{i}, r_{i +\frac{1}{2}}] \times  [z_{j}, z_{j +\frac{1}{2}}] $
для $ i = 0 $ и $ j = 0$.
\[ v_{0,0} = \varphi_3(r_0) \]

\subsection{Левый-верхний угол}

$ i = 0 $ и $ j = N_z $

\[ v_{0,N_z} = \varphi_4(r_0) \]

\subsection{Правый-верхний угол}

$ i = N_r $ и $ j = N_z $, возьмём известное граничное условие:

\[ v_{N_r,N_z} = \varphi_4(r_{N_r}) \]

\subsection{Правый-нижний угол}

$ i = N_r $ и $ j = 0 $.

\[ v_{N_r,0} = \varphi_2(z_0) \]

\subsection{На нижней границе}
$ [r_{i  -\frac{1}{2}}, r_{i +\frac{1}{2}}] \times  [z_{j}, z_{j +\frac{1}{2}}] $
для $ i = 1, 2, \dots, N_r - 1 $ и $ j = 0$.

\[ \left . u \right \vert_{z=0} = \varphi_3(r) \]
\[ v_{i,0} = \varphi_3(r_i) \]

\subsection{На верхней границе}
$ i = 1, 2, \dots, N_r - 1 $ и $ j = N_z $
\[ \left . u \right \vert_{z=L} = \varphi_4(r) \]
\[ v_{i,N_z} = \varphi_4(r_i) \]


	\subsection{Метод прогонки}

Метод прогонки это простой способ решать трёхдиагональные системы.
\begin{align*}
  \begin{pmatrix}
    c_1 & b_1 & &  &  &  & 0 \\
    a_2 & c_2 & b_2 &  & & & \\
     &   \ddots & \ddots & \ddots & & & \\
     & &  \ddots & \ddots & \ddots & &  \\
     & & & \ddots & \ddots & \ddots &  \\
     &  &  & & a_{n-1} & c_{n-1} & b_{n-1} \\
     0 & & & &  &a_n & c_n
  \end{pmatrix}
  \begin{pmatrix}
    x_1 \\
    x_2 \\
    \vdots \\
    \vdots \\
    \vdots \\
    x_{n-1} \\
    x_n
  \end{pmatrix} =
  \begin{pmatrix}
    g_1 \\
    g_2 \\
    \vdots \\
    \vdots \\
    \vdots \\
    g_{n-1} \\
    g_n
  \end{pmatrix}
\end{align*}
\subsubsection*{Этап 1}
\textbf{Строка 1.} Разделим первую строку на $c_1$:
\begin{align*}
  &c_1 x_1+b_1x_2 = g_1 \\
  &x_1 + \textcolor{red}{\gamma_1} x_2 = \textcolor{red}{\rho_1},\ \gamma_1 = \frac{b_1}{c_1},\ \rho_1 = \frac{g_1}{c_1}
\end{align*}
\begin{align*}
  \begin{pmatrix}
    \textcolor{red}{1} & \textcolor{red}{\gamma_1} & &  &  &  & 0 \\
    a_2 & c_2 & b_2 &  & & & \\
     &   \ddots & \ddots & \ddots & & & \\
     & &  \ddots & \ddots & \ddots & &  \\
     & & & \ddots & \ddots & \ddots &  \\
     &  &  & & a_{n-1} & c_{n-1} & b_{n-1} \\
     0 & & & &  &a_n & c_n
  \end{pmatrix}
  \begin{pmatrix}
    x_1 \\
    x_2 \\
    \vdots \\
    \vdots \\
    \vdots \\
    x_{n-1} \\
    x_n
  \end{pmatrix} =
  \begin{pmatrix}
    \textcolor{red}{\rho_1} \\
    g_2 \\
    \vdots \\
    \vdots \\
    \vdots \\
    g_{n-1} \\
    g_n
  \end{pmatrix}
\end{align*}
\textbf{Строки от 2 до N-1}. Здесь представлена общая формула для всех строк в промежутке
\[	a_n x_{n-1} + c_n x_n + b_n x_{n + 1} = g_n,\ n = 2, 3, \dots, N-1 \]
Умножим n-1 строку на $a_n$ и вычтем из строки под номером n. Получим строку
\[ (c_n - a_n \cdot \textcolor{red}{\gamma_{n-1}})x_n + c_n x_{n + 1} = g_n - a_n \textcolor{red}{\rho_{n - 1}} \]
Разделим на $ (c_n - a_n \cdot \gamma_{n-1}) $
\[	x_n + \frac{b_n}{c_n - a_n \gamma_{n-1}} x_{n+1} = \frac{g_n - a_n \rho_{n-1}}{c_n - a_n\gamma_{n-1}} \]
\[	x_n + \textcolor{red}{\gamma_n} x_{n+1} = \textcolor{red}{\rho_n},\ \gamma_n = \frac{b_n}{c_n - a_n\gamma_{n-1}},\ \rho_n = \frac{g_n - a_n\rho_{n-1}}{c_n-a_n\gamma_{n-1}} \]

\begin{align*}
  \begin{pmatrix}
    \textcolor{red}{1} & \textcolor{red}{\gamma_1} & &  &  &  & 0 \\
    0 & \textcolor{red}{1} & \textcolor{red}{\gamma_2} &  & & & \\
     &   \ddots & \ddots & \ddots & & & \\
     & &  \ddots & \ddots & \ddots & &  \\
     & & & \ddots & \ddots & \ddots &  \\
     &  &  & & a_{n-1} & c_{n-1} & b_{n-1} \\
     0 & & & &  &a_n & c_n
  \end{pmatrix}
  \begin{pmatrix}
    x_1 \\
    x_2 \\
    \vdots \\
    \vdots \\
    \vdots \\
    x_{n-1} \\
    x_n
  \end{pmatrix} =
  \begin{pmatrix}
    \textcolor{red}{\rho_1} \\
    \textcolor{red}{\rho_2} \\
    \vdots \\
    \vdots \\
    \vdots \\
    g_{n-1} \\
    g_n
  \end{pmatrix}
\end{align*}
\newline
\textbf{Строка N.}
\begin{align*}
  &a_n x_{n-1} + c_n x_n = g_n \\
  &x_n= \textcolor{red}{\rho_n},\ \rho_n=\frac{r_n - a_n \rho_{n-1}}{c_n -a_n \gamma_{n-1}}
\end{align*}
\begin{align*}
  \begin{pmatrix}
    \textcolor{red}{1} & \textcolor{red}{\gamma_1} & &  &  &  & 0 \\
    0 & \textcolor{red}{1} & \textcolor{red}{\gamma_2} &  & & & \\
     &   \ddots & \ddots & \ddots & & & \\
     & &  \ddots & \ddots & \ddots & &  \\
     & & & \ddots & \ddots & \ddots &  \\
     &  &  & & \textcolor{red}{0} & \textcolor{red}{1} & \textcolor{red}{\gamma_{n-1}} \\
     0 & & & &  & \textcolor{red}{0} & \textcolor{red}{1}
  \end{pmatrix}
  \begin{pmatrix}
    x_1 \\
    x_2 \\
    \vdots \\
    \vdots \\
    \vdots \\
    x_{n-1} \\
    x_n
  \end{pmatrix} =
  \begin{pmatrix}
    \textcolor{red}{\rho_1} \\
    \textcolor{red}{\rho_2} \\
    \vdots \\
    \vdots \\
    \vdots \\
    \textcolor{red}{\rho_{n-1}} \\
    \textcolor{red}{\rho_n}
  \end{pmatrix}
\end{align*}
\subsubsection*{Этап 2}
Чтобы узнать значения вектора x нам нужно "подняться"\ по уже вычисленным значеням.
\begin{align*}
  &x_n= \textcolor{red}{\rho_n} \\
  &x_i = \textcolor{red}{\rho_i - \gamma_{i} x_{i+1}},\ i= n-1, n-2, \dots, 1
\end{align*}
	\newpage

	\subsection{Оценка погрешности}

\subsubsection{Невязка разностной схемы}

\[
  Av = g;\ A - (N + 1) \cdot (N + 1); v, g \in R^{(N + 1)}
\]

Пусть $ v $ - это точное решение разностной схемы, $ u $ -  точное решение дифференциального уравнения, 
$ \tilde{v} $ - полученное решение разностной схемы.

Ищем погрешность решения разностной схемы:
\[
  \varepsilon = \tilde{v} - u
\]

Введем обозначения:
\begin{itemize}
  \item Погрешность решения системы линейных алгебраических уравнений
  \[ z = \tilde{v} - v \]
  \item Погрешность от аппроксимации дифференциального уравнения разностной схемой
  \[ \zeta = v - u \]
  \item Невязка разностной схемы
  \[ \xi = g - Au \]
  \item Невязка алгебраической системы
  \[ r = g - A\tilde{v} \]
\end{itemize}

\subsubsection{Структура погрешности разностной схемы}
\[
  \left\lVert \varepsilon \right\rVert = \left\lVert \tilde{v} - u \right\rVert =
  \left\lVert \tilde{v} - v + v - u \right\rVert = \left\lVert z + \zeta  \right\rVert \leq \left\lVert z \right\rVert
  + \left\lVert \zeta \right\rVert 
\]

Для $\left\lVert \zeta\right\rVert$:
\[
  \xi = g - Au = A(A^{-1}g - u) = A(v - u)
\]
\[
  A\zeta  = \xi
\]

Тем самым погрешность от аппроксимации дифференциального уравнения разностной схемой, связана с невязкой разностной схемы:
\[
  \zeta = A^{-1} \xi 
\]

Для $\left\lVert z \right\rVert$:
\[
  r = g - A\tilde{v} = A(A^{-1}g - \tilde{v}) = A(v - \tilde{v})
\]
\[
  Az = r
\]
Тем самым погрешность решения системы линейных алгебраических уравнений, связана с невязкой алгебраической системы:
\[
  z = A^{-1}r
\]

Подставим в наше неравенство, тем самым получаем:
\[
  \left\lVert \varepsilon \right\rVert \leq \left\lVert A^{-1}r \right\rVert + \left\lVert A^{-1}\xi \right\rVert
  \leq \left\lVert A^{-1} \right\rVert ( \left\lVert r \right\rVert  + \left\lVert \xi \right\rVert)
  \quad \left\lVert A^{-1}\right\rVert  < C
\]

\subsubsection{Вклад от погрешности решения системы алгебраических уравнений}
\[
  \left\lVert z \right\rVert \leq \left\lVert A^{-1} \right\rVert \left\lVert r \right\rVert =
  \left\lVert A \right\rVert \left\lVert A^{-1} \right\rVert \frac{\left\lVert r \right\rVert }{\left\lVert A \right\rVert } 
\]

Знаем что:
\[
  \left\lVert A \right\rVert \geq \frac{\left\lVert g \right\rVert }{\left\lVert v\right\rVert }
\]

Из этого получаем:
\[
  \left\lVert z \right\rVert \leq \left\lVert A \right\rVert \left\lVert A^{-1} \right\rVert
  \frac{\left\lVert r \right\rVert }{\left\lVert A \right\rVert } \left\lVert v \right\rVert 
\]
\[
  cond(A) = \left\lVert A \right\rVert \left\lVert A^{-1} \right\rVert
\]
\[
  \frac{\left\lVert r \right\rVert }{\left\lVert A \right\rVert } \sim \varepsilon_{M}
\]


\[
  \left\lVert z \right\rVert \leq cond(A) \varepsilon_{M} \left\lVert v \right\rVert 
\]

Чтобы оценить поведение числа обусловленности, можно взять простую модельную квадратичную задачу, модифицированную для поиска собственных значений её матрицы А.
\[
  -\frac{d^2u}{dx} = \lambda u(x), x \in [0, L], u(0) = u(L) = 0
\]
% \begin{align*}
%     -\frac{d^2u}{dx} = \lambda u(x) & x \in [0, L] & u(0) = 0  u(L) = 0
% \end{align*}
Далее можно решить и получить собственные значения с помощью численного решения через разностную схему. Окажется, что собственные значения неотрицательны и отношение наибольшего к наименьшему примерно пропорционально $ N^2 $. При этом число обусловенности \( cond(A) = \frac{|\lambda|_{max}}{|\lambda|_{min}} \). Тогда погрешность $ \left\lVert z \right\rVert \sim \frac{1}{h^2} $. Используем это и в качестве грубой оценки погрешности для других задач.

\subsubsection{Разложение невязки}

Введём букву n = 1.

Граничное условие слева:
\begin{center}
  \includegraphics[width=0.7\textwidth]{img/ei0.png}
\end{center}
p = 2 - 0 = 2

Граничное условие справа:
\begin{center}
  \includegraphics[width=0.7\textwidth]{img/eiN.png}
\end{center}
Второе слагаемое из условия равно нулю.
p = 2 - 0 = 2

Основное уравнение:
\begin{center}
  \includegraphics[width=0.7\textwidth]{img/emid.png}
\end{center}
p = 3 - 1 = 2

\subsubsection{Зависимость погрешности от числа разбиений}

Выходит, что во всех случаях невязка $\left\lVert \xi \right\rVert \sim h^2$. А вот  погрешность решения системы алгебраических уравнений $ \left\lVert z \right\rVert \sim \frac{1}{h^2} $.
\begin{center}
  \includegraphics[width=0.7\textwidth]{img/err.pdf}
\end{center}

Значит существует такое число разбиений где погрешность минимальная.
	\newpage

	\subsection{Тестирование}

\subsubsection{Метод прогонки}

Сначала протестируем на простом случае
\begin{align*}
a = \begin{pmatrix} 0 & 0 & 0 \end{pmatrix}^\mathsf{T} \quad
b = \begin{pmatrix} 0 & 0 & 0 \end{pmatrix}^\mathsf{T} \quad
c = \begin{pmatrix} 1 & 1 & 1 \end{pmatrix}^\mathsf{T} \quad
g = \begin{pmatrix} 1 & 1 & 1 \end{pmatrix}^\mathsf{T}
\end{align*}
Очевидно что ответ \(u = \begin{pmatrix} 1 & 1 & 1 \end{pmatrix}^\mathsf{T}\)

Потом возьмем пример из задачника
\begin{align*}
a = \begin{pmatrix} 0 & 2 & 2 & 3 \end{pmatrix}^\mathsf{T} \quad
b = \begin{pmatrix} -1 & -1 & -0.8 & 0 \end{pmatrix}^\mathsf{T} \quad
c = \begin{pmatrix} 5 & 4,6 & 3,6 & 4,4 \end{pmatrix}^\mathsf{T} \\
g = \begin{pmatrix} 2 & 3,3 & 2,6 & 7,2 \end{pmatrix}^\mathsf{T} \quad
u = \begin{pmatrix} 0.5256 & 0.628 & 0.64 & 1.2 \end{pmatrix}^\mathsf{T}
\end{align*}

Для каждого из тестов посчитаем разницу полученного решения и истинного (u) и выберем наибольший по модулю элемент полученного вектора

1. 0

2. 1.1E-16

\subsubsection{Интегро-интерполяционный метод}

Выполним нашу программу на ряде входных параметров. Берем тестовые k, q, u, R, $\chi$, потом вычисляем f, $\nu$.

\begin{table}[H]
  \centering
  \begin{tabular}{ c | *{8}c }
    \toprule
    Номер теста & k & q & u & R & $\chi $ & f & $\nu $ \\
    \midrule
    1 & 3r & 4r+3 & 1 & 1 & 1 & 4r+3 & 1 \\
    \midrule
    2 & $ r+1 $ & $ 2r+2 $ & $ r^2 $ & 1 & 1 & $ 2r^3+2r^2-6r-4 $ & 5 \\
    \bottomrule
  \end{tabular}
\end{table}

Первый тест характеризуется отсутствием погрешности разностной схемы, а второй - присутствием.

После запуска программы мы получаем следующие результаты. Выпишем наибольшие по модулю элементы вектора разности истинного решения с полученным для разного значения N – количество отсчетов.

  \begin{table}[H]
    \centering
    \begin{tabular}{c | c | c}
      \toprule
      N & Тест 1 & Тест 2 \\
      \midrule
4 & 0.0E+00 & 4.1E-02 \\
8 & 0.0E+00 & 1.0E-02 \\
16 & 0.0E+00 & 2.6E-03 \\
32 & 0.0E+00 & 6.5E-04 \\
64 & 2.6E-05 & 1.4E-04 \\
128 & 4.3E-05 & 2.4E-05 \\
256 & 1.0E-05 & 3.9E-05 \\
512 & 6.1E-05 & 6.4E-05 \\
1024 & 8.9E-05 & 9.0E-05 \\
2048 & 6.0E-02 & 4.5E-02 \\
4096 & 1.0E-01 & 5.4E-02 \\
8192 & 3.4E-01 & 2.8E+00 \\

      \bottomrule
    \end{tabular}
    \caption{Наибольшие по модулю элементы вектора разности истинного решения с полученным (погрешности)}
  \end{table}

Видно, что для первого теста ошибок сначала нет, а потом они появляются и растут. Так происходит потому что ошибок разностной схемы то тут нет а решить СЛАУ для небольшого N совсем просто.

Для второго теста ошибка сначала падает, а потом растет. Можно заметить что рост начинается как раз когда ошибка доходит до порядка -5. Дальше она растет похожим на тест 1 образом.

Покажем во сколько раз убывают погрешности во втором тесте с ростом N.

  \begin{table}[H]
    \centering
    \begin{tabular}{c | c}
      \toprule
      N & Тест 2 \\
      \midrule
4 & 3.9 \\
8 & 4.0 \\
16 & 4.0 \\
32 & 4.7 \\
64 & 5.9 \\
128 & 0.6 \\
256 & 0.6 \\
512 & 0.7 \\
1024 & 0.0 \\
2048 & 0.8 \\
4096 & 0.0 \\

      \bottomrule
    \end{tabular}
    \caption{Отношения погрешностей (текущее к следующему)}
  \end{table}

Можно проследить зависимость погрешности от числа разбиений, которую мы выявили в предыдущей главе - при увеличении шага в 2 раза погрешность должна уменьшаться примерно в 4 раза.

Все вычисления проведены с одинарной точностью.

% На тесте 1 можно видеть линейное падение погрешности, так как в случае u=1 все производные начиная с первой равны нулю и $ h^3 \cdot 0 = 0 $. Остается только линейное слагаемое в формуле погрешности основного уравнения.
% Также можно проследить что есть число разбиений при котором погрешность минимальна. При дальнейшем увеличении числа разбиений погрешность увеличивается (данное явление можно заметить на тестах одинарной точности).

	\newpage

	\section{Заключение}
	\subsection{Вывод}
	Задание выполнено в полном объеме.
	Был написан метод приближенного решения краевой задачи и метод прогонки.
	Программа была протестирована на разных функциях.
	Была оценена погрешность и выявлена зависимость погрешности решения от числа разбиений.
	\newpage
	\subsection{Код}
	\inputminted{python}{main.py}
\end{document}
