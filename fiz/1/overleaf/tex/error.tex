\subsection{Оценка погрешности}

\subsubsection{Невязка разностной схемы}

\[
  Av = g;\ A - (N + 1) \cdot (N + 1); v, g \in R^{(N + 1)}
\]

Пусть $ v $ - это точное решение разностной схемы, $ u $ -  точное решение дифференциального уравнения, 
$ \tilde{v} $ - полученное решение разностной схемы.

Ищем погрешность решения разностной схемы:
\[
  \varepsilon = \tilde{v} - u
\]

Введем обозначения:
\begin{itemize}
  \item Погрешность решения системы линейных алгебраических уравнений
  \[ z = \tilde{v} - v \]
  \item Погрешность от аппроксимации дифференциального уравнения разностной схемой
  \[ \zeta = v - u \]
  \item Невязка разностной схемы
  \[ \xi = g - Au \]
  \item Невязка алгебраической системы
  \[ r = g - A\tilde{v} \]
\end{itemize}

\subsubsection{Структура погрешности разностной схемы}
\[
  \left\lVert \varepsilon \right\rVert = \left\lVert \tilde{v} - u \right\rVert =
  \left\lVert \tilde{v} - v + v - u \right\rVert = \left\lVert z + \zeta  \right\rVert \leq \left\lVert z \right\rVert
  + \left\lVert \zeta \right\rVert 
\]

Для $\left\lVert \zeta\right\rVert$:
\[
  \xi = g - Au = A(A^{-1}g - u) = A(v - u)
\]
\[
  A\zeta  = \xi
\]

Тем самым погрешность от аппроксимации дифференциального уравнения разностной схемой, связана с невязкой разностной схемы:
\[
  \zeta = A^{-1} \xi 
\]

Для $\left\lVert z \right\rVert$:
\[
  r = g - A\tilde{v} = A(A^{-1}g - \tilde{v}) = A(v - \tilde{v})
\]
\[
  Az = r
\]
Тем самым погрешность решения системы линейных алгебраических уравнений, связана с невязкой алгебраической системы:
\[
  z = A^{-1}r
\]

Подставим в наше неравенство, тем самым получаем:
\[
  \left\lVert \varepsilon \right\rVert \leq \left\lVert A^{-1}r \right\rVert + \left\lVert A^{-1}\xi \right\rVert
  \leq \left\lVert A^{-1} \right\rVert ( \left\lVert r \right\rVert  + \left\lVert \xi \right\rVert)
  \quad \left\lVert A^{-1}\right\rVert  < C
\]

\subsubsection{Вклад от погрешности решения системы алгебраических уравнений}
\[
  \left\lVert z \right\rVert \leq \left\lVert A^{-1} \right\rVert \left\lVert r \right\rVert =
  \left\lVert A \right\rVert \left\lVert A^{-1} \right\rVert \frac{\left\lVert r \right\rVert }{\left\lVert A \right\rVert } 
\]

Знаем что:
\[
  \left\lVert A \right\rVert \geq \frac{\left\lVert g \right\rVert }{\left\lVert v\right\rVert }
\]

Из этого получаем:
\[
  \left\lVert z \right\rVert \leq \left\lVert A \right\rVert \left\lVert A^{-1} \right\rVert
  \frac{\left\lVert r \right\rVert }{\left\lVert A \right\rVert } \left\lVert v \right\rVert 
\]
\[
  cond(A) = \left\lVert A \right\rVert \left\lVert A^{-1} \right\rVert
\]
\[
  \frac{\left\lVert r \right\rVert }{\left\lVert A \right\rVert } \sim \varepsilon_{M}
\]


\[
  \left\lVert z \right\rVert \leq cond(A) \varepsilon_{M} \left\lVert v \right\rVert 
\]

Чтобы оценить поведение числа обусловленности, можно взять простую модельную квадратичную задачу, модифицированную для поиска собственных значений её матрицы А.
\[
  -\frac{d^2u}{dx} = \lambda u(x), x \in [0, L], u(0) = u(L) = 0
\]
% \begin{align*}
%     -\frac{d^2u}{dx} = \lambda u(x) & x \in [0, L] & u(0) = 0  u(L) = 0
% \end{align*}
Далее можно решить и получить собственные значения с помощью численного решения через разностную схему. Окажется, что собственные значения неотрицательны и отношение наибольшего к наименьшему примерно пропорционально $ N^2 $. При этом число обусловенности \( cond(A) = \frac{|\lambda|_{max}}{|\lambda|_{min}} \). Тогда погрешность $ \left\lVert z \right\rVert \sim \frac{1}{h^2} $. Используем это и в качестве грубой оценки погрешности для других задач.

\subsubsection{Разложение невязки}

Введём букву n = 1.

Граничное условие слева:
\begin{center}
  \includegraphics[width=0.7\textwidth]{img/ei0.png}
\end{center}
p = 2 - 0 = 2

Граничное условие справа:
\begin{center}
  \includegraphics[width=0.7\textwidth]{img/eiN.png}
\end{center}
Второе слагаемое из условия равно нулю.
p = 2 - 0 = 2

Основное уравнение:
\begin{center}
  \includegraphics[width=0.7\textwidth]{img/emid.png}
\end{center}
p = 3 - 1 = 2

\subsubsection{Зависимость погрешности от числа разбиений}

Выходит, что во всех случаях невязка $\left\lVert \xi \right\rVert \sim h^2$. А вот  погрешность решения системы алгебраических уравнений $ \left\lVert z \right\rVert \sim \frac{1}{h^2} $.
\begin{center}
  \includegraphics[width=0.7\textwidth]{img/err.pdf}
\end{center}

Значит существует такое число разбиений где погрешность минимальная.