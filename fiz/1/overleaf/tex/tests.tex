\subsection{Тестирование}

\subsubsection{Метод прогонки}

Сначала протестируем на простом случае
\begin{align*}
a = \begin{pmatrix} 0 & 0 & 0 \end{pmatrix}^\mathsf{T} \quad
b = \begin{pmatrix} 0 & 0 & 0 \end{pmatrix}^\mathsf{T} \quad
c = \begin{pmatrix} 1 & 1 & 1 \end{pmatrix}^\mathsf{T} \quad
g = \begin{pmatrix} 1 & 1 & 1 \end{pmatrix}^\mathsf{T}
\end{align*}
Очевидно что ответ \(u = \begin{pmatrix} 1 & 1 & 1 \end{pmatrix}^\mathsf{T}\)

Потом возьмем пример из задачника
\begin{align*}
a = \begin{pmatrix} 0 & 2 & 2 & 3 \end{pmatrix}^\mathsf{T} \quad
b = \begin{pmatrix} -1 & -1 & -0.8 & 0 \end{pmatrix}^\mathsf{T} \quad
c = \begin{pmatrix} 5 & 4,6 & 3,6 & 4,4 \end{pmatrix}^\mathsf{T} \\
g = \begin{pmatrix} 2 & 3,3 & 2,6 & 7,2 \end{pmatrix}^\mathsf{T} \quad
u = \begin{pmatrix} 0.5256 & 0.628 & 0.64 & 1.2 \end{pmatrix}^\mathsf{T}
\end{align*}

Для каждого из тестов посчитаем разницу полученного решения и истинного (u) и выберем наибольший по модулю элемент полученного вектора

1. 0

2. 1.1E-16

\subsubsection{Интегро-интерполяционный метод}

Выполним нашу программу на ряде входных параметров. Берем тестовые k, q, u, R, $\chi$, потом вычисляем f, $\nu$.

\begin{table}[H]
  \centering
  \begin{tabular}{ c | *{8}c }
    \toprule
    Номер теста & k & q & u & R & $\chi $ & f & $\nu $ \\
    \midrule
    1 & $ 3r+2 $ & $ 4r+3 $ & 1 & 1 & 1 & $ 4r+3 $ & 1 \\
    \midrule
    2 & $ r+1 $ & $ 2r+2 $ & $ r^2 $ & 1 & 1 & $ 2r^3+2r^2-6r-4 $ & 5 \\
    \bottomrule
  \end{tabular}
\end{table}

Первый тест характеризуется отсутствием погрешности разностной схемы, а второй - присутствием.

После запуска программы мы получаем следующие результаты. Выпишем наибольшие по модулю элементы вектора разности истинного решения с полученным для разного значения N – количество отсчетов.

  \begin{table}[H]
    \centering
    \begin{tabular}{c | c | c}
      \toprule
      N & Тест 1 & Тест 2 \\
      \midrule
4 & 0.0E+00 & 4.1E-02 \\
8 & 0.0E+00 & 1.0E-02 \\
16 & 0.0E+00 & 2.6E-03 \\
32 & 0.0E+00 & 6.5E-04 \\
64 & 5.7E-05 & 1.4E-04 \\
128 & 2.8E-05 & 2.4E-05 \\
256 & 7.1E-05 & 3.9E-05 \\
512 & 1.8E-04 & 6.4E-05 \\
1024 & 2.2E-02 & 9.0E-05 \\
2048 & 5.2E-02 & 4.5E-02 \\
4096 & 5.4E-02 & 5.4E-02 \\
8192 & 4.2E-01 & 2.8E+00 \\

      \bottomrule
    \end{tabular}
    \caption{Наибольшие по модулю элементы вектора разности истинного решения с полученным (погрешности)}
  \end{table}

Поведение погрешности в тесте 1 можно объяснить необходимостью решения системы алгебраических уравнений.

Поведение погрешности в тесте 2 можно объяснить тем, что сначала действует погрешность разностной схемы, а потом начинает проявляться погрешность решения системы алгебраических уравнений.

Покажем во сколько раз убывают погрешности во втором тесте с ростом N.

  \begin{table}[H]
    \centering
    \begin{tabular}{c | c}
      \toprule
      N & Тест 2 \\
      \midrule
4 & 3.9 \\
8 & 4.0 \\
16 & 4.0 \\
32 & 4.7 \\
64 & 5.9 \\
128 & 0.6 \\
256 & 0.6 \\
512 & 0.7 \\
1024 & 0.0 \\
2048 & 0.8 \\
4096 & 0.0 \\

      \bottomrule
    \end{tabular}
    \caption{Отношения погрешностей (текущее к следующему)}
  \end{table}

Можно проследить зависимость погрешности от числа разбиений, которую мы выявили в предыдущей главе - при увеличении шага в 2 раза погрешность должна уменьшаться примерно в 4 раза.

Вычисления элементов матрицы проведены с двойной точностью, приведены к одинарной и снова к двойной. Вычисления коэффициентов метода прогонки проведены с двойной точностью, а результат v приведен к одинарной.
